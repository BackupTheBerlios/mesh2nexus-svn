\clearpage
\subsubsection{MiniPCI(e) WLAN-Karten}

MiniPCI(e) ist eine vor allem f"ur die Nutzung in Notebooks und Laptops
miniaturisierte Version des PCI Steckplatzes, wie er in allen Desktop
PCs vorkommt. Die Abmessungen einer MiniPCI Karte betragen 6,0 x 4,6 x 0,5 cm.
Die Abmessungen einer MiniPCIe Karte betragen 3 cm x 5 cm x 0.4 cm.

MiniPCI(e) WLAN-Karten sind urspr"unglich f"ur Laptops gedacht, k"onnen aber
mit entschprechenden Adaptern (MiniPCI(e)-to-PCI) und externen Antennen auch
in normalen PCs verwendet werden.

Im folgenden werden Vorteile und Nachteile von MiniPCI(e) WLAN-Karten
erl"autert.

\textbf{Vorteile:}

\begin{itemize}
\item K"onnen mit Hilfe eines Adapters zu einer PCI WLAN-Karte umgebaut werden
\item Leicht austauschbar
\item Sehr gute Treiber-Unterst"utzung unter Linux und Windows
\end{itemize}

\textbf{Nachteile:}

\begin{itemize}
\item Brauchen einen PCI-Adapter f"ur den PCI-Bus
\item Haben keine Antenne (extra Kosten)
\end{itemize}

Es wurden nur zwei MiniPCI und zwei MiniPCIe WLAN-Karten gefunden,
die den Ad-Hoc Modus im 5 GHz Frequenzband unterst"utzen. Diese WLAN-Karten
basieren entweder auf Intel Chips"atzen oder Atheros Chips"atzen.

%%%%%%%%%%%%%%%%%%%%%%%%%%%%%%%%%%%%%%%%%%%%%%%%%%%%%%%%%%%%%%%%%%%%%%%%%%%%
%
% Wistron CM9 Atheros AR5213A
%
%%%%%%%%%%%%%%%%%%%%%%%%%%%%%%%%%%%%%%%%%%%%%%%%%%%%%%%%%%%%%%%%%%%%%%%%%%%%
\begin{wlandevice}{Wistron CM9 Atheros AR5213A}

\wlanimage{Wistron_CM9}{Wistron CM9 Atheros AR5213A}

\wlanchipset{Atheros AR5213A}

\begin{wlanieeestandard}
\item 802.11a/b/g
\end{wlanieeestandard}

\begin{wlanmode}
\item Ad-Hoc
\item Infrastruktur
\end{wlanmode}

\begin{wlansecurity}
\item WEP (40-, 104-, 128-bit)
\item WPA
\item WPA2
\end{wlansecurity}

\begin{wlandriver}
\item
Herrvorragende Unterst"utzung von MadWifi-Treiber \cite{madwifi},
auch Ad-Hoc-Modus.
\end{wlandriver}

\wlanprice{40}

\begin{wlaninstall}
\item
\url{http://madwifi.org/wiki/UserDocs/FirstTimeHowTo}
\end{wlaninstall}

\begin{wlanlink}
\item \url{http://www.alix-board.de/produkte/wistroncm9.html}
\item \url{http://www.pcengines.ch/cm9.htm}
\item \url{http://forum.openwrt.org/viewtopic.php?pid=10213}
\item \url{http://madwifi.org/ticket/1209}
\end{wlanlink}

\end{wlandevice}

%%%%%%%%%%%%%%%%%%%%%%%%%%%%%%%%%%%%%%%%%%%%%%%%%%%%%%%%%%%%%%%%%%%%%%%%%%%%
%
% Intel PRO/Wireless 3945
%
%%%%%%%%%%%%%%%%%%%%%%%%%%%%%%%%%%%%%%%%%%%%%%%%%%%%%%%%%%%%%%%%%%%%%%%%%%%%
\begin{wlandevice}{Intel PRO/Wireless 3945}

\wlanimage{Intel_3945ABG}{Intel PRO/Wireless 3945}

\wlanchipset{Intel}

\begin{wlanieeestandard}
\item 802.11a/b/g
\end{wlanieeestandard}

\begin{wlanmode}
\item Ad-Hoc
\item Infrastruktur
\end{wlanmode}

\begin{wlansecurity}
\item WEP (40-, 104-bit)
\item WPA
\item WPA2
\end{wlansecurity}

\begin{wlandriver}
\item
Es werden von Intel Treiber sowohl f"ur Windows als auch f"ur Linux
bereitgestellt.

\url{http://downloadcenter.intel.com/Product_Filter.aspx?ProductID=2259}

Von Intel wurde ein Projket f"ur die Unterst�tzung von Intel PRO/Wireless
3945 erstellt.

\url{http://ipw3945.sourceforge.net}

Der ipw3945-Treiber funktioniert auch im Ad-Hoc-Modus, aber nicht sehr stabil,
es kommt oft zu Verbindungsabbr"uchen.
\end{wlandriver}

\wlanprice{20-30}

\begin{wlaninstall}
\item
Im Gegensatz zu den "`klassischen"' Intel Wireless-Chips"atzen 2100- und
2200BG-Chips"atzen ist der Treiber f"ur den 3945ABG noch nicht im Kernel
verf"ugbar. Um auch damit kabellos ins Internet zu gehen,
sind ein paar Handgriffe notwendig.

\url{http://ipw3945.sourceforge.net/README.ipw3945}

\url{http://ipw3945.sourceforge.net/INSTALL}
\end{wlaninstall}

\begin{wlanlink}
\item \url{http://www.intel.com/network/connectivity/products/wireless/prowireless_mobile.htm}
\item \url{http://downloadcenter.intel.com/Product_Filter.aspx?ProductID=2259}
\item \url{http://ipw3945.sourceforge.net/}
\item \url{http://ipw3945.sourceforge.net/README.ipw3945}
\item \url{http://ipw3945.sourceforge.net/INSTALL}
\end{wlanlink}

\end{wlandevice}

%%%%%%%%%%%%%%%%%%%%%%%%%%%%%%%%%%%%%%%%%%%%%%%%%%%%%%%%%%%%%%%%%%%%%%%%%%%%
%
% Intel PRO/Wireless 2915
%
%%%%%%%%%%%%%%%%%%%%%%%%%%%%%%%%%%%%%%%%%%%%%%%%%%%%%%%%%%%%%%%%%%%%%%%%%%%%
\begin{wlandevice}{Intel PRO/Wireless 2915}

\wlanimage{Intel_2915ABG}{Intel PRO/Wireless 2915}

\wlanchipset{Intel}

\begin{wlanieeestandard}
\item 802.11a/b/g
\end{wlanieeestandard}

\begin{wlanmode}
\item Ad-Hoc
\item Infrastruktur
\end{wlanmode}

\begin{wlansecurity}
\item WEP (40-, 104-bit)
\item WPA
\item WPA2
\end{wlansecurity}

\begin{wlandriver}
\item
Es werden von Intel Treiber sowohl f"ur Windows als auch f"ur Linux
bereitgestellt.

\url{http://downloadcenter.intel.com/Product_Filter.aspx?ProductID=1847}

Von Intel wurde ein Projket f"ur die Unterst"utzung von Intel PRO/Wireless
2915 erstellt.

\url{http://ipw2200.sourceforge.net}

Der ipw2200-Treiber funktioniert auch im Ad-Hoc-Modus, aber nicht
sehr stabil, es kommt oft zu verbindungsabbr�chen. Der ipw2200-Treiber
ist im Kernel 2.6 enthalten, kann aber auch separat als Modul kompiliert
werden. Der im Kernel enthaltene Treiber unterst"utzt den Monitor-Modus
nicht.
\end{wlandriver}

\wlanprice{30}

\begin{wlaninstall}
\item
\url{http://ipw2200.sourceforge.net/README.ipw2200}

\url{http://ipw2200.sourceforge.net/INSTALL}
\end{wlaninstall}

\begin{wlanlink}
\item \url{http://support.intel.com/support/wireless/wlan/pro2915abg}
\item \url{http://download.intel.com/support/wireless/wlan/pro2915abg/sb/303330002us_channel.pdf}
\item \url{http://ipw2200.sourceforge.net/}
\item \url{http://www.intel.com/cd/personal/computing/emea/deu/234998.htm}
\item \url{http://downloadcenter.intel.com/Product_Filter.aspx?ProductID=1847}
\end{wlanlink}

\end{wlandevice}

%%%%%%%%%%%%%%%%%%%%%%%%%%%%%%%%%%%%%%%%%%%%%%%%%%%%%%%%%%%%%%%%%%%%%%%%%%%%
%
% Intel Wireless WiFi Link 4965AGN
%
%%%%%%%%%%%%%%%%%%%%%%%%%%%%%%%%%%%%%%%%%%%%%%%%%%%%%%%%%%%%%%%%%%%%%%%%%%%%
\begin{wlandevice}{Intel Wireless WiFi Link 4965AGN}

\wlanimage{Intel_4965AGN}{Intel Wireless WiFi Link 4965AGN}

\wlanchipset{Intel}

\begin{wlanieeestandard}
\item 802.11a/b/g/n(draft)
\end{wlanieeestandard}

\begin{wlanmode}
\item Ad-Hoc
\item Infrastruktur
\end{wlanmode}

\begin{wlansecurity}
\item WEP (40-, 104-bit)
\item WPA
\item WPA2
\end{wlansecurity}

\begin{wlandriver}
\item
\url{http://www.intellinuxwireless.org/}
\end{wlandriver}

\wlanprice{30}

\begin{wlaninstall}
\item
\url{http://www.intellinuxwireless.org/}
\end{wlaninstall}

\begin{wlanlink}
\item \url{http://www.intel.com/network/connectivity/products/wireless/wireless_n/overview.htm}
\item \url{http://www.intellinuxwireless.org/}
\item \url{http://downloadcenter.intel.com/filter_results.aspx?strTypes=all&ProductID=2753&OSFullName=Linux*&lang=eng&strOSs=39&submit=Go\%21}
\end{wlanlink}

\end{wlandevice}
