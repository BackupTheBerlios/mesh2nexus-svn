\subsubsection{PCI-WLAN-Karten}

TODO: \todo{Ziel des Abschnittes! Anfang ist schlecht..}

Peripheral Component Interconnect, meist PCI abgek"urzt, ist ein
Bus-Standard zur Verbindung von Peripherieger"aten mit dem Chipsatz
eines Prozessors. Da der PCI-Bus vom Prozessor relativ unabh"angig ist,
wird er nicht nur im PC benutzt, sondern z.B. auch im Alpha-PC oder
im Macintosh. "Uber den PCI-Bus kann der Prozessor die wichtigsten Ein-
und Ausgabekomponenten (z.B. Controller, Grafikkarte) "`lokal"' und damit
schneller ansprechen. Urspr"unglich sollte der PCI-Bus die Anforderungen in
PCs f"ur Grafik-, Netzwerk- und andere Schnittstellenkarten "uber l"angere
Zeit erf"ullen. Allerdings wurde er schon nach kurzer Zeit zu langsam f"ur
schnelle Grafikkarten, so dass f"ur diese 1997 ein zus"atzlicher Steckplatz,
der Accelerated Graphics Port (AGP), eingef"uhrt wurde. F"ur so gut wie
alle anderen Steckkarten-Typen blieb PCI dagegen bis heute Standard, soll
aber ab \textbf{2005} schrittweise von PCI-Express ersetzt werden.
TODO: \todo{heute ist 2005)) }

PCI-WLAN-Karten werden in einen freien PCI-Steckplatz des Mainboards
gesteckt.
Ein Vorteil von PCI-WLAN-Karten ist die bessere Stabilit"at 
TODO: \todo{gegenuber was..}im
Betrieb. Weiterhin besitzen die meisten PCI-WLAN-Karten die M"oglichkeit
die mitgelieferte Antenne gegen eine andere zu tauschen. Zu beachten ist,
dass die Antenne "ublicherweise direkt hinten an der Karte angebracht
ist und somit in unmittelbarer N"ahe zum PC-Geh"ause ist. Dies kann
jedoch negative Auswirkungen auf die Reichweite oder den Datendurchsatz
haben. Deshalb kann es f"ur einen bessere Verbindung notwendig sein,
die Antenne mit einem Koaxialkabels vom Rechnergeh"ause zu entfernen.

\textbf{Vorteile:}

\begin{itemize}
	\item Bessere Stabilit"at im Betrieb
	\item Meistens abschraubbare Antenne
	\item Verschwinden im Geh"ause, Platz wird nicht verschwendet 
\end{itemize}

\textbf{Nachteile:}

\begin{itemize}
	\item Oft recht schlechte Empfangs/Sendeleistung, weil die kleine Antenne
				direkt hinten am Rechner sitzt (L"osung: zus"atzliche Antenne)
\end{itemize}

TODO: \todo{Allgemein: Welche chipsatze in Frage kommen (atheros, intel..), Diskussion der Karten..}
TODO: \todo{Sicherheit - erlaeterung!}
%%%%%%%%%%%%%%%%%%%%%%%%%%%%%%%%%%%%%%%%%%%%%%%%%%%%%%%%%%%%%%%%%%%%%%%%%%%%
%
% Linksys WMP55AG
%
%%%%%%%%%%%%%%%%%%%%%%%%%%%%%%%%%%%%%%%%%%%%%%%%%%%%%%%%%%%%%%%%%%%%%%%%%%%%
\begin{wlandevice}{Linksys WMP55AG}

\wlanimage{Linksys_WMP55AG}{Linksys WMP55AG}

\wlanchipset{Atheros AR5213A}

\begin{wlanieeestandard}
\item 802.11a/b/g
\end{wlanieeestandard}

\begin{wlanmode}
\item Ad-Hoc
\item Infrastruktur
\end{wlanmode}

\begin{wlansecurity}
\item WEP (40-, 104-, 128-bit)
\item WPA
\item LEAP
\end{wlansecurity}

\begin{wlandriver}
\item
Sehr gute Linux-Unterst"utzung, MadWifi-Treiber funktioniert
mit dieser WLAN PCI-Karte ohne Probleme.
Windows-Treiber werden von Linksys bereitgestellt.
\end{wlandriver}

\wlanprice{90}

\begin{wlaninstall}
\item
Lasst sich leicht sowohl unter Windows als auch unter Linux (MadWifi-Treiber)
installieren.

\url{http://madwifi.org/wiki/UserDocs/FirstTimeHowTo}
\end{wlaninstall}

\begin{wlanlink}
\item \url{http://madwifi.org/wiki/Compatibility/Linksys}
\item \url{http://forums.fedoraforum.org/showthread.php?t=91165}
\item \url{http://www.pcworld.com/product/specs/prtprdid,704176/%
	wireless_ag_54mbps_pci_adptr_80211a80211b80211g_compatible.html}
\item \url{http://www.linksys.com/servlet/Satellite?c=L_CASupport_C2&childpagename=US\%2FLayout&cid=1169671168007&pagename=Linksys\%2FCommon\%2FVisitorWrapper&lid=6800768007N09}
\end{wlanlink}

\end{wlandevice}

%%%%%%%%%%%%%%%%%%%%%%%%%%%%%%%%%%%%%%%%%%%%%%%%%%%%%%%%%%%%%%%%%%%%%%%%%%%%
%
% Netgear WAG311
%
%%%%%%%%%%%%%%%%%%%%%%%%%%%%%%%%%%%%%%%%%%%%%%%%%%%%%%%%%%%%%%%%%%%%%%%%%%%%
\begin{wlandevice}{Netgear WAG311}

\wlanimage{Netgear_WAG311}{Netgear WAG311}

\wlanchipset{Atheros AR5212}

\begin{wlanieeestandard}
\item 802.11a/b/g
\end{wlanieeestandard}

\begin{wlanmode}
\item Ad-Hoc
\item Infrastruktur
\end{wlanmode}

\begin{wlansecurity}
\item WEP (40-, 104-, 128-bit)
\item WPA, WPA-PSK
\item PPTP, P2TP, IPSec VPN pass-through
\end{wlansecurity}

\begin{wlandriver}
\item
Sehr gute Linux-Unterst"utzung, MadWifi-Treiber funktioniert
mit dieser WLAN PCI-Karte ohne Probleme.
\end{wlandriver}

\wlanprice{50-60}

\begin{wlaninstall}
\item
\url{http://madwifi.org/wiki/UserDocs/FirstTimeHowTo}

\url{http://www.packetpro.com/~peterson/linux-netgear_wg311t_pci.html}
\end{wlaninstall}

\begin{wlanextrainfo}
\item
Externe Antenne, die mit der WLAN-PCI-Karte durch langes Kabel verbunden ist.
Das Kabel l"asst sich nicht von der PCI-Karte trennen.
\end{wlanextrainfo}

\begin{wlanlink}
\item \url{http://www.netgear.com/Products/Adapters/AGDualBandWirelessAdapters/WAG311.aspx}
\item \url{http://madwifi.org/wiki/Compatibility/Netgear}
\item \url{http://www.linuxquestions.org/questions/mandriva-30/using-netgear-wag311-via-madwifi-270229/}
\item \url{http://www.packetpro.com/~peterson/linux-netgear_wg311t_pci.html}
\item \url{http://www.netgear.com/upload/product/wag311/enus_ds_wag311.pdf}
\end{wlanlink}

\end{wlandevice}

%%%%%%%%%%%%%%%%%%%%%%%%%%%%%%%%%%%%%%%%%%%%%%%%%%%%%%%%%%%%%%%%%%%%%%%%%%%%
%
% D-Link DWL-A520
%
%%%%%%%%%%%%%%%%%%%%%%%%%%%%%%%%%%%%%%%%%%%%%%%%%%%%%%%%%%%%%%%%%%%%%%%%%%%%
\begin{wlandevice}{D-Link DWL-A520}

\wlanimage{DLink_DWLA520}{D-Link DWL-A520}

\wlanchipset{Atheros AR5210}

\begin{wlanieeestandard}
\item 802.11a
\end{wlanieeestandard}

\begin{wlanmode}
\item Ad-Hoc
\item Infrastruktur
\end{wlanmode}

\begin{wlansecurity}
\item WEP (40-, 104-, 128-bit)
\end{wlansecurity}

\begin{wlandriver}
\item
Von D-Link werden nur Treiber f"ur Windows bereitgestellt.
Sehr gute Linux-Unterst"utzung, MadWifi-Treiber funktioniert
mit dieser WLAN PCI-Karte ohne Probleme.
\end{wlandriver}

\wlanprice{70-80}

\begin{wlaninstall}
\item
\url{http://madwifi.org/wiki/UserDocs/FirstTimeHowTo}
\end{wlaninstall}

\begin{wlanextrainfo}
\item
Antenne ist nicht abschraubbar.
\end{wlanextrainfo}

\begin{wlanlink}
\item \url{http://support.dlink.com/products/print.asp?productid=DWL-A520}
\item \url{http://madwifi.org/wiki/Compatibility/D-Link}
\end{wlanlink}

\end{wlandevice}

%%%%%%%%%%%%%%%%%%%%%%%%%%%%%%%%%%%%%%%%%%%%%%%%%%%%%%%%%%%%%%%%%%%%%%%%%%%%
%
% Gigabyte GN-WPEAG
%
%%%%%%%%%%%%%%%%%%%%%%%%%%%%%%%%%%%%%%%%%%%%%%%%%%%%%%%%%%%%%%%%%%%%%%%%%%%%
\begin{wlandevice}{Gigabyte GN-WPEAG}

\wlanimage{Gigabyte_GNWPEAG}{Gigabyte GN-WPEAG}

\wlanchipset{Atheros AR5212}

\begin{wlanieeestandard}
\item 802.11a/b/g
\end{wlanieeestandard}

\begin{wlanmode}
\item Ad-Hoc
\item Infrastruktur
\end{wlanmode}

\begin{wlansecurity}
\item WEP (40-, 104-, 128-bit)
\item WPA
\item WPA2
\end{wlansecurity}

\begin{wlandriver}
\item
Von Gigabyte werden nur Treiber f"ur Windows bereitgestellt.

\url{http://www.gigabyte.com.tw/Support/Communication/Driver_Model.aspx?ProductID=952}

Sehr gute Linux-Unterst"utzung, MadWifi-Treiber funktioniert
mit dieser WLAN PCI-Karte ohne Probleme.

\url{http://madwifi.org/wiki/UserDocs/FirstTimeHowTo}
\end{wlandriver}

\wlanprice{70-80}

\begin{wlaninstall}
\item \url{http://madwifi.org/wiki/UserDocs/FirstTimeHowTo}
\end{wlaninstall}

\begin{wlanextrainfo}
\item
Abschraubbare Antenne mit reversed SMA.
Eigentlich ist das eine Mini-PCI-Karte mit PCI-Adapter.
\end{wlanextrainfo}

\begin{wlanlink}
\item \url{http://www.gigabyte.com.tw/Products/Communication/Products_Spec.aspx?ProductID=952}
\item \url{http://www.gigabyte.com.tw/Support/Communication/Driver_Model.aspx?ProductID=952}
\item \url{http://madwifi.org/wiki/Compatibility/Gigabyte}
\end{wlanlink}

\end{wlandevice}

%%%%%%%%%%%%%%%%%%%%%%%%%%%%%%%%%%%%%%%%%%%%%%%%%%%%%%%%%%%%%%%%%%%%%%%%%%%%
%
% D-Link DWL-G550
%
%%%%%%%%%%%%%%%%%%%%%%%%%%%%%%%%%%%%%%%%%%%%%%%%%%%%%%%%%%%%%%%%%%%%%%%%%%%%
\begin{wlandevice}{D-Link DWL-G550}

\wlanimage{DLink_DWLG550.jpg}{D-Link DWL-G550}

\wlanchipset{Atheros AR5212}

\begin{wlanieeestandard}
\item 802.11a/b/g
\end{wlanieeestandard}

\begin{wlanmode}
\item Ad-Hoc
\item Infrastruktur
\end{wlanmode}

\begin{wlansecurity}
\item WEP (40- and 104-bit)
\item WPA
\item WPA2
\end{wlansecurity}

\begin{wlandriver}
\item
Von D-Link werden nur Treiber f"ur Windows bereitgestellt. Sehr gute Linux-
Unterst"utzung, MadWifi-Treiber funktioniert mit dieser WLAN PCI-Karte
ohne Probleme.
\end{wlandriver}

\wlanprice{60}

\begin{wlaninstall}
\item \url{http://madwifi.org/wiki/UserDocs/FirstTimeHowTo}
\end{wlaninstall}

\begin{wlanextrainfo}
\item
Die Karte hat eine externe abschraubbare Antenne.
\end{wlanextrainfo}

\begin{wlanlink}
\item \url{http://www.dlink.com/products/?pid=414}
\item \url{http://madwifi.org/wiki/Compatibility/D-Link}
\end{wlanlink}

\end{wlandevice}

%%%%%%%%%%%%%%%%%%%%%%%%%%%%%%%%%%%%%%%%%%%%%%%%%%%%%%%%%%%%%%%%%%%%%%%%%%%%
%
% D-Link DWL-AG530
%
%%%%%%%%%%%%%%%%%%%%%%%%%%%%%%%%%%%%%%%%%%%%%%%%%%%%%%%%%%%%%%%%%%%%%%%%%%%%
\begin{wlandevice}{D-Link DWL-AG530}

\wlanimage{DLink_DWLAG530.jpg}{D-Link DWL-AG530}

\wlanchipset{Atheros AR5212 oder AR5213}

\begin{wlanieeestandard}
\item 802.11a/b/g
\end{wlanieeestandard}

\begin{wlanmode}
\item Ad-Hoc
\item Infrastruktur
\end{wlanmode}

\begin{wlansecurity}
\item WEP (40-, 104 and 128-bit)
\item WPA
\item WPA2
\end{wlansecurity}

\begin{wlandriver}
\item
Von D-Link werden nur Treiber f"ur Windows bereitgestellt. Sehr gute Linux-
Unterst"utzung, MadWifi-Treiber funktioniert mit dieser WLAN PCI-Karte
ohne Probleme.
\end{wlandriver}

\wlanprice{80}

\begin{wlaninstall}
\item \url{http://madwifi.org/wiki/UserDocs/FirstTimeHowTo}
\end{wlaninstall}

\begin{wlanextrainfo}
\item
Die Karte hat eine externe abschraubbare Antenne.
\end{wlanextrainfo}

\begin{wlanlink}
\item \url{http://www.dlink.com/products/?pid=306}
\item \url{http://madwifi.org/wiki/Compatibility/D-Link}
\end{wlanlink}

\end{wlandevice}

%%%%%%%%%%%%%%%%%%%%%%%%%%%%%%%%%%%%%%%%%%%%%%%%%%%%%%%%%%%%%%%%%%%%%%%%%%%%
%
% Intel PRO/Wireless 5000
%
%%%%%%%%%%%%%%%%%%%%%%%%%%%%%%%%%%%%%%%%%%%%%%%%%%%%%%%%%%%%%%%%%%%%%%%%%%%%

\begin{wlandevice}{Intel PRO/Wireless 5000}

\wlanimage{Intel_5000.jpg}{Intel PRO/Wireless 5000}

\wlanchipset{Intel}

\begin{wlanieeestandard}
\item 802.11a
\end{wlanieeestandard}

\begin{wlanmode}
\item Ad-Hoc
\item Infrastruktur
\end{wlanmode}

\begin{wlansecurity}
\item WEP (40- and 104-bit)
\end{wlansecurity}

\begin{wlandriver}
\item
Von Intel werden nur Treiber f"ur Windows bereitgestellt. Zur Zeit existiert
kein Linux-Treiber f"ur diese Karte.
\end{wlandriver}

\wlanprice{200}

\begin{wlaninstall}
\item \url{http://support.intel.com/support/wireless/wlan/pro5000/pciadapter/}
\end{wlaninstall}

\begin{wlanlink}
\item \url{http://support.intel.com/support/wireless/wlan/pro5000/pciadapter}
\item \url{ftp://download.intel.com/support/wireless/wlan/pro5000/PRO5000_INFO.pdf}
\end{wlanlink}

\end{wlandevice}
