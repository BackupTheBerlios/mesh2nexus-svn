\subsubsection{Professionelle Router}

In diesem Abschnitt werden sogenannte professionelle Mesh-Router betrachtet. 
Die Begriffe, die daf"ur oft als Synonyme verwendet werden, sind dabei: 

\begin{itemize}	
	\item Routerboards
	\item Stand-alone Mesh-Router 
	\item Minicomputers 
	\item Single-Board-Computers (SBC) 
	\item Access Points (AP)	
\end{itemize}

TODO: \todo{ueberarbeiten und SEPARATE diskussion in eigenem Abschnitt}

\textbf{Im Projekt (\url{http://umic-mesh.net}) wurden professionelle Router
eingesetzt, das sind spezielle Router-Boards mit Steckpl"atzen f"ur
MiniPCI WLAN-karten. Boards kosten etwa 100-200 Euro, dazu muss man allerdings
noch passende WLAN-Karten kaufen + Antennen + Kabel + Netzteil + Geh"ause,
also keine billige L"osung. }

TODO: \todo{identisch zur bereits diskutierten PC-Loesung?}
\emph{Man k"onnte aber nur diese Karten kaufen + Adapter PCI-MiniPCI und in Rechner
einbauen (Das w"are dann die platzsparende Version von \emph{PCs + WLAN-Karte}).
WLAN-Karten z.B Wistron Neweb CM9 Atheros 802.11a/b/g Mini-PCI, hier 
(\url{http://www.pcengines.ch/cm9.htm}), und Boards sind hier 
(\url{http://www.pcengines.ch/wrap.htm, http://www.pcengines.ch/alix.htm}).}

\textbf{Es gibt noch diese kleine Mesh-Router, wie von Meraki. Die haben wohl
ihre eigene Firmware drin und eigene Routingprotokolle oder eigene
Implementierungen davon besser gesagt. Hier ein Paar, die 802.11a
unterst"utzen, sind aber outdoor, haben also gro"se Reichweiten.
Ob es sinnvoll ist, sie im Gebaude einzusetzen:}

\begin{itemize}
	\item Aphelion 3300AG Outdoor Wireless Access Point - 802.11a/b/g
	\item Aphelion 600AG/605AG Intelligenter sequentieller Wireless Access Point f"ur
				den Au"senbereich mit den Standards 802.11a/b/g 
				(\url{http://www.abcdata.de/abcdataneu/WLAN_MESH_Aphelion.php})
	\item PLANET MAP-2100 - indoor - sind aber zum Teil sehr teuer (1200 Euro)
\end{itemize}

\textbf{Vorteile:}

\begin{itemize}	
	\item Outdoor (in unserem Fall unrelevant) 
	\item Gro�e Reichweiten
\end{itemize}

\textbf{Nachteile:} 

\begin{itemize}	
	\item Sehr teuer
	\item Closed source
\end{itemize}

\textbf{Links:}

\begin{itemize}	
	\item \url{http://wiki.opennet-initiative.de/index.php/WRAP}
	\item \url{http://www.abcdata.de/abcdataneu/WLAN_MESH_Aphelion.php}
	\item \url{http://www.aerial.net/shop/product_info.php?cPath=33&products_id=351}
	\item \url{http://forum.openwrt.org/viewtopic.php?id=9655}
\end{itemize}

\subsubsection{Access Points}

Ein WLAN-Accesspoint ist der Verbindungspunkt eines kabelbasierten
Netzwerkes zu einem WLAN. Der Accesspoint ist Basisstation f"ur alle
WLAN-Clients, zu der sie eine drahtlose Verbindung aufbauen.
Sendet ein WLAN-Client Daten, die f"ur einen Empf"anger im kabelbasierten
Netzwerkteil bestimmt sind, so \emph{reicht} der Accesspoint diese Daten "uber
das Kabelnetz an den Empf"anger weiter. Weiterhin kann ein Accesspoint
auch mehrere WLAN-Clienten untereinander verbinden. Somit ist der Accesspoint
quasi ein kabelloser Switch. 

Dieser hat (je nach Austattung) einige der folgenden Optionen: 
\begin{itemize}
	\item Ein oder mehrere integrierte WLAN-Module
	\item Einen integrierten DHCP-Server 
	\item Umfangreiche Sicherheits- und Verschl"usselungsmoglichkeiten 
	(WEP, WPA und WPA2 dienen der Verschl"usselung der zu ubetragenden Daten ;
   MAC-Filter und SSID Optionen ;
   Einstellungen bez"uglich des Remotezugriffs)	
	\item Verschiedene Arbeitsmodi (Accesspoint (AP), 
				Bridge (Point-to-Point oder Point-to-Multipoint), Repeater, MESSID) 
\end{itemize}

Einige Access Points:
\begin{itemize}
\item Intel PRO/Wireless 5000 

\url{http://support.intel.com/support/wireless/wlan/pro5000/accesspoint}

\url{http://www.pcmag.com/article2/0,1759,5524,00.asp}

\item Linksys WAP55AG 

\url{http://www.tomsnetworking.de/content/aktuelles/news_beitrag/news/851/6/index.html}

\item NETGEAR WAB102 

\url{http://kbserver.netgear.com/products/WAB102.asp}

\url{http://reviews.cnet.com/wireless-access-points/netgear-wab102-802-11a/4505-3265\_7-20708150.html}

\url{http://archive.cert.uni-stuttgart.de/bugtraq/2003/12/msg00159.html}

\end{itemize}
