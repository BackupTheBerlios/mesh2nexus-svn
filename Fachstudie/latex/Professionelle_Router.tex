\clearpage
\subsubsection{Professionelle Router}

In diesem Abschnitt werden sogenannte professionelle Mesh-Router betrachtet. 
Die Begriffe, die daf"ur oft als Synonyme verwendet werden, sind dabei: 

\begin{itemize}	
\item Routerboards
\item Stand-alone Mesh-Router 
\item Minicomputers 
\item Single-Board-Computers (SBC) 
\item Access Points
\end{itemize}

Die meisten professionellen WLAN-Router sind sehr teuer und kommen deswegen
f"ur unsere Zwecke nicht in Frage. Viele dieser WLAN-Router verwenden auch
propritere Routing-Protokolle und sind deswegen f"ur Forschungszwecke
ungeeignet. Zus"atzlich sind diese professionellen WLAN-Router sehr
leistungsstark und werden eher im Aussenbereich verwendet und
sind f"ur ein Geb"aude ungeeignet. Ausserdem sind leistungsstarke
professionelle WLAN-Router im 5GHz Frequenzband in Deutschland
nicht zugelassen.

Es gibt allerdings auch SBCs, die sehr wohl als Mesh-Router in einem WMN
eingesetzt werden k"onnen. Das Projekt UMIC-Mesh (\url{http://umic-mesh.net})
verwendet solche SBCs f"ur WMNs. Die vom Projekt eingesetzten SBCs
haben 2 MiniPCI Slots, 128 MB Speicher, 233 MHz AMD Geode SC1100 CPU,
100 MB/s LAN-Schnittstelle und einen seriellen Port. Diese SBCs k"onnten
als alternative f"ur Mesh-Router auf PC-Basis eingesetzt werden, allerdings
sind sie nicht so flexibel und erweiterbar wie PCs.

Zusammengefasst sind professionelle WLAN-Router f"ur den Aufbau eines
WMN in einem Geb"aude ungeeignet und werden deswegen hier nicht weiter betrachtet.

\subsubsection{Access Points}

Ein Access Point ist der Verbindungspunkt eines kabelbasierten
Netzwerkes zu einem WLAN. Der Access Point ist eine Basisstation f"ur alle
WLAN-Clients, zu der sie eine drahtlose Verbindung aufbauen.
Sendet ein WLAN-Client Daten, die f"ur einen Empf"anger im kabelbasierten
Netzwerkteil bestimmt sind, so reicht der Access Point diese Daten "uber
das Kabelnetz an den Empf"anger weiter. Weiterhin kann ein Access Point
auch mehrere WLAN-Clienten untereinander verbinden. Somit ist der Access Point
quasi ein kabelloser Switch.

Access Points kommen als Mesh-Router nicht in Frage, denn Access Points
werden in der Regel nur im Infrastruktur Modus eingesetzt. Um ein Access
Point auch im Ad-Hoc Modus zu betreiben, wird eine alternative Firmware
gebraucht, die den Access Point im Ad-Hoc Modus betreiben kann. Wir
haben allerdings keine Access Points gefunden, f"ur die es alternative
Firmware mit Ad-Hoc Modus Unterst"utzung gibt.

Hier sind einige Access Points aufgelistet, die den Standard IEEE 802.11a
unterst"utzen:

\begin{itemize}
\item Intel PRO/Wireless 5000 

\url{http://support.intel.com/support/wireless/wlan/pro5000/accesspoint}

\url{http://www.pcmag.com/article2/0,1759,5524,00.asp}

\item Linksys WAP55AG 

\url{http://www.tomsnetworking.de/content/aktuelles/news_beitrag/news/851/6/index.html}

\item NETGEAR WAB102 

\url{http://kbserver.netgear.com/products/WAB102.asp}

\url{http://reviews.cnet.com/wireless-access-points/netgear-wab102-802-11a/4505-3265\_7-20708150.html}

\url{http://archive.cert.uni-stuttgart.de/bugtraq/2003/12/msg00159.html}

\end{itemize}
