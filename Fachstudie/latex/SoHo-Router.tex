\subsubsection{SoHO-Router}

Man kann herk"ommliche WLAN-Router f"ur Heimanwender (SoHO-Router - Small
or Home Office) kaufen, die sich mit alternativer Firmware (spezielle
Linux-Software) zu einem Mesh-Router umr"usten lassen. Ein WLAN-Router
ist die Kombination eines normalen Router (Kabelrouter) mit einem
Accesspoint. Es gibt solche mit eingebautem Modem und andere mit einem
Anschluss (WAN-Port) daf"ur (f"ur Modems mit LAN-Anschluss).
Die meisten SoHO WLAN-Router unterst"utzen den Ad-Hoc Modus im 5 GHz
Frequenzband gar nicht, sie arbeiten nur im Inrastruktur Modus. Deswegen
muss bei diesen Routern zuerst eine neue WLAN-karte eingebaut werden,
den Ad-Hoc Modus im 5 GHz Frequenzband unterst"utzt, und dann muss noch
alternative Firmware auf Routern installiert werden, die diese Router
im Ad-Hoc Modus betreiben kann.

Die hier betrachteten SoHO WLAN-Router haben deswegen mindestens einen
MiniPCI-Slot vorhanden. In diesen Slot kann z.B. eine Atheros MiniPCI
WLAN-Karte eingebaut werden.

Die meisten Hersteller legen den Quellcode des Betriebssystems, das in
ihren SoHO WLAN-Routern eingesetzt wird, nicht frei. Aber es gibt
mehrere freie Firmwares f"ur SoHO WLAN-Router, siehe
\ref{sec:Open-Source Firmware f"ur SoHO WLAN-Router}.

Im folgenden sind Vorteile und Nachteile von SoHO WLAN-Routern erl"autert.

\textbf{Vorteile:}

\begin{itemize}
	\item Klein und	und handlich
	\item Mobil und flexibel
	\item Sehr g"unstig
	\item Gute Reichweite
	\item Wenig Stromverbrauch
	\item Leichte Konfiguration und Installation
\end{itemize}

\textbf{Nachteile:}

\begin{itemize}
	\item Eingeschr"ankte Software-Unterst"utzung
	\item Open-Source Firmware schwer zu finden
	\item Durch das "Offnen von Ger"aten und das Einspielen von
	fremder Firmware erlischt die Garantie des Herstellers
	\item Eingeschr"ankter Funktionsumfang
\end{itemize}

%%%%%%%%%%%%%%%%%%%%%%%%%%%%%%%%%%%%%%%%%%%%%%%%%%%%%%%%%%%%%%%%%%%%%%%%%%%%
%
% Linksys WRT54G v1.0
%
%%%%%%%%%%%%%%%%%%%%%%%%%%%%%%%%%%%%%%%%%%%%%%%%%%%%%%%%%%%%%%%%%%%%%%%%%%%%
\begin{wlandevice}{Linksys WRT54G v1.0}

\wlanimage{Linksys_WRT54G}{Linksys WRT54G v1.0}

\begin{wlanieeestandard}
\item 802.11b/g
\item 802.11a/b/g (wenn man die mitgelieferte MiniPCI WLAN-Karte
durch z.B. Atheros 802.11a/b/g WLAN-Karte austauscht)
\end{wlanieeestandard}

\begin{wlanmode}
\item Ad-Hoc
\item Infrastruktur
\end{wlanmode}

\begin{wlanfirmware}
\item
Es gibt mehrere fremde frei verf"ugbare Firmware f"ur dieses Ger"at.
Alle unten aufgef"uhrten Firmware sind Open-Source Projekte:

OpenWRT

\url{http://wiki.openwrt.org/OpenWrtDocs/Hardware/Linksys/WRT54G}

DD-WRT

\url{http://www.dd-wrt.com/wiki/index.php/Linksys_WRT54G/GL/GS/GX}
\end{wlanfirmware}

\wlanprice{40-50}

\begin{wlaninstall}
\item
Die mitgelieferte MiniPCI WLAN-Karte durch z.B. Atheros 802.11a MiniPCI
austauschen und oben erw"ahnte frei verf"ugbare Firmware installieren
(siehe oben Firmware).
\end{wlaninstall}

\begin{wlanextrainfo}
\item
Ein MiniPCI Slot ist f"ur eine WLAN-Karte vorhanden.
\end{wlanextrainfo}

\begin{wlanlink}
\item \url{http://wiki.openwrt.org/OpenWrtDocs/Hardware/Linksys/WRT54G}
\item \url{http://www.dd-wrt.com/wiki/index.php/Linksys_WRT54G/GL/GS/GX}
\item \url{http://www.linksysinfo.org/forums/showthread.php?t=47124}
\end{wlanlink}

\end{wlandevice}

%%%%%%%%%%%%%%%%%%%%%%%%%%%%%%%%%%%%%%%%%%%%%%%%%%%%%%%%%%%%%%%%%%%%%%%%%%%%
%
% Linksys WRT55AG
%
%%%%%%%%%%%%%%%%%%%%%%%%%%%%%%%%%%%%%%%%%%%%%%%%%%%%%%%%%%%%%%%%%%%%%%%%%%%%
\begin{wlandevice}{Linksys WRT55AG}

\wlanimage{Linksys_WRT55AG}{Linksys WRT55AG}

\begin{wlanieeestandard}
\item 802.11a/b/g
\end{wlanieeestandard}

\begin{wlanmode}
\item Ad-Hoc
\item Infrastruktur
\end{wlanmode}

\begin{wlanfirmware}
\item
Open-Source Firmware befindet sich noch in Entwicklung:

Modifizierte Version von OpenWRT Kamikaze

\url{http://legacy.not404.com/cgi-bin/trac.fcgi/wiki/OpenWRT/Atheros/Linksys/WRT55AGv2#KamikazeKernelonWRT55AGv2}

OpenWRT

\url{http://wiki.openwrt.org/OpenWrtDocs/Hardware/Linksys/WRT55AG}
\end{wlanfirmware}

\wlanprice{70-80}

\begin{wlanextrainfo}
\item
2 Slots sind f"ur MiniPCI WLAN-Karten vorhanden.
\end{wlanextrainfo}

\begin{wlanlink}
\item \url{http://wiki.openwrt.org/OpenWrtDocs/Hardware/Linksys/WRT55AG}
\item \url{http://www.tomsnetworking.de/content/tests/j2003a/test_linksys_wrt55ag/index.html}
\item \url{http://reviews.cnet.com/routers/linksys-wrt55ag-wireless-a/4505-3319_7-21131921.html}
\item \url{http://legacy.not404.com/cgi-bin/trac.fcgi/wiki/OpenWRT/Atheros/Linksys/WRT55AGv2}
\end{wlanlink}

\end{wlandevice}

%%%%%%%%%%%%%%%%%%%%%%%%%%%%%%%%%%%%%%%%%%%%%%%%%%%%%%%%%%%%%%%%%%%%%%%%%%%%
%
% Asus WL500G/GP
%
%%%%%%%%%%%%%%%%%%%%%%%%%%%%%%%%%%%%%%%%%%%%%%%%%%%%%%%%%%%%%%%%%%%%%%%%%%%%
\begin{wlandevice}{Asus WL500G/GP}

\wlanimage{Asus_WL500G}{Asus WL500G/GP}

\begin{wlanieeestandard}
\item 802.11b/g
\item 802.11a/b/g (wenn man die mitgelieferte MiniPCI WLAN-Karte
durch z.B. Atheros 802.11a/b/g WLAN-Karte austauscht)
\end{wlanieeestandard}

\begin{wlanmode}
\item Ad-Hoc
\item Infrastruktur
\end{wlanmode}

\begin{wlanfirmware}
\item
Es sind mehrere fremde frei verf"ugbare Firmware f"ur dieses Ger"at.
Alle unten aufgef"uhrten Firmware sind Open-Source Projekte:

OpenWRT

\url{http://wiki.openwrt.org/OpenWrtDocs/Hardware/Asus/WL500G}

\url{http://wiki.openwrt.org/OpenWrtDocs/Hardware/Asus/WL500GP}

FreeWRT

\url{http://www.freewrt.org/trac/wiki/Documentation/Hardware/AsusWL500G}

\url{http://www.freewrt.org/trac/wiki/Documentation/Hardware/AsusWL500GP}

Olegs custom firmware

\url{http://oleg.wl500g.info}
\end{wlanfirmware}

\wlanprice{70-80}

\begin{wlaninstall}
\item
Die mitgelieferte MiniPCI WLAN-Karte durch z.B. Atheros 802.11a MiniPCI
austauschen und oben erw"ahnte frei verf"ugbare Firmware installieren
(siehe oben Firmware).

\url{http://wiki.opennet-initiative.de/index.php/Mini-PCI_Umbau}
\end{wlaninstall}

\begin{wlanextrainfo}
\item
Ein MiniPCI Slot ist f"ur eine WLAN-Karte vorhanden.
\end{wlanextrainfo}

\begin{wlanlink}
\item \url{http://wiki.opennet-initiative.de/index.php/AP9}
\item \url{http://wiki.openwrt.org/OpenWrtDocs/Hardware/Asus/WL500G}
\item \url{http://wiki.openwrt.org/OpenWrtDocs/Hardware/Asus/WL500GP}
\item \url{http://www.freewrt.org/trac/wiki/Documentation/Hardware/AsusWL500G}
\item \url{http://www.freewrt.org/trac/wiki/Documentation/Hardware/AsusWL500GP}
\item \url{http://wl500g.dyndns.org/}
\item \url{http://oleg.wl500g.info/}
\item \url{http://au.asus.com/products.aspx?l1=12&l2=43}
\item \url{http://www.freifunk-bno.de/component/option,com_smf/Itemid,88/topic,910.msg10357/}
\item \url{http://www.cyber-wulf.de/a_wl500g.html}
\item \url{http://wiki.openwrt.org/OpenWrtDocs/Hardware/Asus/WL500G}
\item \url{http://forum.opennet-initiative.de/print.php?threadid=505&page=6&sid=460903353d70c65fad4960105ab76cdd}
\item \url{http://forum.openwrt.org/viewtopic.php?pid=41756}
\item \url{http://www.familie-prokop.de/asus-wl500gp/index.html}
\end{wlanlink}

\end{wlandevice}

\paragraph{Andere WLAN-Router}

\begin{itemize}

\item Netgear HR314

802.11a WLAN-Router, unterst"utzt Ad-Hoc- und Infrastruktur-Modus,
keine Open-Source Firmware vorhanden, kostet ca. 30 Euro

\url{http://www.wi-fiplanet.com/reviews/article.php/1559091}

\end{itemize}
