\subsection{PCs + WLAN-Karten}

Die einfachste M"oglichkeit w"are die herk"ommlichen PCs mit
WLAN-Karten zu einem Mesh-Router einzurichten.  Man nimmt dabei
einfach die WLAN-Karten (PCI, MiniPCI(e) oder PCMCIA) und baut diese
in PCs oder in Laptops ein. Man installiert dann auf diesen Rechnern
entsprechende Treiber, die WLAN-Karten im Ad-Hoc Modus betreiben k"onnen
und Routing-Software, z.B. OLSR-Protokoll.

Es besteht aber das Problem, dass heutige WLAN-Karten den Ad-Hoc Modus im 5 GHz
Frequenzband gar nicht oder sehr schlecht unterst"utzen. Das liegt daran,
dass der Ad-Hoc Modus viel komplizierter als Infrastruktur Modus ist und
in den meisten F"allen werden WLAN-Karten sowieso nur im Infrastruktur Modus
eingesetzt.

Es gibt aber WLAN-Karten, bei denen der Ad-Hoc Modus im 5 GHz Frequenzband
sehr gut unterst"utzt wird. Das sind WLAN-Karten von Intel und WLAN-Karten,
die auf Atheros Chips"atzen basieren. Diese WLAN-Karten werden 
in folgenden Abschnitten detaillierter beschrieben und analysiert.

Die Hardware f"ur Mesh-Router auf PC-Basis kann man in 3 Gruppen unterteilen:

\begin{itemize}
\item PCI WLAN-Karten
\item MiniPCI(e) WLAN-Karten
\item PCMCIA WLAN-Karten
\end{itemize}

F"ur den Einsatz in Mesh-Routern kommen eigentlich nur PCI und MiniPCI(e)
WLAN-Karten in Frage. PCMCIA WLAN-Karten sind eher f"ur Mesh-Clients geeignet,
obwohl sie auch in Mesh-Routern ohne Probleme eingesetzt werden k"onnen.
Die meisten PCs haben n"ahmlich keinen PCMCIA Bus.

Im folgenden werden Vorteile und Nachteile von Mesh-Routern auf PC-Basis
erw"ahnt.

\textbf{Vorteile:}

\begin{itemize} 
\item PCs sind nicht teuer, flexibel und leicht erweiterbar
\item Hardware kann sp"ater f"ur andere Zwecke eingesetzt werden
\item Installation und Konfiguration von Hardware und Software ist einfacher
im Vergleich zu SoHO-Routern
\item Sehr gro"se Menge an verschiedener Software vorhanden
\item Mehrere WLAN- und Ethernet-Schnittstellen m"oglich 
\end{itemize}

\textbf{Nachteile: }

\begin{itemize}
\item PCs sind gro"s und station"ar
\item Brauchen mehr Strom im Vergleich zu SoHO-Routern
\item Ohne externe Antennen schlechte Sende- und Empfangqualit"at,
da sich die Antennen im elektromagnetischen St"orfeld des PCs befindet
\end{itemize}

\subsubsection{PCI-WLAN-Karten}

\begin{wlandevice}{Linksys WMP55AG}
\wlanimage{Linksys_WMP55AG}{Linksys WMP55AG}
\wlanchipset
Atheros AR5213A
\wlanieeestandard
802.11a/b/g
\wlanmode
Ad-Hoc-Modus, Infrastruktur-Modus
\wlansecurity
WPA
LEAP
WEP (40-, 104-, 128-bit)
\wlandriver
Sehr gute Linux-Unterstutzung, madwifi-Treiber funktioniert
mit dieser WLAN PCI-Karte ohne Probleme.
Windows-Treiber werden von Linksys bereitgestellt.
\wlanprice
ca. 90 Euro
\wlaninstall
Lasst sich leicht sowohl unter Windows als auch unter Linux (madwifi-Treiber) installieren.
http://madwifi.org/wiki/UserDocs/FirstTimeHowTo
\end{wlandevice}

\begin{wlandevice}{Netgear WAG311}
\wlanimage{Netgear_WAG311}{Netgear WAG311}
\wlanchipset
\wlanieeestandard
\wlanmode
\wlansecurity
\wlandriver
\wlanprice
\wlaninstall
\end{wlandevice}

\begin{wlandevice}{D-Link DWL-A520}
\wlanimage{DLink_DWLA520}{D-Link DWL-A520}
\wlanchipset
\wlanieeestandard
\wlanmode
\wlansecurity
\wlandriver
\wlanprice
\wlaninstall
\end{wlandevice}

\begin{wlandevice}{Gigabyte GN-WPEAG}
\wlanimage{Gigabyte_GNWPEAG}{Gigabyte GN-WPEAG}
\wlanchipset
\wlanieeestandard
\wlanmode
\wlansecurity
\wlandriver
\wlanprice
\wlaninstall
\end{wlandevice}

\subsubsection{MiniPCI(e) WLAN-Karten}

MiniPCI(e) ist eine vor allem f"ur die Nutzung in Notebooks und Laptops
miniaturisierte Version des PCI Steckplatzes, wie er in allen Desktop
PCs vorkommt. Die Abmessungen einer MiniPCI Karte betragen 6,0 x 4,6 x 0,5 cm.
Die Abmessungen einer MiniPCIe Karte betragen 3 cm x 5 cm x 0.4 cm.

MiniPCI(e) WLAN-Karten sind urspr"unglich f"ur Laptops gedacht, k"onnen aber
mit entschprechenden Adaptern (MiniPCI(e)-to-PCI) und externen Antennen auch
in normalen PCs verwendet werden.

Im folgenden werden Vorteile und Nachteile von MiniPCI(e) WLAN-Karten
erl"autert.

\textbf{Vorteile:}

\begin{itemize}
\item K"onnen mit Hilfe eines Adapters zu einer PCI WLAN-Karte umgebaut werden
\item Leicht austauschbar
\item Sehr gute Treiber-Unterst"utzung unter Linux und Windows
\end{itemize}

\textbf{Nachteile:}

\begin{itemize}
\item Brauchen einen PCI-Adapter f"ur den PCI-Bus
\item Haben keine Antenne (extra Kosten)
\end{itemize}

Wir haben nur eine MiniPCI und drei MiniPCIe WLAN-Karten gefunden,
die den Ad-Hoc Modus im 5 GHz Frequenzband unterst"utzen. Diese WLAN-Karten
basieren entweder auf Intel Chips"atzen oder Atheros Chips"atzen.

%%%%%%%%%%%%%%%%%%%%%%%%%%%%%%%%%%%%%%%%%%%%%%%%%%%%%%%%%%%%%%%%%%%%%%%%%%%%
%
% Wistron CM9 Atheros AR5213A
%
%%%%%%%%%%%%%%%%%%%%%%%%%%%%%%%%%%%%%%%%%%%%%%%%%%%%%%%%%%%%%%%%%%%%%%%%%%%%
\begin{wlandevice}{Wistron CM9 Atheros AR5213A}

\wlanimage{Wistron_CM9}{Wistron CM9 Atheros AR5213A}

\wlanchipset{Atheros AR5213A}

\begin{wlanieeestandard}
\item 802.11a/b/g
\end{wlanieeestandard}

\begin{wlanmode}
\item Ad-Hoc
\item Infrastruktur
\end{wlanmode}

\begin{wlansecurity}
\item WEP (40-, 104-, 128-bit)
\item WPA
\item WPA2
\end{wlansecurity}

\begin{wlandriver}
\item
Herrvorragende Unterst"utzung von MadWifi-Treiber \cite{madwifi},
auch Ad-Hoc-Modus.
\end{wlandriver}

\wlanprice{40}

\begin{wlaninstall}
\item
\url{http://madwifi.org/wiki/UserDocs/FirstTimeHowTo}
\end{wlaninstall}

\begin{wlanlink}
\item \url{http://www.alix-board.de/produkte/wistroncm9.html}
\item \url{http://www.pcengines.ch/cm9.htm}
\item \url{http://forum.openwrt.org/viewtopic.php?pid=10213}
\item \url{http://madwifi.org/ticket/1209}
\end{wlanlink}

\end{wlandevice}

%%%%%%%%%%%%%%%%%%%%%%%%%%%%%%%%%%%%%%%%%%%%%%%%%%%%%%%%%%%%%%%%%%%%%%%%%%%%
%
% Intel PRO/Wireless 3945
%
%%%%%%%%%%%%%%%%%%%%%%%%%%%%%%%%%%%%%%%%%%%%%%%%%%%%%%%%%%%%%%%%%%%%%%%%%%%%
\begin{wlandevice}{Intel PRO/Wireless 3945}

\wlanimage{Intel_3945ABG}{Intel PRO/Wireless 3945}

\wlanchipset{Intel}

\begin{wlanieeestandard}
\item 802.11a/b/g
\end{wlanieeestandard}

\begin{wlanmode}
\item Ad-Hoc
\item Infrastruktur
\end{wlanmode}

\begin{wlansecurity}
\item WEP (40-, 104-bit)
\item WPA
\item WPA2
\end{wlansecurity}

\begin{wlandriver}
\item
Es werden von Intel Treiber sowohl f"ur Windows als auch f"ur Linux
bereitgestellt.

\url{http://downloadcenter.intel.com/Product_Filter.aspx?ProductID=2259}

Von Intel wurde ein Projket f"ur die Unterst�tzung von Intel PRO/Wireless
3945 erstellt.

\url{http://ipw3945.sourceforge.net}

Der ipw3945-Treiber funktioniert auch im Ad-Hoc-Modus, aber nicht sehr stabil,
es kommt oft zu Verbindungsabbr"uchen.
\end{wlandriver}

\wlanprice{20-30}

\begin{wlaninstall}
\item
Im Gegensatz zu den "`klassischen"' Intel Wireless-Chips"atzen 2100- und
2200BG-Chips"atzen ist der Treiber f"ur den 3945ABG noch nicht im Kernel
verf"ugbar. Um auch damit kabellos ins Internet zu gehen,
sind ein paar Handgriffe notwendig.

\url{http://ipw3945.sourceforge.net/README.ipw3945}

\url{http://ipw3945.sourceforge.net/INSTALL}
\end{wlaninstall}

\begin{wlanlink}
\item \url{http://www.intel.com/network/connectivity/products/wireless/prowireless_mobile.htm}
\item \url{http://downloadcenter.intel.com/Product_Filter.aspx?ProductID=2259}
\item \url{http://ipw3945.sourceforge.net/}
\item \url{http://ipw3945.sourceforge.net/README.ipw3945}
\item \url{http://ipw3945.sourceforge.net/INSTALL}
\end{wlanlink}

\end{wlandevice}

%%%%%%%%%%%%%%%%%%%%%%%%%%%%%%%%%%%%%%%%%%%%%%%%%%%%%%%%%%%%%%%%%%%%%%%%%%%%
%
% Intel PRO/Wireless 2915
%
%%%%%%%%%%%%%%%%%%%%%%%%%%%%%%%%%%%%%%%%%%%%%%%%%%%%%%%%%%%%%%%%%%%%%%%%%%%%
\begin{wlandevice}{Intel PRO/Wireless 2915}

\wlanimage{Intel_2915ABG}{Intel PRO/Wireless 2915}

\wlanchipset{Intel}

\begin{wlanieeestandard}
\item 802.11a/b/g
\end{wlanieeestandard}

\begin{wlanmode}
\item Ad-Hoc
\item Infrastruktur
\end{wlanmode}

\begin{wlansecurity}
\item WEP (40-, 104-bit)
\item WPA
\item WPA2
\end{wlansecurity}

\begin{wlandriver}
\item
Es werden von Intel Treiber sowohl f"ur Windows als auch f"ur Linux
bereitgestellt.

\url{http://downloadcenter.intel.com/Product_Filter.aspx?ProductID=1847}

Von Intel wurde ein Projket f"ur die Unterst"utzung von Intel PRO/Wireless
2915 erstellt.

\url{http://ipw2200.sourceforge.net}

Der ipw2200-Treiber funktioniert auch im Ad-Hoc-Modus, aber nicht
sehr stabil, es kommt oft zu verbindungsabbr�chen. Der ipw2200-Treiber
ist im Kernel 2.6 enthalten, kann aber auch separat als Modul kompiliert
werden. Der im Kernel enthaltene Treiber unterst"utzt den Monitor-Modus
nicht.
\end{wlandriver}

\wlanprice{30}

\begin{wlaninstall}
\item
\url{http://ipw2200.sourceforge.net/README.ipw2200}

\url{http://ipw2200.sourceforge.net/INSTALL}
\end{wlaninstall}

\begin{wlanlink}
\item \url{http://support.intel.com/support/wireless/wlan/pro2915abg}
\item \url{http://download.intel.com/support/wireless/wlan/pro2915abg/sb/303330002us_channel.pdf}
\item \url{http://ipw2200.sourceforge.net/}
\item \url{http://www.intel.com/cd/personal/computing/emea/deu/234998.htm}
\item \url{http://downloadcenter.intel.com/Product_Filter.aspx?ProductID=1847}
\end{wlanlink}

\end{wlandevice}

%%%%%%%%%%%%%%%%%%%%%%%%%%%%%%%%%%%%%%%%%%%%%%%%%%%%%%%%%%%%%%%%%%%%%%%%%%%%
%
% Intel Wireless WiFi Link 4965AGN
%
%%%%%%%%%%%%%%%%%%%%%%%%%%%%%%%%%%%%%%%%%%%%%%%%%%%%%%%%%%%%%%%%%%%%%%%%%%%%
\begin{wlandevice}{Intel Wireless WiFi Link 4965AGN}

\wlanimage{Intel_4965AGN}{Intel Wireless WiFi Link 4965AGN}

\wlanchipset{Intel}

\begin{wlanieeestandard}
\item 802.11a/b/g/n(draft)
\end{wlanieeestandard}

\begin{wlanmode}
\item Ad-Hoc
\item Infrastruktur
\end{wlanmode}

\begin{wlansecurity}
\item WEP (40-, 104-bit)
\item WPA
\item WPA2
\end{wlansecurity}

\begin{wlandriver}
\item
\url{http://www.intellinuxwireless.org/}
\end{wlandriver}

\wlanprice{30}

\begin{wlaninstall}
\item
\url{http://www.intellinuxwireless.org/}
\end{wlaninstall}

\begin{wlanlink}
\item \url{http://www.intel.com/network/connectivity/products/wireless/wireless_n/overview.htm}
\item \url{http://www.intellinuxwireless.org/}
\item \url{http://www.wifi-info.de/intel-kuendigt-11n-chipsatz-fuer-centrino-notebooks-an/01/2007/}
\item \url{http://downloadcenter.intel.com/filter_results.aspx?strTypes=all&ProductID=2753&OSFullName=Linux*&lang=eng&strOSs=39&submit=Go\%21}
\end{wlanlink}

\end{wlandevice}

\newpage
\subsubsection{PCMCIA WLAN-Karten}

Diese WLAN-Karten sind f"ur Notebooks gedacht. Heutzutage ist es jedoch
"ublich, dass die Notebooks schon ein integriertes WLAN-Modul
(MiniPCI oder MiniPCIe) eingebaut haben. Damit ist die Notwendigkeit
dieser Module nur noch f"ur Notebooks "alterer Generationen notwendig.
Die PCMCIA WLAN-Karten spielen f"ur die Zwecke der Fachstudie keine
besonders gro"se Rolle und werden hier nur wegen eines m"oglichen
Einsatzes in mobilen Clients des Mesh-Netzes betrachtet.

Im folgenden werden die Vorteile und Nachteile der PCMCIA WLAN-Karten
aufgelistet.

\textbf{Vorteile:}

\begin{itemize}
	\item Leichte Installation
	\item Leicht austauschbar
	\item Sehr gute Unterst"utzung durch Treiber sowohl unter Linux als
	auch Windows
	\item Herrvorragend geeignet f"ur mobile Clients des Mesh-Netzes
\end{itemize}

\textbf{Nachteile:}

\begin{itemize}
	\item Sie sind veraltet und deswegen schwer zu finden
	\item Selten ein Anschluss f"ur eine externe Antenne vorhanden
	\item Relativ teuer im Vergleich zu MiniPCI und MiniPCIe Karten
\end{itemize}

%%%%%%%%%%%%%%%%%%%%%%%%%%%%%%%%%%%%%%%%%%%%%%%%%%%%%%%%%%%%%%%%%%%%%%%%%%%%
%
% Netgear WAG311
%
%%%%%%%%%%%%%%%%%%%%%%%%%%%%%%%%%%%%%%%%%%%%%%%%%%%%%%%%%%%%%%%%%%%%%%%%%%%%
\begin{wlandevice}{Proxim Orinoco Gold 8480-WD}

\wlanimage{Proxim_Orinoco_Gold_8480WD}{Proxim Orinoco Gold 8480-WD}

\wlanchipset{Atheros AR5212}

\begin{wlanieeestandard}
\item 802.11a/b/g
\end{wlanieeestandard}

\begin{wlanmode}
\item Ad-Hoc
\item Infrastruktur
\end{wlanmode}

\begin{wlansecurity}
\item WEP (40-, 104-bit)
\item WPA
\item WPA2
\end{wlansecurity}

\begin{wlandriver}
\item
Unter Linux herrvorragende Unterst"utzung von MadWifi-Treiber \cite{madwifi},
auch Ad-Hoc-Modus.
\end{wlandriver}

\wlanprice{80}

\begin{wlaninstall}
\item
\url{http://madwifi.org/wiki/UserDocs/FirstTimeHowTo}
\end{wlaninstall}

\begin{wlanlink}
\item \url{http://www.proxim.com/products/wifi/client/abgcard/index.html}
\item \url{http://madwifi.org/wiki/Compatibility/Proxim}
\end{wlanlink}

\end{wlandevice}

%%%%%%%%%%%%%%%%%%%%%%%%%%%%%%%%%%%%%%%%%%%%%%%%%%%%%%%%%%%%%%%%%%%%%%%%%%%%
%
% Netgear WAG511
%
%%%%%%%%%%%%%%%%%%%%%%%%%%%%%%%%%%%%%%%%%%%%%%%%%%%%%%%%%%%%%%%%%%%%%%%%%%%%
\begin{wlandevice}{Netgear WAG511}

\wlanimage{Netgear_WAG511}{Netgear WAG511}

\wlanchipset{Atheros AR5001X+}

\begin{wlanieeestandard}
\item 802.11a/b/g
\end{wlanieeestandard}

\begin{wlanmode}
\item Ad-Hoc
\item Infrastruktur
\end{wlanmode}

\begin{wlansecurity}
\item WEP (40-, 104-, 128-bit)
\item WPA
\item WPA2
\item PTP, P2TP, IPSec, VPN pass-through
\end{wlansecurity}

\begin{wlandriver}
\item
Von Netgear werden nur Windows Treiber angeboten.

\url{http://www.netgear.de/de/Support/download.html?func=Detail&id=10676}

Unter Linux herrvorragende Unterst"utzung von MadWifi-Treiber,
auch Ad-Hoc-Modus.

\url{http://madwifi.org/}
\end{wlandriver}

\wlanprice{50-60}

\begin{wlaninstall}
\item
\url{http://www.lrz-muenchen.de/services/netz/mobil/funklan-installation/installation/windowsxp_wg511/flan-instwxp.html}

\url{http://madwifi.org/wiki/UserDocs/FirstTimeHowTo}
\end{wlaninstall}

\begin{wlanlink}
\item \url{http://www.netgear.com/Products/Adapters/AGDualBandWirelessAdapters/WAG511.aspx}
\item \url{http://www.netgear.de/de/Support/download.html?func=Detail&id=10676}
\item \url{http://www.netgear.de/Produkte/Wireless/DualBand/WAG511/index.html}
\item \url{http://madwifi.org/wiki/Compatibility/Netgear}
\end{wlanlink}

\end{wlandevice}

%%%%%%%%%%%%%%%%%%%%%%%%%%%%%%%%%%%%%%%%%%%%%%%%%%%%%%%%%%%%%%%%%%%%%%%%%%%%
%
% SMC 2536W-AG
%
%%%%%%%%%%%%%%%%%%%%%%%%%%%%%%%%%%%%%%%%%%%%%%%%%%%%%%%%%%%%%%%%%%%%%%%%%%%%
\begin{wlandevice}{SMC 2536W-AG}

\wlanimage{SMC_2536WAG}{SMC 2536W-AG}

\wlanchipset{Atheros AR5001}

\begin{wlanieeestandard}
\item 802.11a/b/g
\end{wlanieeestandard}

\begin{wlanmode}
\item Ad-Hoc
\item Infrastruktur
\end{wlanmode}

\begin{wlansecurity}
\item WEP (40-, 104-, 128-bit)
\item WPA
\item WPA2
\end{wlansecurity}

\begin{wlandriver}
\item
Von SMC werden nur Treiber f"ur Windows angeboten.

\url{http://www.smc.com/index.cfm?event=downloads.searchResultsDetail&localeCode=EN_USA&productCategory=9&partNumber=2916&modelNumber=348&knowsPartNumber=false&userPartNumber=&docId=3103}

Unter Linux herrvorragende Unterst"utzung von MadWifi-Treiber  \cite{madwifi},
auch Ad-Hoc-Modus.
\end{wlandriver}

\wlanprice{80}

\begin{wlaninstall}
\item
\url{http://madwifi.org/wiki/UserDocs/FirstTimeHowTo}
\end{wlaninstall}

\begin{wlanlink}
\item \url{http://www.smc.com/index.cfm?event=viewProduct&cid=9&scid=49&localeCode=EN\%5FUSA&pid=348}
\item \url{http://www.smc.com/files/AC/2536Wag_Ds_ww.pdf}
\item \url{http://madwifi.org/wiki/Compatibility/SMC}
\item \url{http://forums.fedoraforum.org/archive/index.php/t-101517.html}
\end{wlanlink}

\end{wlandevice}

%%%%%%%%%%%%%%%%%%%%%%%%%%%%%%%%%%%%%%%%%%%%%%%%%%%%%%%%%%%%%%%%%%%%%%%%%%%%
%
% Linksys WPC55AG
%
%%%%%%%%%%%%%%%%%%%%%%%%%%%%%%%%%%%%%%%%%%%%%%%%%%%%%%%%%%%%%%%%%%%%%%%%%%%%
\begin{wlandevice}{Linksys WPC55AG}

\wlanimage{Linksys_WPC55AG}{Linksys WPC55AG}

\wlanchipset{Atheros AR5212 oder AR5006X}

\begin{wlanieeestandard}
\item 802.11a/b/g
\end{wlanieeestandard}

\begin{wlanmode}
\item Ad-Hoc
\item Infrastruktur
\end{wlanmode}

\begin{wlansecurity}
\item WEP (40-, 104-, 128-bit)
\item WPA
\item WPA2
\end{wlansecurity}

\begin{wlandriver}
\item
Von Linksys werden nur Treiber f"ur Windows bereitgestellt.

\url{http://www.linksys.com/servlet/Satellite?c=L_Product_C2&childpagename=US\%2FLayout&cid=1115416827328&pagename=Linksys\%2FCommon\%2FVisitorWrapper}

Auch hier kann man Treiber f"ur Windows finden:

\url{http://www.phoenixnetworks.net/atheros.php}

Unter Linux herrvorragende Unterst"utzung von MadWifi-Treiber  \cite{madwifi},
auch Ad-Hoc-Modus.
\end{wlandriver}

\wlanprice{50-60}

\begin{wlaninstall}
\item
\url{http://madwifi.org/wiki/UserDocs/FirstTimeHowTo}
\end{wlaninstall}

\begin{wlanlink}
\item \url{http://www.linksys.com/servlet/Satellite?c=L_Product_C2&childpagename=US\%2FLayout&cid=1115416827328&pagename=Linksys\%2FCommon\%2FVisitorWrapper}
\item \url{http://www.phoenixnetworks.net/atheros.php}
\item \url{http://madwifi.org/}
\item \url{http://madwifi.org/wiki/Compatibility/Linksys}
\item \url{http://reviews.cnet.com/adapters-nics/linksys-wpc55ag-dual-band/4505-3380_7-21128291.html}
\item \url{http://www.google.de/search?q=Linksys+WPC55AG}
\item \url{http://lists.funkfeuer.at/pipermail/discuss/2006-September/001592.html}
\item \url{http://www.uk-surplus.com/manuals/brochures/linksyswpc55duser.pdf}
\end{wlanlink}

\end{wlandevice}

\paragraph{Andere PCMCIA-WLAN-Karten}\mbox{}

Hier werden noch einige andere PCMCIA WLAN-Karten aufgelistet, die
analysiert wurden. Da PCMCIA WLAN-Karten sowieso f"ur den Einsatz in
Mesh-Routern nicht geeignet sind, werden sie hier nur kurz beschrieben.
Es gibt noch viel mehr PCMCIA WLAN-Karten, die im Ad-Hoc Modus im 5 GHz
Frequenzband arbeiten, als hier aufgelistet sind. Sie haben sehr
"ahnliche Eigenschaften und die meisten verwenden Atheros Chips"atze
und somit funktionieren unter Linux mit MadWifi-Treiber. Windows-Treiber
werden von allen Herstellern bereitgestellt.

\begin{itemize}

\item Intel PRO/Wireless 5000

Chipsatz Intel,
802.11a WLAN PCMCIA-Karte, unterst"utzt Ad-Hoc- und Infrastruktur-Modus,
Treiber von Intel nur f"ur Windows vorhanden, f"ur Linux werden keine
Treiber entwickelt, kostet ca. 150 Euro

\url{http://www.intel.com/support/wireless/wlan/pro5000/lancardbus}

\item Netgear WAB501
Chipsatz Atheros AR5211,
802.11a/b WLAN-Karte,
MadWifi-Treiber Unterst"utzung

\url{http://kbserver.netgear.com/products/WAB501.asp}\\
\url{http://madwifi.org/wiki/Compatibility/Netgear}

\item Netgear WG511U

Chipsatz Atheros AR5004X,
802.11a/g WLAN-Karte,
MadWifi-Treiber Unterst"utzung

\url{http://www.netgear.com/Products/Adapters/AGDualBandWirelessAdapters/WG511U.aspx}\\
\url{http://madwifi.org/wiki/Compatibility/Netgear}

\item Proxim Orinoco Silver 8481-WD

Chipsatz Atheros AR5001X+,
802.11a/b/g WLAN-Karte,
MadWifi-Treiber Unterst"utzung,
kostet ca 80-90 Euro

\url{http://www.proxim.com/products/wifi/client/abgcard/index.html}\\
\url{http://madwifi.org/wiki/Compatibility/Proxim}

\item Cisco Aironet CB21AG

Chipsatz Atheros 5212,
802.11a/b/g WLAN-Karte,
Madwifi-Treiber Unterst"utzung,
kostet ca 100 Euro

\url{http://madwifi.org/wiki/Compatibility/Cisco}

\end{itemize}

