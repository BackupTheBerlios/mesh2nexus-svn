\section{Aufgabenstellung}
\label{sec:Aufgabenstellung}

F"ur Forschungszwecke soll innerhalb des SFB Nexus \cite{nexus}
ein WMN installiert werden. 

Dieses WMN dient 
\begin{itemize}
	\item einerseits Nexus-Anwendungen, insbesondere Anwendungen 
	auf mobilen Ger"aten, als \emph{Kommunikationsmedium}. 

	\item andererseits auch als \emph{Testbed} 
	zur Erforschung verschiedener Erweiterungen von WMNs.
\end{itemize}

Dieses WMN soll beispielsweise der Untersuchung neuartiger kontextbezogener 
Kommunikationsmechanismen, der Erforschung von 
Publish/Subscribe-Diensten f"ur WMNs oder der 
Verwaltung von Umgebungsmodellen innerhalb eines 
hybriden Systems wie es ein WMN darstellt, dienen.

Ziel dieser Fachstudie ist die Ausarbeitung einer 
Empfehlung f"ur die Beschaffung entsprechender Ger"ate 
(\emph{Hardwareplattformen und Systemsoftware}) f"ur
den Aufbau eines WMN. \\

Das Vorgehen umfasst im einzelnen:

\begin{itemize}
	
	\item Einarbeitung in grundlegende WMN-Technologien
	\item Analyse der Anforderungen des Nexus-Projektes an ein WMN
	\item Erstellung einer "Ubersicht "uber aktuelle verf"ugbare 
	Hardwareplattformen und Systemsoftware f"ur WMN
	\item Bewertung der analysierten Systeme hinsichtlich 
	der ermittelten Anforderungen 	
	\item Ausarbeitung einer Empfehlung f"ur eines geeigneten 
	WMN hinsichtlich Hardwareplattform und Systemsoftware
	
\end{itemize}


\section{Anforderungen}

Nach der Einarbeitung in WMN-Technologien und Analyse der Anforderungen
des Nexus-Projektes an ein WMN wurde folgendes festgehalten:

Anforderungen an Hardware (Mesh-Router):

\begin{itemize}
\item Mesh-Router m"ussen den Standard IEEE 802.11a unterst"utzen, denn
im 2.4 GHz Frequenzband gibt es keine freie Kan"ale mehr, der 2.4 GHz
Frequenzbereich ist "ubers"attigt und die Folge davon sind hohe
Signalst"orungen.
\item Mesh-Router m"ussen im Ad-Hoc Modus im 5 GHz Frequenzband arbeiten
k"onnen.
\item Mesh-Router m"ussen mindestens eine LAN-Schnittstelle haben, damit man
sie nicht nur "uber WLAN erreichen kann. Die LAN-Schnittstelle soll
haupts"achlich zur Konfiguration und zur Verwaltung von Mesh-Routern dienen.
\item Mesh-Router m"ussen mindestens 2 WLAN-Schnittstellen haben, damit sie
auf verschiedenen Kan"alen gleichzeitig arbeiten k"onnen, um den Durchsatz
zu erh"ohen.
\item Es w"are vorteilhaft, wenn Mesh-Router auch den Standard IEEE 802.11n
unterst"utzen.
\end{itemize}

Anforderungen an Software (Betriebssystem, Treiber, Routing-Software und
		Firmware):

\begin{itemize}
\item Betriebssystem f"ur Mesh-Router ist nicht vorgeschrieben (Linux, UNIX oder
Windows). Betriebssystem ist ein Ergebnis der Fachstudie.
\item Routing-Protokoll, der im WMN verwendet wird, ist auch nicht
vorgeschrieben und ist ein Ergebnis der Fachstudie. Routing-Protokoll muss
aber sp"ater austauschbar sein. Routing-Protokoll soll flexibel und
konfigurierbar sein.
\item Routing-Protokoll muss die Erfassung von Topologie unterst"utzen.
\end{itemize}

Zus"atzliche Anforderungen:

\begin{itemize}
\item WMN muss das ganze Geb"aude auf Universit"atsstra"se 38 abdecken.
\item Der Aufwand f"ur den Aufbau des WMN darf nicht 25.000 Euro "uberschreiten.
\item WMN muss mit dem Uni-Netzwerk "uber ein Gateway verbunden werden und
Kommunikation aus dem WMN nach aussen soll nur "uber dieses Gateway
stattfinden.
\end{itemize}