\section{Aufgabenstellung}
\label{sec:Aufgabenstellung}

F"ur Forschungszwecke soll innerhalb des SFB Nexus \cite{nexus}
ein WMN installiert werden. 

Dieses WMN dient 
\begin{itemize}
	\item einerseits Nexus-Anwendungen, insbesondere Anwendungen 
	auf mobilen Ger"aten, als \emph{Kommunikationsmedium}. 

	\item andererseits auch als \emph{Testbed} 
	zur Erforschung verschiedener Erweiterungen von WMNs.
\end{itemize}

Dieses WMN soll beispielsweise der Untersuchung neuartiger kontextbezogener 
Kommunikationsmechanismen, der Erforschung von 
Publish/Subscribe-Diensten f"ur WMNs oder der 
Verwaltung von Umgebungsmodellen innerhalb eines 
hybriden Systems wie es ein WMN darstellt, dienen.

Ziel dieser Fachstudie ist die Ausarbeitung einer 
Empfehlung f"ur die Beschaffung entsprechender Ger"ate 
(\emph{Hardwareplattformen und Systemsoftware}) f"ur
den Aufbau eines WMN. \\

Das Vorgehen umfasst im einzelnen:

\begin{itemize}
	
	\item Einarbeitung in grundlegende WMN-Technologien
	\item Analyse der Anforderungen des Nexus-Projektes an ein WMN
	\item Erstellung einer "Ubersicht "uber aktuelle verf"ugbare 
	Hardwareplattformen und Systemsoftware f"ur WMN
	\item Bewertung der analysierten Systeme hinsichtlich 
	der ermittelten Anforderungen 	
	\item Ausarbeitung einer Empfehlung f"ur eines geeigneten 
	WMN hinsichtlich Hardwareplattform und Systemsoftware
	
\end{itemize}


\section{Anforderungen}

Nach der Einarbeitung in WMN-Technologien und 
Analyse der Anforderungen des Nexus-Projektes
wurde folgendes festgehalten:

TODO: \todo{anforderungen ausformulieren, begruendung, erklaerung, usw..}
TODO: \todo{ warum 5GHz - da 2,4 belegt..}
\subsection{Anforderungen an Hardware}
\begin{itemize}
	\item IEEE 802.11a kompatibel (5GHz Frequenzen) 
	\item Ad-hoc Modus (auch bei 5GHz Frequenzen) 
	\item Treiber f"ur Linux und Windows
	\item Open-Source Firmware f"ur Router
	\item Zus"atzlich zu Wireless Mesh Network auch weitere
	(Netzwerk-)Schnittstellen zur Verwaltung vorsehen (LAN)
	\item Nach M"oglichkeit keine selber gebastelten L"osungen 
	\item Unterst"utzung von IEE 802.11n w"are vorteilhaft
	\item 2 Mini PCI-Slots bei Routern, damit auf verschiedenen
	Frequenzb"andern gesendet und empfangen werden kann
	(Performanceverbesserung) 
\end{itemize}

\subsection{Anforderungen an Software}
\begin{itemize}
	\item Betriebsystem ist nicht festgelegt,
	soll Ergebnis der Fachstudie sein
	\item Freiheit bei Routingprotokollen (Austauschbarkeit)
	\item Software soll Konfigurierung und Instrumentierung von WMN
	erm"oglichen
	\item Topologie ver"andern bzw. erfassen soll m"oglich sein
	\item Visualisierung von Anfragen und Topologie 
\end{itemize}

\subsection{Zus"atzliche Anforderungen}
TODO: \todo{ Aufwandsch"atzung in aufgaben!}
\begin{itemize}
	\item Abdeckung des Geb"audes Universit"atsstra"se 38
	\item Verbindung mit Informatik-Netz nur "uber Gateway
	mit strikter Filterung (nur eine Richtung (UNI->Mesh)
	f"ur die Verwaltung)
	\item Aufwandsch"atzung (wie viele Knoten usw.)
	\item Budget max. 25.000 Euro (evtl. mehr in Zukunft)
	\item 4-5 Mesh-Knoten pro Quadrat 
	\item Speicherkapazit"at wichtig bei Routern
	\item PC + WLAN-Karten sind wegen Flexibilit"at vorzuziehen
	\item PDAs (bzw. andere kleine Clients) mit 802.11a
\end{itemize}


