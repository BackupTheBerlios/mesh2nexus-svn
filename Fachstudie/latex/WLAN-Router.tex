\newpage

\subsection{WLAN-Router}

Die Kombination aus Access Point und Router wird h"aufig als WLAN-Router
bezeichnet. Das ist solange korrekt, soweit es einen WAN-Port gibt. Das
Routing findet dann zwischen WLAN, LAN und WAN statt. Fehlt dieser
WAN-Port, handelt es sich hier lediglich um Marketing-Begriffe, da reine
Access Points auf OSI-Ebene 2 arbeiten und somit Bridges und keine
Router sind. Oft sind das aber keine vollst"andigen Router, da diese
Ger"ate ausschlie"slich als Internetzugangs-Systeme dienen und nur mit
aktiviertem PPPoE (oder PPPoA) sowie NAT-Routing (oder IP-Masquerading)
eingesetzt werden k"onnen.

In diesem Abschnitt werden sogenannte stand-alone WLAN-Router
betrachtet, die als Mesh-Router in einem WMN in Frage kommen.
Alle solche WLAN-Router kann man in 2 Gruppen unterteilen:

\begin{itemize}
\item SoHO-Router
\item Professionelle Router
\end{itemize}

In folgenden Abschnitten werden die beide Gruppen von WLAN-Routern genauer
betrachtet.

\subsubsection{SoHo-Router}

Man kann herkommliche WLAN-Router fur Heimanwender (SoHO-Router -small
or home office)zu kaufen, die sich mit alternativer Firmware (spezielle
Linux software mit OLSR Implementierung) zu einem Mesh-Router umrusten
lassen. Ein WLAN-Router ist die Kombination von eines normalen Routers
(Kabelrouter) und mit einem Accesspoint. Es gibt solche mit eingebauten
Modem und andere mit einem Anschluss (WAN-Port) daf�r (f�r Modems mit
LAN-Anschluss). Ein Nachteil ist, dass es viele Modelle gibt, die eine
fix verbaute Antenne haben, die nicht gewechselt werden kann.

Kosten in der Regel etwa 40-80 euro, haben gute Reichweite, sind klein
und handlich.

Vorteile:
\begin{itemize}
\item klein
\item mobil
\item g�nstig
\item gute Reichweite
\item wenig Strom 
\end{itemize}

Nachteile:
\begin{itemize}
\item meistens fix verbaute Antenne 
\end{itemize}

Durch das �ffnen von Ger�ten und das Einspielen von fremder Firmware
erlischt die Garantie des Herstellers !!!

\paragraph{OpenWRT}

OpenWRT ist eine GNU/Linux-Distribution f�r WLAN-Router. Anstatt einer
statischen Firmware setzt OpenWRT auf ein voll beschreibbares Dateisystem
sowie einen Paketmanager. OpenWRT l�uft unter anderem auf Ger�ten der
Firmen Linksys, ALLNET, ASUS, Belkin, Buffalo, Microsoft und Siemens.

Vorteile:
\begin{itemize}
\item Flexibilit�t
\item Erweiterbarkeit
\item Individualisierbarkeit
\item Sicherheit
\item Gewohnte Linux-Flexibilit�t und Funktionsumfang!!! 
\end{itemize}

Nachteile:
\begin{itemize}
\item Standardm��ig sind nur die n�tigsten Unix-Tools vorhanden 
\end{itemize}

Links:
\begin{itemize}
\item \url{http://openwrt.org/}
\item \url{http://toh.openwrt.org/}
\end{itemize}

%%%%%%%%%%%%%%%%%%%%%%%%%%%%%%%%%%%%%%%%%%%%%%%%%%%%%%%%%%%%%%%%%%%%%%%%%%%%
%
% Linksys WRT54G v1.0
%
%%%%%%%%%%%%%%%%%%%%%%%%%%%%%%%%%%%%%%%%%%%%%%%%%%%%%%%%%%%%%%%%%%%%%%%%%%%%
\begin{wlandevice}{Linksys WRT54G v1.0}

\wlanimage{Linksys_WRT54G}{Linksys WRT54G v1.0}

\begin{wlanieeestandard}
\item 802.11b/g
\item 802.11a/b/g (wenn man die mitgelieferte Mini-PCI WLAN-Karte
durch z.B. Atheros 802.11a/b/g WLAN-Karte austauscht)
\end{wlanieeestandard}

\begin{wlanmode}
\item Ad-Hoc
\item Infrastruktur
\end{wlanmode}

\begin{wlanfirmware}
\item
Es sind mehrere fremde frei verf�gbare Firmware f�r dieses Ger�t.
Alle unten aufgef�hrten Firmware sind Open-Source Projekte:
OpenWRT \url{http://wiki.openwrt.org/OpenWrtDocs/Hardware/Linksys/WRT54G}
DD-WRT \url{http://www.dd-wrt.com/wiki/index.php/Linksys_WRT54G/GL/GS/GX}
\end{wlanfirmware}

\wlanprice{40-50}

\begin{wlaninstall}
\item
Die mitgelieferte Mini-PCI WLAN-Karte durch z.B. Atheros 802.11a Mini-PCI
austauschen und oben erw�hnte frei verf�gbare Firmware installieren
(siehe oben Firmware).
\end{wlaninstall}

\begin{wlanextrainfo}
\item
Ein Mini-PCI Slot ist f�r eine WLAN-Karte vorhanden.
\end{wlanextrainfo}

\begin{wlanlink}
\item \url{http://wiki.openwrt.org/OpenWrtDocs/Hardware/Linksys/WRT54G}
\item \url{http://www.dd-wrt.com/wiki/index.php/Linksys_WRT54G/GL/GS/GX}
\item \url{http://forum.opennet-initiative.de/thread.php?threadid=505&sid=56c53647db6353a41e9a3100f00d02c4}
\item \url{http://www.linksysinfo.org/forums/showthread.php?t=47124}
\end{wlanlink}

\end{wlandevice}

%%%%%%%%%%%%%%%%%%%%%%%%%%%%%%%%%%%%%%%%%%%%%%%%%%%%%%%%%%%%%%%%%%%%%%%%%%%%
%
% Linksys WRT55AG
%
%%%%%%%%%%%%%%%%%%%%%%%%%%%%%%%%%%%%%%%%%%%%%%%%%%%%%%%%%%%%%%%%%%%%%%%%%%%%
\begin{wlandevice}{Linksys WRT55AG}

\wlanimage{Linksys_WRT55AG}{Linksys WRT55AG}

\begin{wlanieeestandard}
\item 802.11a/b/g
\end{wlanieeestandard}

\begin{wlanmode}
\item Ad-Hoc
\item Infrastruktur
\end{wlanmode}

\begin{wlanfirmware}
\item
Open-Source Firmware befindet sich noch in Entwicklung.
Modifizierte Version von OpenWRT Kamikaze
\url{http://legacy.not404.com/cgi-bin/trac.fcgi/wiki/OpenWRT/Atheros/Linksys/WRT55AGv2#KamikazeKernelonWRT55AGv2}
OpenWRT
\url{http://wiki.openwrt.org/OpenWrtDocs/Hardware/Linksys/WRT55AG}
\end{wlanfirmware}

\wlanprice{70-80}

\begin{wlanextrainfo}
\item
2xMini-PCI Slots sind f�r WLAN-Karten vorhanden.
\end{wlanextrainfo}

\begin{wlanlink}
\item \url{http://wiki.openwrt.org/OpenWrtDocs/Hardware/Linksys/WRT55AG}
\item \url{http://www.tomsnetworking.de/content/tests/j2003a/test_linksys_wrt55ag/index.html}
\item \url{http://reviews.cnet.com/routers/linksys-wrt55ag-wireless-a/4505-3319_7-21131921.html}
\item \url{http://legacy.not404.com/cgi-bin/trac.fcgi/wiki/OpenWRT/Atheros/Linksys/WRT55AGv2}
\end{wlanlink}

\end{wlandevice}

%%%%%%%%%%%%%%%%%%%%%%%%%%%%%%%%%%%%%%%%%%%%%%%%%%%%%%%%%%%%%%%%%%%%%%%%%%%%
%
% Asus WL500G/GP
%
%%%%%%%%%%%%%%%%%%%%%%%%%%%%%%%%%%%%%%%%%%%%%%%%%%%%%%%%%%%%%%%%%%%%%%%%%%%%
\begin{wlandevice}{Asus WL500G/GP}

\wlanimage{Asus_WL500G}{Asus WL500G/GP}

\begin{wlanieeestandard}
\item 802.11b/g
\item 802.11a/b/g (wenn man die mitgelieferte Mini-PCI WLAN-Karte
durch z.B. Atheros 802.11a/b/g WLAN-Karte austauscht)
\end{wlanieeestandard}

\begin{wlanmode}
\item Ad-Hoc
\item Infrastruktur
\end{wlanmode}

\begin{wlanfirmware}
\item
Es sind mehrere fremde frei verf�gbare Firmware f�r dieses Ger�t.
Alle unten aufgef�hrten Firmware sind Open-Source Projekte:
OpenWRT
\url{http://wiki.openwrt.org/OpenWrtDocs/Hardware/Asus/WL500G}
\url{http://wiki.openwrt.org/OpenWrtDocs/Hardware/Asus/WL500GP}
FreeWRT
\url{http://www.freewrt.org/trac/wiki/Documentation/Hardware/AsusWL500G}
\url{http://www.freewrt.org/trac/wiki/Documentation/Hardware/AsusWL500GP}
Olegs custom firmware
\url{http://oleg.wl500g.info}
\end{wlanfirmware}

\wlanprice{70-80}

\begin{wlaninstall}
\item
Die mitgelieferte Mini-PCI WLAN-Karte durch z.B. Atheros 802.11a Mini-PCI
austauschen und oben erw�hnte frei verf�gbare Firmware installieren
(siehe oben Firmware).
\url{http://wiki.opennet-initiative.de/index.php/Mini-PCI_Umbau}
\end{wlaninstall}

\begin{wlanextrainfo}
\item
Ein Mini-PCI Slot ist f�r eine WLAN-Karte vorhanden.
\end{wlanextrainfo}

\begin{wlanlink}
\item \url{http://wiki.opennet-initiative.de/index.php/AP9}
\item \url{http://wiki.openwrt.org/OpenWrtDocs/Hardware/Asus/WL500G}
\item \url{http://wiki.openwrt.org/OpenWrtDocs/Hardware/Asus/WL500GP}
\item \url{http://www.freewrt.org/trac/wiki/Documentation/Hardware/AsusWL500G}
\item \url{http://www.freewrt.org/trac/wiki/Documentation/Hardware/AsusWL500GP}
\item \url{http://wl500g.dyndns.org/}
\item \url{http://oleg.wl500g.info/}
\item \url{http://au.asus.com/products.aspx?l1=12&l2=43}
\item \url{http://www.freifunk-bno.de/component/option,com_smf/Itemid,88/topic,910.msg10357/}
\item \url{http://www.cyber-wulf.de/a_wl500g.html}
\item \url{http://wiki.openwrt.org/OpenWrtDocs/Hardware/Asus/WL500G}
\item \url{http://forum.opennet-initiative.de/print.php?threadid=505&page=6&sid=460903353d70c65fad4960105ab76cdd}
\item \url{http://forum.openwrt.org/viewtopic.php?pid=41756}
\item \url{http://www.familie-prokop.de/asus-wl500gp/index.html}
\end{wlanlink}

\end{wlandevice}

\paragraph{Andere WLAN-Router}

\begin{itemize}

\item Netgear HR314\\
802.11a WLAN-Router, unterst�tzt Ad-Hoc- und Infrastruktur-Modus,\\
keine Open-Source Firmware vorhanden, kostet ca. 30 Euro\\
\url{http://www.wi-fiplanet.com/reviews/article.php/1559091}

\end{itemize}

\subsubsection{Professionelle Router}

In diesem Abschnitt werden sogenannte professionelle Mesh-Router betrachtet. 
Die Begriffe, die daf"ur oft als Synonyme verwendet werden, sind dabei: 

\begin{itemize}	
\item Routerboards
\item Stand-alone Mesh-Router 
\item Minicomputers 
\item Single-Board-Computers (SBC) 
\item Access Points
\end{itemize}

Die meisten professionellen WLAN-Router sind sehr teuer und kommen deswegen
nicht in Frage f"ur unsere Zwecke. Viele dieser Router verwenden auch
proprit"are Routing-Protokolle und sind deswegen f"ur Forschungszwecke
ungeeignet. Zus"atzlich, diese professionellen WLAN-Router sind sehr
leistungsstark und werden deswegen eher im Aussenbereich verwendet und
sind f"ur ein Geb"aude deswegen ungeeignet.

Es gibt allerdings auch SBCs, die sehr wohl als Mesh-Router in einem WMN
eingesetzt werden k"onnen. Das Projekt UMIC-Mesh (\url{http://umic-mesh.net})
verwendet solche SBCs f"ur WMNs. Die vom Projekt eingesetzten SBCs
haben 2 MiniPCI Slots, 128 MB Speicher, 233 MHz AMD Geode SC1100 CPU,
100 MB/s LAN-Schnittstelle und einen seriellen Port. Diese SBCs k"onnten
als alternative f"ur Mesh-Router auf PC-Basis eingesetzt werden, allerdings
sind sie nicht so flexibel und erweiterbar wie PCs.

Zusammengefasst, sind professionelle WLAN-Router f"ur uns ungeeignet und werden
weiter auch nicht betrachtet.

\subsubsection{Access Points}

Ein Access Point ist der Verbindungspunkt eines kabelbasierten
Netzwerkes zu einem WLAN. Der Access Point ist eine Basisstation f"ur alle
WLAN-Clients, zu der sie eine drahtlose Verbindung aufbauen.
Sendet ein WLAN-Client Daten, die f"ur einen Empf"anger im kabelbasierten
Netzwerkteil bestimmt sind, so reicht der Accesspoint diese Daten "uber
das Kabelnetz an den Empf"anger weiter. Weiterhin kann ein Access Point
auch mehrere WLAN-Clienten untereinander verbinden. Somit ist der Access Point
quasi ein kabelloser Switch.

Access Points kommen als Mesh-Router nicht in Frage, denn Access Points
werden in der Regel nur im Infrastruktur Modus eingesetzt. Um ein Access
Point auch im Ad-Hoc Modus zu betreiben, wird eine alternative Firmware
gebraucht, die den Access Point im Ad-Hoc Modus betreiben kann. Wir
haben allerdings keine Access Points gefunden, f"ur die es alternative
Firmware mit Ad-Hoc Modus Unterst"utzung gibt.

Hier sind einige Access Points aufgelistet, die den Standard IEEE 802.11a
unterst"utzen:

\begin{itemize}
\item Intel PRO/Wireless 5000 

\url{http://support.intel.com/support/wireless/wlan/pro5000/accesspoint}

\url{http://www.pcmag.com/article2/0,1759,5524,00.asp}

\item Linksys WAP55AG 

\url{http://www.tomsnetworking.de/content/aktuelles/news_beitrag/news/851/6/index.html}

\item NETGEAR WAB102 

\url{http://kbserver.netgear.com/products/WAB102.asp}

\url{http://reviews.cnet.com/wireless-access-points/netgear-wab102-802-11a/4505-3265\_7-20708150.html}

\url{http://archive.cert.uni-stuttgart.de/bugtraq/2003/12/msg00159.html}

\end{itemize}

