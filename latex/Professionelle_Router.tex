\subsubsection{Professionelle Router}

In diesem Abschnitt werden so genannte Stand-alone 
Mesh-Router betrachtet. 
Die Begriffe, die dafur oft als Synonume verwendet werden, sind dabei: 

\begin{itemize}	
	\item Routerboards
	\item Stand-alone Mesh-Router 
	\item Minicomputers 
	\item Single-Board-Computers (SBC) 
	\item Access Points 	
\end{itemize}
 

Im Projekt (\url{http://umic-mesh.net}) wurden professionelle Router eingesetzt, das sind spezielle RouterBoards mit Steckplatzen fur MiniPCI Wlankarten. Boards kosten etwa 100-200 Euro, dazu muss man allerdings noch passende Wlan-karten kaufen + Antennen + Kabel + netzteil+ gehause, also keine billige losung. 

Man konnte aber nur diese Karten kaufen, + Adapter PCI-MiniPCI und in Rechner einbauen. (Das ware dann die Platzsparende Reprasentation von \emph{PCs + WLAN-Karte} Moglichkeit) Karten z.B: Wistron Neweb CM9 Atheros 802.11a/b/g mini-PCI, hier 
\url{http://www.pcengines.ch/cm9.htm}

Boards sind hier: 
\url{http://www.pcengines.ch/wrap.htm, http://www.pcengines.ch/alix.htm}

Es gibt noch diese kleine Mesh-Router, wie von Meraki. Die haben wohl ihre eigene Firmware drin und eigene Routingprotokolle, oder eigene Implementierungen davon besser gesagt. Hier ein Paar, die 802.11a unterstutzen, sind aber outdoor, haben also gro?e Reichweiten. Ob es sinnvoll ist, sie im Gebaude einzusetzen.. 

Aphelion 3300AG Outdoor Wireless Access Point - 802.11a/b/g , Aphelion 600AG / 605AG Intelligenter, sequentieller Wireless Access Point fur den Au?enbereich mit den Standards 802.11a/b/g 
\url{http://www.abcdata.de/abcdataneu/WLAN_MESH_Aphelion.php}

oder PLANET MAP-2100 indoor sind aber zum Teil sehr teuer (1200 Euro !!!) 

\textbf{Vorteile:}

\begin{itemize}	
	\item Outdoor (in unserem Fall unrelevant) 
	\item Gro�e Reichweiten
\end{itemize}
 

\textbf{Nachteile:} 

\begin{itemize}	
	\item Zum Teil sehr teuer (1200 Euro !!!)
\end{itemize}
 

\textbf{Links:}

\begin{itemize}	
	\item \url{http://wiki.opennet-initiative.de/index.php/WRAP}
	\item \url{http://www.abcdata.de/abcdataneu/WLAN_MESH_Aphelion.php}
	\item \url{http://www.aerial.net/shop/product_info.php?cPath=33&products_id=351}
	\item \url{http://forum.openwrt.org/viewtopic.php?id=9655}
\end{itemize}

\subsection{Access Points}

Ein WLAN-Accesspoint ist der Verbindungspunkt eines kabelbasierten Netzwerkes zu einem WLAN. Der Accesspoint ist Basisstation fur alle WLAN-Clienten, zu der sie eine drahtlose Verbindung aufbauen. Sendet ein WLAN-Client Daten, die fur einen Empfanger im kabelbasierten Netzwerkteil bestimmt sind, so \emph{reicht} der Accesspoint diese Daten uber das Kabelnetz an den Empfanger weiter. Weiterhin kann ein Accesspoint auch mehrere WLAN-Clienten untereinander verbinden. Somit ist der Accesspoint quasi ein kabelloser Switch. 

Dieser hat (je nach Austattung) einige der folgenden Optionen: 
\begin{itemize}
	\item Ein oder mehrer integrierte WLAN-Module
	\item Einen integrierten DHCP-Server 
	\item Umfangreiche Sicherheits- und Verschlusselungsmoglichkeiten 
	

		\item WEP, WPA und WPA2 dienen der Verschlusselung der zu ubetragenden Daten 
		\item MAC-Filter und SSID Optionen 
		\item Einstellungen bezuglich dem Remotezugriff 

	
	\item Verschiedenen Arbeitsmodi 

		\item Accesspoint (AP) 
		\item Bridge (Point-to-Point oder Point-to-Multipoint) 
		\item Repeater
		\item MESSID 

\end{itemize}
 
Intel PRO/Wireless 5000 

\begin{itemize}
  \item \url{http://support.intel.com/support/wireless/wlan/pro5000/accesspoint}
  \item \url{http://www.pcmag.com/article2/0,1759,5524,00.asp}
\end{itemize}

Linksys WAP55AG 

\begin{itemize}
  \item \url{http://www.tomsnetworking.de/content/aktuelles/news_beitrag/news/851/6/index.html}
\end{itemize}

NETGEAR WAB102 

\begin{itemize}
  \item \url{http://kbserver.netgear.com/products/WAB102.asp}
  \item \url{http://reviews.cnet.com/wireless-access-points/netgear-wab102-802-11a/4505-3265_7-20708150.html}
 \item \url{http://archive.cert.uni-stuttgart.de/bugtraq/2003/12/msg00159.html}
\end{itemize}