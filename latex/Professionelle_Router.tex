\subsubsection{Professionelle Router}

In diesem Abschnitt werden so genannte Stand-alone Mesh-Router betrachtet. 
Die Begriffe, die daf"ur oft als Synonyme verwendet werden, sind dabei: 

\begin{itemize}	
	\item Routerboards
	\item Stand-alone Mesh-Router 
	\item Minicomputers 
	\item Single-Board-Computers (SBC) 
	\item Access Points 	
\end{itemize}
 

Im Projekt (\url{http://umic-mesh.net}) wurden professionelle Router
eingesetzt, das sind spezielle Router-Boards mit Steckpl"atzen f"ur
MiniPCI WLAN-karten. Boards kosten etwa 100-200 Euro, dazu muss man allerdings
noch passende WLAN-Karten kaufen + Antennen + Kabel + Netzteil + Geh"ause,
also keine billige L"osung. 

Man k"onnte aber nur diese Karten kaufen + Adapter PCI-MiniPCI und in Rechner
einbauen (Das w"are dann die platzsparende Version von \emph{PCs + WLAN-Karte}).
WLAN-Karten z.B Wistron Neweb CM9 Atheros 802.11a/b/g Mini-PCI, hier 
\url{http://www.pcengines.ch/cm9.htm}.

Boards sind hier
\url{http://www.pcengines.ch/wrap.htm, http://www.pcengines.ch/alix.htm}

Es gibt noch diese kleine Mesh-Router, wie von Meraki. Die haben wohl
ihre eigene Firmware drin und eigene Routingprotokolle oder eigene
Implementierungen davon besser gesagt. Hier ein Paar, die 802.11a
unterst"utzen, sind aber outdoor, haben also gro"se Reichweiten.
Ob es sinnvoll ist, sie im Gebaude einzusetzen:
Aphelion 3300AG Outdoor Wireless Access Point - 802.11a/b/g,
Aphelion 600AG/605AG Intelligenter sequentieller Wireless Access Point f"ur
den Au"senbereich mit den Standards 802.11a/b/g 
\url{http://www.abcdata.de/abcdataneu/WLAN_MESH_Aphelion.php}
oder PLANET MAP-2100 indoor sind aber zum Teil sehr teuer (1200 Euro !!!). 

\textbf{Vorteile:}

\begin{itemize}	
	\item Outdoor (in unserem Fall unrelevant) 
	\item Gro�e Reichweiten
\end{itemize}
 

\textbf{Nachteile:} 

\begin{itemize}	
	\item Zum Teil sehr teuer (1200 Euro !!!)
\end{itemize}
 

\textbf{Links:}

\begin{itemize}	
	\item \url{http://wiki.opennet-initiative.de/index.php/WRAP}
	\item \url{http://www.abcdata.de/abcdataneu/WLAN_MESH_Aphelion.php}
	\item \url{http://www.aerial.net/shop/product_info.php?cPath=33&products_id=351}
	\item \url{http://forum.openwrt.org/viewtopic.php?id=9655}
\end{itemize}

\subsection{Access Points}

Ein WLAN-Accesspoint ist der Verbindungspunkt eines kabelbasierten
Netzwerkes zu einem WLAN. Der Accesspoint ist Basisstation f"ur alle
WLAN-Clienten, zu der sie eine drahtlose Verbindung aufbauen.
Sendet ein WLAN-Client Daten, die f"ur einen Empf"anger im kabelbasierten
Netzwerkteil bestimmt sind, so \emph{reicht} der Accesspoint diese Daten "uber
das Kabelnetz an den Empf"anger weiter. Weiterhin kann ein Accesspoint
auch mehrere WLAN-Clienten untereinander verbinden. Somit ist der Accesspoint
quasi ein kabelloser Switch. 

Dieser hat (je nach Austattung) einige der folgenden Optionen: 
\begin{itemize}
	\item Ein oder mehrer integrierte WLAN-Module
	\item Einen integrierten DHCP-Server 
	\item Umfangreiche Sicherheits- und Verschl"usselungsmoglichkeiten 
	

		\item WEP, WPA und WPA2 dienen der Verschl"usselung der zu ubetragenden Daten 
		\item MAC-Filter und SSID Optionen 
		\item Einstellungen bez"uglich dem Remotezugriff 

	
	\item Verschiedene Arbeitsmodi 

		\item Accesspoint (AP) 
		\item Bridge (Point-to-Point oder Point-to-Multipoint) 
		\item Repeater
		\item MESSID 

\end{itemize}
 
Intel PRO/Wireless 5000 

\begin{itemize}
  \item \url{http://support.intel.com/support/wireless/wlan/pro5000/accesspoint}
  \item \url{http://www.pcmag.com/article2/0,1759,5524,00.asp}
\end{itemize}

Linksys WAP55AG 

\begin{itemize}
  \item \url{http://www.tomsnetworking.de/content/aktuelles/news_beitrag/news/851/6/index.html}
\end{itemize}

NETGEAR WAB102 

\begin{itemize}
  \item \url{http://kbserver.netgear.com/products/WAB102.asp}
  \item \url{http://reviews.cnet.com/wireless-access-points/netgear-wab102-802-11a/4505-3265_7-20708150.html}
 \item \url{http://archive.cert.uni-stuttgart.de/bugtraq/2003/12/msg00159.html}
\end{itemize}
