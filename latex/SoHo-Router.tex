\subsubsection{SoHo-Router}

Man kann herkommliche WLAN-Router fur Heimanwender (SoHO-Router -small
or home office)zu kaufen, die sich mit alternativer Firmware (spezielle
Linux software mit OLSR Implementierung) zu einem Mesh-Router umrusten
lassen. Ein WLAN-Router ist die Kombination von eines normalen Routers
(Kabelrouter) und mit einem Accesspoint. Es gibt solche mit eingebauten
Modem und andere mit einem Anschluss (WAN-Port) daf�r (f�r Modems mit
LAN-Anschluss). Ein Nachteil ist, dass es viele Modelle gibt, die eine
fix verbaute Antenne haben, die nicht gewechselt werden kann.

Kosten in der Regel etwa 40-80 euro, haben gute Reichweite, sind klein
und handlich.

Vorteile:
\begin{itemize}
\item klein
\item mobil
\item g�nstig
\item gute Reichweite
\item wenig Strom 
\end{itemize}

Nachteile:
\begin{itemize}
\item meistens fix verbaute Antenne 
\end{itemize}

Durch das �ffnen von Ger�ten und das Einspielen von fremder Firmware
erlischt die Garantie des Herstellers !!!

\paragraph{OpenWRT}

OpenWRT ist eine GNU/Linux-Distribution f�r WLAN-Router. Anstatt einer
statischen Firmware setzt OpenWRT auf ein voll beschreibbares Dateisystem
sowie einen Paketmanager. OpenWRT l�uft unter anderem auf Ger�ten der
Firmen Linksys, ALLNET, ASUS, Belkin, Buffalo, Microsoft und Siemens.

Vorteile:
\begin{itemize}
\item Flexibilit�t
\item Erweiterbarkeit
\item Individualisierbarkeit
\item Sicherheit
\item Gewohnte Linux-Flexibilit�t und Funktionsumfang!!! 
\end{itemize}

Nachteile:
\begin{itemize}
\item Standardm��ig sind nur die n�tigsten Unix-Tools vorhanden 
\end{itemize}

Links:
\begin{itemize}
\item \url{http://openwrt.org/}
\item \url{http://toh.openwrt.org/}
\end{itemize}

%%%%%%%%%%%%%%%%%%%%%%%%%%%%%%%%%%%%%%%%%%%%%%%%%%%%%%%%%%%%%%%%%%%%%%%%%%%%
%
% Linksys WRT54G v1.0
%
%%%%%%%%%%%%%%%%%%%%%%%%%%%%%%%%%%%%%%%%%%%%%%%%%%%%%%%%%%%%%%%%%%%%%%%%%%%%
\begin{wlandevice}{Linksys WRT54G v1.0}

\wlanimage{Linksys_WRT54G}{Linksys WRT54G v1.0}

\begin{wlanieeestandard}
\item 802.11b/g
\item 802.11a/b/g (wenn man die mitgelieferte Mini-PCI WLAN-Karte
durch z.B. Atheros 802.11a/b/g WLAN-Karte austauscht)
\end{wlanieeestandard}

\begin{wlanmode}
\item Ad-Hoc
\item Infrastruktur
\end{wlanmode}

\begin{wlanfirmware}
\item
Es sind mehrere fremde frei verf�gbare Firmware f�r dieses Ger�t.
Alle unten aufgef�hrten Firmware sind Open-Source Projekte:
OpenWRT \url{http://wiki.openwrt.org/OpenWrtDocs/Hardware/Linksys/WRT54G}
DD-WRT \url{http://www.dd-wrt.com/wiki/index.php/Linksys_WRT54G/GL/GS/GX}
\end{wlanfirmware}

\wlanprice{40-50}

\begin{wlaninstall}
\item
Die mitgelieferte Mini-PCI WLAN-Karte durch z.B. Atheros 802.11a Mini-PCI
austauschen und oben erw�hnte frei verf�gbare Firmware installieren
(siehe oben Firmware).
\end{wlaninstall}

\begin{wlanextrainfo}
\item
Ein Mini-PCI Slot ist f�r eine WLAN-Karte vorhanden.
\end{wlanextrainfo}

\begin{wlanlink}
\item \url{http://wiki.openwrt.org/OpenWrtDocs/Hardware/Linksys/WRT54G}
\item \url{http://www.dd-wrt.com/wiki/index.php/Linksys_WRT54G/GL/GS/GX}
\item \url{http://forum.opennet-initiative.de/thread.php?threadid=505&sid=56c53647db6353a41e9a3100f00d02c4}
\item \url{http://www.linksysinfo.org/forums/showthread.php?t=47124}
\end{wlanlink}

\end{wlandevice}

%%%%%%%%%%%%%%%%%%%%%%%%%%%%%%%%%%%%%%%%%%%%%%%%%%%%%%%%%%%%%%%%%%%%%%%%%%%%
%
% Linksys WRT55AG
%
%%%%%%%%%%%%%%%%%%%%%%%%%%%%%%%%%%%%%%%%%%%%%%%%%%%%%%%%%%%%%%%%%%%%%%%%%%%%
\begin{wlandevice}{Linksys WRT55AG}

\wlanimage{Linksys_WRT55AG}{Linksys WRT55AG}

\begin{wlanieeestandard}
\item 802.11a/b/g
\end{wlanieeestandard}

\begin{wlanmode}
\item Ad-Hoc
\item Infrastruktur
\end{wlanmode}

\begin{wlanfirmware}
\item
Open-Source Firmware befindet sich noch in Entwicklung.
Modifizierte Version von OpenWRT Kamikaze
\url{http://legacy.not404.com/cgi-bin/trac.fcgi/wiki/OpenWRT/Atheros/Linksys/WRT55AGv2#KamikazeKernelonWRT55AGv2}
OpenWRT
\url{http://wiki.openwrt.org/OpenWrtDocs/Hardware/Linksys/WRT55AG}
\end{wlanfirmware}

\wlanprice{70-80}

\begin{wlanextrainfo}
\item
2xMini-PCI Slots sind f�r WLAN-Karten vorhanden.
\end{wlanextrainfo}

\begin{wlanlink}
\item \url{http://wiki.openwrt.org/OpenWrtDocs/Hardware/Linksys/WRT55AG}
\item \url{http://www.tomsnetworking.de/content/tests/j2003a/test_linksys_wrt55ag/index.html}
\item \url{http://reviews.cnet.com/routers/linksys-wrt55ag-wireless-a/4505-3319_7-21131921.html}
\item \url{http://legacy.not404.com/cgi-bin/trac.fcgi/wiki/OpenWRT/Atheros/Linksys/WRT55AGv2}
\end{wlanlink}

\end{wlandevice}

%%%%%%%%%%%%%%%%%%%%%%%%%%%%%%%%%%%%%%%%%%%%%%%%%%%%%%%%%%%%%%%%%%%%%%%%%%%%
%
% Asus WL500G/GP
%
%%%%%%%%%%%%%%%%%%%%%%%%%%%%%%%%%%%%%%%%%%%%%%%%%%%%%%%%%%%%%%%%%%%%%%%%%%%%
\begin{wlandevice}{Asus WL500G/GP}

\wlanimage{Asus_WL500G}{Asus WL500G/GP}

\begin{wlanieeestandard}
\item 802.11b/g
\item 802.11a/b/g (wenn man die mitgelieferte Mini-PCI WLAN-Karte
durch z.B. Atheros 802.11a/b/g WLAN-Karte austauscht)
\end{wlanieeestandard}

\begin{wlanmode}
\item Ad-Hoc
\item Infrastruktur
\end{wlanmode}

\begin{wlanfirmware}
\item
Es sind mehrere fremde frei verf�gbare Firmware f�r dieses Ger�t.
Alle unten aufgef�hrten Firmware sind Open-Source Projekte:
OpenWRT
\url{http://wiki.openwrt.org/OpenWrtDocs/Hardware/Asus/WL500G}
\url{http://wiki.openwrt.org/OpenWrtDocs/Hardware/Asus/WL500GP}
FreeWRT
\url{http://www.freewrt.org/trac/wiki/Documentation/Hardware/AsusWL500G}
\url{http://www.freewrt.org/trac/wiki/Documentation/Hardware/AsusWL500GP}
Olegs custom firmware
\url{http://oleg.wl500g.info}
\end{wlanfirmware}

\wlanprice{70-80}

\begin{wlaninstall}
\item
Die mitgelieferte Mini-PCI WLAN-Karte durch z.B. Atheros 802.11a Mini-PCI
austauschen und oben erw�hnte frei verf�gbare Firmware installieren
(siehe oben Firmware).
\url{http://wiki.opennet-initiative.de/index.php/Mini-PCI_Umbau}
\end{wlaninstall}

\begin{wlanextrainfo}
\item
Ein Mini-PCI Slot ist f�r eine WLAN-Karte vorhanden.
\end{wlanextrainfo}

\begin{wlanlink}
\item \url{http://wiki.opennet-initiative.de/index.php/AP9}
\item \url{http://wiki.openwrt.org/OpenWrtDocs/Hardware/Asus/WL500G}
\item \url{http://wiki.openwrt.org/OpenWrtDocs/Hardware/Asus/WL500GP}
\item \url{http://www.freewrt.org/trac/wiki/Documentation/Hardware/AsusWL500G}
\item \url{http://www.freewrt.org/trac/wiki/Documentation/Hardware/AsusWL500GP}
\item \url{http://wl500g.dyndns.org/}
\item \url{http://oleg.wl500g.info/}
\item \url{http://au.asus.com/products.aspx?l1=12&l2=43}
\item \url{http://www.freifunk-bno.de/component/option,com_smf/Itemid,88/topic,910.msg10357/}
\item \url{http://www.cyber-wulf.de/a_wl500g.html}
\item \url{http://wiki.openwrt.org/OpenWrtDocs/Hardware/Asus/WL500G}
\item \url{http://forum.opennet-initiative.de/print.php?threadid=505&page=6&sid=460903353d70c65fad4960105ab76cdd}
\item \url{http://forum.openwrt.org/viewtopic.php?pid=41756}
\item \url{http://www.familie-prokop.de/asus-wl500gp/index.html}
\end{wlanlink}

\end{wlandevice}

\paragraph{Andere WLAN-Router}

\begin{itemize}

\item Netgear HR314\\
802.11a WLAN-Router, unterst�tzt Ad-Hoc- und Infrastruktur-Modus,\\
keine Open-Source Firmware vorhanden, kostet ca. 30 Euro\\
\url{http://www.wi-fiplanet.com/reviews/article.php/1559091}

\end{itemize}
