\section{Aufgabenstellung}

Recherche im WWW durchfuhren, 
ob es 802.11a Karten gibt, die im Ad-hoc-Modus arbeiten? 
ob es neben einzelnen Karten auch komplette stand-alone Mesh-Produkte gibt, die 802.11a kompatibel sind? 


\subsection{Anforderungen}

Mesh Netzwerk fur die Forschung der Ad-Hoc Routing-Protokolle aufzubauen. 

Nachtrag: 
ganze Gehause abdecken 
Konfigurierung und Instrumentierung 
Topologie verandern bzw. erfassen 
Abfragen Visualisieren 
Weitere Schnittstellen (keine Funk) 
Betriebsystem vorschlagen.. 
Routing Protokolle auswechelbar.. (daemon start, exit..) 
25.000 Euro 
nicht zu viel Basteln 
MESH STANDART 11n Draft als Vorteil 
4-5 Mesh-Knoten pro Quadrat 
Vollstandige Verbindung mit Uni-Netz/Internet ist nicht gestattet 
nur eine Richtung (UNItoMesh) fur die Verwaltung ) 
Gateways 
AUFWANDSCHATZUNG (wie viele Knoten usw. ) 
Bei Router - Speicherkapazitat wichtig (falls uberhaupt in Frage kommt) 
FOCUS -> PC + Wlan-Karten + 
MIMO System (PC + 2 Wlan-Karten) 
PDA, Handy 11a 

3.2.2. Weitere Anforderungen (aus der Diskussion): 

Abdeckung des Gebaudes Universitatsstra?e 38 

Zusatzlich zu Wireless Mesh Network auch weitere (Netzwerk-)Schnittstelle zur Verwaltung vorsehen 

OS nicht festgelegt, soll Ergebnis der Fachstudie sein 

Freiheit bei Routingprotokollen 

Budget max. 25.000 Euro (evtl. mehr in Zukunft) 

Verbindung mit Informatik-Netz nur uber Gateway mit strikter Filterung 

Nach Moglichkeit keine selber gebastelten Losungen 

Schon ware, die angestrebte Standardisierung von Mesh-Netzen zu unterstutzen 

PDAs (bzw. andere kleine Clients) mit 802.11a? 

Funk auf verschiedenen Frequenzbandern (Performanceverbesserung) 

Separates Gateway notwendig? Vermutlich sinnvoll. 

3.2 Rahmenbediengung

IEEE 802.11a kompatibel (5Ghz-Frequenzen) 
Ad-hoc Modus 
Innenbereich (Uni-Stuttgart Informatik Gebaude 2. Stock) 
Nicht zu teuer 
Treiber fu Linux (und Windows) 
Open-Source Firmware fur Router 

\section{Existierende Losungen und Projekte}

FreiFunk http://freifunk.net/wiki/Meshing 
OpenNet http://wiki.opennet-initiative.de/index.php/Hauptseite 
http://www-i4.informatik.rwth-aachen.de/mcg/projects/umic-mesh/ 
http://umic-mesh.net/