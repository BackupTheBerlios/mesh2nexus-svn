\section{Aufgabenstellung}

F"ur Forschungszwecke soll innerhalb des SFB Nexus
(\url{http://www.nexus.uni-stuttgart.de}) 
ein WMN installiert werden. 

Dieses WMN dient 
\begin{itemize}
	\item einerseits Nexus-Anwendungen, insbesondere Anwendungen 
	auf mobilen Ger"aten, als \emph{Kommunikationsmedium}. 

	\item andererseits auch als \emph{Testbed} 
	zur Erforschung verschiedener Erweiterungen von WMNs.
\end{itemize}

Dieses WMN soll beispielsweise der Untersuchung neuartige kontextbezogener 
Kommunikationsmechanismen, der Erforschung von 
Publish/Subscribe-Diensten f"ur WMNs oder der 
Verwaltung von Umgebungsmodellen innerhalb eines 
hybriden Systems wie es ein WMN darstellt, dienen.

Ziel dieser Fachstudie ist die Ausarbeitung einer 
Empfehlung f"ur die Beschaffung entsprechender Ger"ate 
(\emph{Hardwareplattformen und Systemsoftware}) f"ur
den Aufbau eines WMN. \\

Das Vorgehen umfasst im einzelnen:

\begin{itemize}
	
	\item Einarbeitung in grundlegende WMN-Technologien
	\item Analyse der Anforderungen des Nexus-Projektes an ein WMN
	\item Erstellung einer "Ubersicht "uber aktuelle verf"ugbare 
	Hardwareplattformen und Systemsoftware f"ur WMN
	\item Bewertung der analysierten Systeme hinsichtlich 
	der ermittelten Anforderungen 	
	\item Ausarbeitung einer Empfehlung f"ur eine geeignetes 
	WMN hinsichtlich Hardwareplattform und Systemsoftware
	
\end{itemize}


\section{Anforderungen}

Nach der Einarbeitung in WMN-Technologien und 
Analyse der Anforderungen des Nexus-Projektes
wurde folgendes festgehalten:

\begin{itemize}
	
	\item IEEE 802.11a kompatibel (5Ghz-Frequenzen) 
	
	ob es 802.11a Karten gibt, die im Ad-hoc-Modus arbeiten? 
	ob es neben einzelnen Karten auch komplette stand-alone 
	Mesh-Produkte gibt, die 802.11a kompatibel sind?

	\item Ad-hoc Modus (erkl"ahrung)
	
	\item Treiber f"ur Linux (und Windows) 

	\item Open-Source Firmware f"ur Router 

	\item Abdeckung des Geb"audes Universit"atsstra"se 38 

	\item Zusatzlich zu Wireless Mesh Network auch weitere
	(Netzwerk-)Schnittstellen zur Verwaltung vorsehen 

	\item OS nicht festgelegt, soll Ergebnis der Fachstudie sein 

	\item Betriebsystem vorschlagen

	\item Freiheit bei Routingprotokollen 

	\item Routing Protokolle auswechelbar (daemon start, exit..) 

	\item Konfigurierung und Instrumentierung 

	\item Topologie verandern bzw. erfassen 

	\item Abfragen Visualisieren 

	\item Nach M"oglichkeit keine selber gebastelten L"osungen 

	\item Schon w"are, die angestrebte Standardisierung von
	Mesh-Netzen zu unterstutzen 

	\item MESH STANDART 11n Draft als Vorteil 

	\item Verbindung mit Informatik-Netz nur "uber Gateway
	mit strikter Filterung 

	\item nur eine Richtung (UNItoMesh) f"ur die Verwaltung ) 

	\item AUFWANDSCHATZUNG (wie viele Knoten usw. ) 

	\item Budget max. 25.000 Euro (evtl. mehr in Zukunft) 

	\item 4-5 Mesh-Knoten pro Quadrat 

	\item Separates Gateway notwendig? Vermutlich sinnvoll. 

	\item Bei Router - Speicherkapazit"at wichtig
	(falls "uberhaupt in Frage kommt) 

	\item FOCUS -> PC + Wlan-Karten

	\item MIMO System (PC + 2 Wlan-Karten) Testen

	\item Funk auf verschiedenen Frequenzb"andern (Performanceverbesserung) 

	\item PDAs (bzw. andere kleine Clients) mit 802.11a?
	
\end{itemize}


