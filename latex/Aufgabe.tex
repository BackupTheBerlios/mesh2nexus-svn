\section{Aufgabenstellung}

F�r Forschungszwecke soll innerhalb des SFB Nexus (URL) 
ein WMN installiert werden. 

Dieses WMN dient 
\begin{itemize}

	\item einerseits Nexus-Anwendungen, insbesondere Anwendungen 
auf mobilen Ger�ten, als \emph{Kommunikationsmedium}. 

\item Andererseits soll dieses WNIN auch als \emph{Testbed} 
zur Erforschung verschiedene Erweiterungen von WMNs dienen, 

\end{itemize}

beispielsweise der Untersuchung neuartige kontextbezogener 
Kommunikationsmechanismen, der Erforschung von 
Publish/Subscribe-Diensten f�r WMNs oder der 
Verwaltung von Umgebungsmodellen innerhalb eines 
hybriden Systems wie es ein WMN darstellt. 

Ziel dieser Fachstudie ist die Ausarbeitung einer 
Empfehlung f�r die Beschaffung entsprechender Ger�te 
(\emph{Hardwareplattformen und Systemsoftware}) f�r
den Aufbau eines WMN.) \\

Das Vorgehen umfasst im einzelnen:

\begin{itemize}
	
	\item Einarbeitung in grundlegende WMN-Technologien
	\item Analyse der Anforderungen des Nexus-Projektes an ein WNN
	\item Erstellung einer �bersicht �ber aktuelle verf�gbare 
	Hardwareplattformen und Systemsoftware f�r WMN
	\item Bewertung der analysierten Systeme hinsichtlich 
	der ermittelten Anforderungen 	
	\item Ausarbeitung einer Empfehlung f�r eine geeignetes 
	WNN hinsichtlich Hardwareplattform und Systemsoftware
	
\end{itemize}


\section{Anforderungen}

Nach der Einarbeitung in WMN-Technologien und 
Analyse der Anforderungen des Nexus-Projektes
wurden folgendes festgehalten:

\begin{itemize}
	
	\item IEEE 802.11a kompatibel (5Ghz-Frequenzen) 
	
	ob es 802.11a Karten gibt, die im Ad-hoc-Modus arbeiten? 
	ob es neben einzelnen Karten auch komplette stand-alone 
	Mesh-Produkte gibt, die 802.11a kompatibel sind?

	\item Ad-hoc Modus (erkl�hrung)
	
	\item Treiber fu Linux (und Windows) 

	\item Open-Source Firmware fur Router 

	\item Abdeckung des Gebaudes Universitatsstra�e 38 

	\item Zusatzlich zu Wireless Mesh Network auch weitere (Netzwerk-)Schnittstelle zur Verwaltung vorsehen 

	\item OS nicht festgelegt, soll Ergebnis der Fachstudie sein 

	\item Betriebsystem vorschlagen.. 

	\item Freiheit bei Routingprotokollen 

	\item Routing Protokolle auswechelbar.. (daemon start, exit..) 

	\item Konfigurierung und Instrumentierung 

	\item Topologie verandern bzw. erfassen 

	\item Abfragen Visualisieren 

	\item Nach Moglichkeit keine selber gebastelten Losungen 

	\item Schon ware, die angestrebte Standardisierung von Mesh-Netzen zu unterstutzen 

	\item MESH STANDART 11n Draft als Vorteil 

	\item Verbindung mit Informatik-Netz nur uber Gateway mit strikter Filterung 

	\item nur eine Richtung (UNItoMesh) fur die Verwaltung ) 

	\item AUFWANDSCHATZUNG (wie viele Knoten usw. ) 

	\item Budget max. 25.000 Euro (evtl. mehr in Zukunft) 

	\item 4-5 Mesh-Knoten pro Quadrat 

	\item Separates Gateway notwendig? Vermutlich sinnvoll. 

	\item Bei Router - Speicherkapazitat wichtig (falls uberhaupt in Frage kommt) 

	\item FOCUS -> PC + Wlan-Karten + 

	\item MIMO System (PC + 2 Wlan-Karten) Testen.. 

	\item Funk auf verschiedenen Frequenzbandern (Performanceverbesserung) 

	\item PDAs (bzw. andere kleine Clients) mit 802.11a?
	
\end{itemize}


