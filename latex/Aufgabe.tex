\section{Aufgabenstellung}

F"ur Forschungszwecke soll innerhalb des SFB Nexus
(\url{http://www.nexus.uni-stuttgart.de}) 
ein WMN installiert werden. 

Dieses WMN dient 
\begin{itemize}
	\item einerseits Nexus-Anwendungen, insbesondere Anwendungen 
	auf mobilen Ger"aten, als \emph{Kommunikationsmedium}. 

	\item andererseits auch als \emph{Testbed} 
	zur Erforschung verschiedener Erweiterungen von WMNs.
\end{itemize}

Dieses WMN soll beispielsweise der Untersuchung neuartige kontextbezogener 
Kommunikationsmechanismen, der Erforschung von 
Publish/Subscribe-Diensten f"ur WMNs oder der 
Verwaltung von Umgebungsmodellen innerhalb eines 
hybriden Systems wie es ein WMN darstellt, dienen.

Ziel dieser Fachstudie ist die Ausarbeitung einer 
Empfehlung f"ur die Beschaffung entsprechender Ger"ate 
(\emph{Hardwareplattformen und Systemsoftware}) f"ur
den Aufbau eines WMN. \\

Das Vorgehen umfasst im einzelnen:

\begin{itemize}
	
	\item Einarbeitung in grundlegende WMN-Technologien
	\item Analyse der Anforderungen des Nexus-Projektes an ein WMN
	\item Erstellung einer "Ubersicht "uber aktuelle verf"ugbare 
	Hardwareplattformen und Systemsoftware f"ur WMN
	\item Bewertung der analysierten Systeme hinsichtlich 
	der ermittelten Anforderungen 	
	\item Ausarbeitung einer Empfehlung f"ur eine geeignete 
	WMN hinsichtlich Hardwareplattform und Systemsoftware
	
\end{itemize}


\section{Anforderungen}
(TODO) korrigieren, vielleicht ganze S"atze machen!

Nach der Einarbeitung in WMN-Technologien und 
Analyse der Anforderungen des Nexus-Projektes
wurde folgendes festgehalten:

\subsection{Anforderungen an Hardware}
\begin{itemize}
	\item IEEE 802.11a kompatibel (5Ghz Frequenzen) 
	\item Ad-hoc Modus (auch bei 5Ghz Frequenzen) 
	\item Treiber fur Linux (und Windows) 
	\item Open-Source Firmware (nur f"ur Router) 
	\item Zusatzlich zu Wireless Mesh Network auch weitere (Netzwerk-)Schnittstelle zur Verwaltung vorsehen 
	\item Nach Moglichkeit keine selber gebastelten Losungen 
	\item Sch"on ware, die angestrebte Standardisierung von Mesh-Netzen zu unterstutzen 
				(MESH STANDART 11n Draft als Vorteil)
	\item 2 Mini PCI-Slots (nur f"ur Router) - 
				Funk auf verschiedenen Frequenzbandern (Performanceverbesserung) 
\end{itemize}

\subsection{Anforderungen an Software}
\begin{itemize}
	\item Betriebsystem ist nicht festgelegt, soll Ergebnis der Fachstudie sein
	\item Freiheit bei Routingprotokollen (Austauschbarkeit)
	\item Software soll Konfigurierung und Instrumentierung von WMN erm"oglichen
	\item Topologie verandern bzw. erfassen 
	\item Visualisierung von Anfragen und Topologie 
\end{itemize}

\subsection{Zusatzliche Anforderungen}
\begin{itemize}
	\item Abdeckung des Gebaudes Universitatsstra?e 38
	\item Verbindung mit Informatik-Netz nur "uber Gateway mit strikter Filterung
	      (nur eine Richtung (UNI->Mesh) f"ur die Verwaltung)
	\item Separates Gateway notwendig? Vermutlich sinnvoll. 
	\item AUFWANDSCHATZUNG (wie viele Knoten usw. )
	\item Budget max. 25.000 Euro (evtl. mehr in Zukunft)
	\item 4-5 Mesh-Knoten pro Quadrat 
	\item Speicherkapazitat wichtig (nur bei Router)
	\item FOCUS: PC + Wlan-Karten, da mehr Flexibilit"at
	\item PC + 2 Wlan-Karten vorteilhaft
	\item PDAs (bzw. andere kleine Clients) mit 802.11a
\end{itemize}


