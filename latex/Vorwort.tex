\section{Einleitung}

In diesem Abschnitt werden einige wichtige Begriffe, die im Laufe des
Dokument auftauchen werden, kurz erlautert.

\subsection{Grundlagen von Mesh Netzen}

Hintergrund:

Ein drahtloses vermaschtes Netz (engl. Wireless Mesh Network, WMN) besteht aus einer Menge von Knoten, die �ber drahtlose Kommunikationstechniken wie beispielsweise IEEE 802.11 Nachrichten austauschen. Die Vemaschung der Knoten erm�glicht dabei nicht nur den Austausch von Nachrichten zwischen unmittelbar benachbarten Knoten, sondern auch die Vermittlung von Nachrichten an entfernte Knoten �ber mehrere Knoten hinweg. Die Vermittlungsfunktionalit�t wird dabei oft von dedizierten Vermittlungsknoten (engl. Mesh Router) bereitgestellt, die somit eine drahtlose Kommunikationsinfrastruktur f�r die Klienten (engt. Mesh Client) bilden. Durch den Einsatz vergleichsweise kosteng�nstiger Hardwarekomponenten und die Vermaschung der Knoten erm�glichen WMNs die kosteng�nstige Vernetzung auch gr��erer Gebiete. Entsprechende Netze, werden beispielsweise von Community-Projekten wie dein Freifunkprojekt oder Firmen wie Google bereits heute in der Praxis f�r den Aufbau gr��erer Netze eingesetzt, um beispielsweise kosteng�nstige Internetzug�nge f�r Stadtteile oder ganze St�dte zu realisieren.

WMNs sind auch f�r den Sonderforschungsbereichs (SFB) Nexus an der Universit�t Stuttgart 
\url{http://www.nexus.uni-stuttgart.de} von gro�em Interesse. Im Zentrum der Forschungen des SFB stehen Umgebungsmodelle f�r mobile kontextbezogene Systeme. Umgebungsmodelle sind digitale Abbilder der physischen Welt, die von kontextbezogenen Systemen genutzt werden, um sich selbst�ndig an die physische Umgebung des Benutzers anzupassen. Ein einfaches Beispiel sind ortsbezogene Anwendungen, die beispielsweise aufgrund der aktuellen geographischen Position eines Ger�ts automatisch Informationen �ber nahe Restaurants. Sehensw�rdigkeiten, usw. selektieren k�nnen. Zur Kommunikation. insbesondere mit mobilen Ger�ten, werden dabei hybride Systeme betrachtet, in denen sowohl eine infrastrukturbasierte Kommunikation als auch die direkte Ad-hoc-Kommunikation zwischen mobilen Endsystemen m�glich ist. Hierbei spielen WMNs als eine spezielle Auspr�gung eines hybriden Kommunikationssystems eine wesentliche Rolle.

\subsubsection{Mesh-Netz}

In einem vermaschten Netz (Mesh-Netz) ist jeder Netzwerkknoten mit einem
oder mehreren anderen verbunden. Die Informationen werden von Knoten
zu Knoten weitergereicht, bis sie das Ziel erreichen. Vermaschte
Netze sind im Regelfall selbstheilend und dadurch sehr zuverlassig:
Wenn ein Knoten oder eine Verbindung blockiert ist oder ausfallt, kann
sich das Netz darum herum neu stricken. Die Daten werden umgeleitet und
das Netzwerk ist nach wie vor betriebsfahig. 

\subsubsection{Ad-Hoc}

Ein Ad-hoc-Netz bezeichnet in der Informationstechnologie eine drahtlose
Netzwerktopologie zwischen zwei oder mehr Endgeraten, die ohne feste
Infrastruktur auskommt.

\subsubsection{IEEE 802.11a/b/g}

\textbf{IEEE 802.11} (auch: Wireless LAN, WLAN, WiFi) bezeichnet eine IEEE-Norm
fur drahtlose Netzwerkkommunikation. Herausgeber ist das Institute of
Electrical and Electronics Engineers (IEEE).

\textbf{802.11a} spezifiziert eine weitere Variante der physikalischen Schicht,
die im 5-GHz-Band arbeitet und Ubertragungsraten bis zu 54 MBit/s
ermoglicht. 

\textbf{802.11b} ist ebenfalls eine alternative Spezifikation der physikalischen
Schicht, die mit dem bisher genutzten 2,4-GHz-Band auskommt und
Ubertragungsraten bis zu 11 MBit/s ermoglicht. 

\textbf{2,4-GHz-Vorteile }
\begin{itemize}	
	\item gebuhrenfreies freigegebenes ISM-Frequenzband 
	\item hohe Verbreitung und daher geringe Geratekosten 
\end{itemize}

\textbf{2,4-GHz-Nachteile }
\begin{itemize}	
	\item Frequenz muss mit anderen Geraten/Funktechniken geteilt werden (Bluetooth,
Mikrowellenherde, etc.) 
	\item storungsfreier Betrieb von nur maximal 3 Netzwerken am selben Ort moglich, 
da effektiv nur 3 brauchbare (kaum uberlappende) Kanale zur Verfugung
stehen (in Deutschland: 1, 7, 13) 
\end{itemize}

\textbf{5-GHz-Vorteile }
\begin{itemize}
	\item weniger genutztes Frequenzband, dadurch haufig storungsfreierer Betrieb
moglich 
	\item in Deutschland 19 (bei BNetzA-Zulassung) nicht uberlappende Kanale 
	\item hohere Reichweite, da mit 802.11h bis zu 1000 mW Sendeleistung moglich 
\end{itemize}

\textbf{5-GHz-Nachteile }
\begin{itemize}
	\item starkere Regulierungen in Europa: auf den meisten Kanalen DFS notig 
	\item auf einigen Kanalen kein Betrieb im Freien erlaubt 
	\item falls kein TPC
benutzt wird, muss die Sendeleistung reduziert werden 
	\item \textbf{Ad-hoc-Modus wird von den meisten Geraten nicht unterstutzt }
	\item \textbf{geringere Verbreitung, daher wenig verfugbare Gerate auf dem Markt und
hohe Kosten }
\end{itemize}



\subsection{Existierende Losungen und Projekte}

FreiFunk http://freifunk.net/wiki/Meshing 
OpenNet http://wiki.opennet-initiative.de/index.php/Hauptseite 
http://www-i4.informatik.rwth-aachen.de/mcg/projects/umic-mesh/ 
http://umic-mesh.net/