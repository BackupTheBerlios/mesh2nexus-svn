\section{Einleitung}

In diesem Abschnitt werden einige wichtige Begriffe, die im Laufe des
Dokuments auftauchen werden, kurz erl"autert.

\subsection{Grundlagen von Mesh-Netzen}

\subsubsection{Hintergrund}

Ein drahtloses vermaschtes Netz (engl. Wireless Mesh Network, WMN) besteht
aus einer Menge von Knoten, die "uber drahtlose Kommunikationstechniken
wie beispielsweise IEEE 802.11 Nachrichten austauschen. Die
Vemaschung der Knoten erm"oglicht dabei nicht nur den Austausch von
Nachrichten zwischen unmittelbar benachbarten Knoten, sondern auch die
Vermittlung von Nachrichten an entfernte Knoten "uber mehrere Knoten
hinweg. Die Vermittlungsfunktionalit"at wird dabei oft von dedizierten
Vermittlungsknoten (engl. Mesh Router) bereitgestellt, die somit eine
drahtlose Kommunikationsinfrastruktur f"ur die Klienten (engl. Mesh
Client) bilden. Durch den Einsatz vergleichsweise kosteng"unstiger
Hardwarekomponenten und die Vermaschung der Knoten erm"oglichen WMNs die
kosteng"unstige Vernetzung auch gr"o"serer Gebiete. Entsprechende Netze
werden beispielsweise von Community-Projekten wie das Freifunk-Projekt
oder Firmen wie Google bereits heute in der Praxis f"ur den Aufbau gr""serer
Netze eingesetzt, um beispielsweise kosteng"unstige Internetzug"ange f"ur
Stadtteile oder ganze St"adte zu realisieren.

WMNs sind auch f"ur den Sonderforschungsbereich (SFB) Nexus an der
Universit"at Stuttgart
\url{http://www.nexus.uni-stuttgart.de} von gro"sem Interesse. Im
Zentrum der Forschungen des SFB stehen Umgebungsmodelle f"ur mobile
kontextbezogene Systeme. Umgebungsmodelle sind digitale Abbilder der
physischen Welt, die von kontextbezogenen Systemen genutzt werden, um
sich selbst"andig an die physische Umgebung des Benutzers anzupassen. Ein
einfaches Beispiel sind ortsbezogene Anwendungen, die beispielsweise
aufgrund der aktuellen geographischen Position eines Ger"ats automatisch
Informationen "uber nahe Restaurants, Sehensw"urdigkeiten, usw. selektieren
k"onnen. Zur Kommunikation, insbesondere mit mobilen Ger"aten, werden dabei
hybride Systeme betrachtet, in denen sowohl eine infrastrukturbasierte
Kommunikation als auch die direkte Ad-hoc-Kommunikation zwischen
mobilen Endsystemen m"oglich ist. Hierbei spielen WMNs als eine spezielle
Auspr"agung eines hybriden Kommunikationssystems eine wesentliche Rolle.

\subsubsection{Mesh-Netz}

In einem vermaschten Netz (Mesh-Netz) ist jeder Netzwerkknoten mit einem
oder mehreren anderen verbunden. Die Informationen werden von Knoten
zu Knoten weitergereicht, bis sie das Ziel erreichen. Vermaschte
Netze sind im Regelfall selbstheilend und dadurch sehr zuverl"assig:
Wenn ein Knoten oder eine Verbindung blockiert ist oder ausf"allt, kann
sich das Netz darum herum neu stricken. Die Daten werden umgeleitet und
das Netzwerk ist nach wie vor betriebsf"ahig. 

\subsubsection{Ad-Hoc}

Ein Ad-hoc-Netz bezeichnet in der Informationstechnologie eine drahtlose
Netzwerktopologie zwischen zwei oder mehr Endger"aten, die ohne feste
Infrastruktur auskommt.

\subsubsection{IEEE 802.11a/b/g}

\textbf{IEEE 802.11} (auch: Wireless LAN, WLAN, WiFi) bezeichnet eine IEEE-Norm
f"ur drahtlose Netzwerkkommunikation. Herausgeber ist das Institute of
Electrical and Electronics Engineers (IEEE).

\textbf{802.11a} spezifiziert eine weitere Variante der physikalischen Schicht,
die im 5-GHz-Band arbeitet und "Ubertragungsraten bis zu 54 MBit/s
ermoglicht. 

\textbf{802.11b} ist ebenfalls eine alternative Spezifikation der physikalischen
Schicht, die mit dem bisher genutzten 2,4-GHz-Band auskommt und
"Ubertragungsraten bis zu 11 MBit/s erm"oglicht. 

\textbf{2,4-GHz-Vorteile}
\begin{itemize}	
	\item gebuhrenfreies freigegebenes ISM-Frequenzband 
	\item hohe Verbreitung und daher geringe Ger"atekosten 
\end{itemize}

\textbf{2,4-GHz-Nachteile}
\begin{itemize}	
	\item Frequenz muss mit anderen Ger"aten/Funktechniken geteilt werden
	(Bluetooth, Mikrowellenherde, etc.) 
	\item st"orungsfreier Betrieb von nur maximal 3 Netzwerken
	am selben Ort m"oglich, da effektiv nur 3 brauchbare
	(kaum "uberlappende) Kan"ale zur Verf"ugung stehen
	(in Deutschland: 1, 7, 13) 
\end{itemize}

\textbf{5-GHz-Vorteile }
\begin{itemize}
	\item weniger genutztes Frequenzband, dadurch h"aufig
	st"orungsfreierer Betrieb m"oglich 
	\item in Deutschland 19 (bei BNetzA-Zulassung) nicht "uberlappende
	Kan"ale 
	\item h"ohere Reichweite, da mit 802.11h bis zu 1000 mW Sendeleistung
	m"oglich 
\end{itemize}

\textbf{5-GHz-Nachteile }
\begin{itemize}
	\item st"arkere Regulierungen in Europa: auf den meisten Kan"alen DFS
	n"otig 
	\item auf einigen Kan"alen kein Betrieb im Freien erlaubt 
	\item falls kein TPC benutzt wird, muss die Sendeleistung reduziert
	werden 
	\item \textbf{Ad-hoc-Modus wird von den meisten Ger"aten
		nicht unterst"utzt }
	\item \textbf{geringere Verbreitung, daher wenig verf"ugbare Ger"ate
		auf dem Markt und hohe Kosten }
\end{itemize}



\subsection{Existierende L"osungen und Projekte}

\begin{itemize}
\item FreiFunk \url{http://freifunk.net/wiki/Meshing}
\item OpenNet \url{http://wiki.opennet-initiative.de/index.php/Hauptseite}
\item \url{http://www-i4.informatik.rwth-aachen.de/mcg/projects/umic-mesh/} 
\item \url{http://umic-mesh.net/}
\end{itemize}
