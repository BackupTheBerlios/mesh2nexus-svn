\section{Abstract}


\section{Einleitung}

In diesem Abschnitt werden einige wichtige Begriffe, die im Laufe des
Dokument auftauchen werden, kurz erlautert.

\subsection{Nexus}

Uni-Stuttgart - NEXUS 
 http://www.nexus.uni-stuttgart.de/
 
\subsection{Ad-Hoc}

Ein Ad-hoc-Netz bezeichnet in der Informationstechnologie eine drahtlose
Netzwerktopologie zwischen zwei oder mehr Endgeraten, die ohne feste
Infrastruktur auskommt.


\subsection{Mesh-Netz}

In einem vermaschten Netz (Mesh-Netz) ist jeder Netzwerkknoten mit einem
oder mehreren anderen verbunden. Die Informationen werden von Knoten
zu Knoten weitergereicht, bis sie das Ziel erreichen. Vermaschte
Netze sind im Regelfall selbstheilend und dadurch sehr zuverlassig:
Wenn ein Knoten oder eine Verbindung blockiert ist oder ausfallt, kann
sich das Netz darum herum neu stricken. Die Daten werden umgeleitet und
das Netzwerk ist nach wie vor betriebsfahig. In conjuction with the
research cluster UMIC, the Mobile Communications Group (MCG) @ Informatik
4 is building up a hybrid wireless mesh network testbed - UMIC-Mesh
(previously known as MCG-Mesh). The goal of this project is twofold. From
the scientific point of view the goal is to build a large and scalable
mesh network to conduct various networking studies. From the application
point of view the goal is to provide the members of the Computer Science
Department and the students with a simple and comfortable way to get
high bandwidth network access anywhere in the computer science center.

\subsection{IEEE 802.11a/b/g}

IEEE 802.11 (auch: Wireless LAN, WLAN, WiFi) bezeichnet eine IEEE-Norm
fur drahtlose Netzwerkkommunikation. Herausgeber ist das Institute of
Electrical and Electronics Engineers (IEEE).

802.11a spezifiziert eine weitere Variante der physikalischen Schicht,
die im 5-GHz-Band arbeitet und Ubertragungsraten bis zu 54 MBit/s
ermoglicht. 


802.11b ist ebenfalls eine alternative Spezifikation der physikalischen
Schicht, die mit dem bisher genutzten 2,4-GHz-Band auskommt und
Ubertragungsraten bis zu 11 MBit/s ermoglicht. 


2,4-GHz-Vorteile 
gebuhrenfreies freigegebenes ISM-Frequenzband 
hohe Verbreitung und daher geringe Geratekosten 

2,4-GHz-Nachteile 
Frequenz muss mit anderen Geraten/Funktechniken geteilt werden (Bluetooth,
Mikrowellenherde, etc.) 

storungsfreier Betrieb von nur maximal 3 Netzwerken am selben Ort moglich, 

da effektiv nur 3 brauchbare (kaum uberlappende) Kanale zur Verfugung
stehen (in Deutschland: 1, 7, 13) 


5-GHz-Vorteile 
weniger genutztes Frequenzband, dadurch haufig storungsfreierer Betrieb
moglich 


in Deutschland 19 (bei BNetzA-Zulassung) nicht uberlappende Kanale 
hohere Reichweite, da mit 802.11h bis zu 1000 mW Sendeleistung moglich 

5-GHz-Nachteile 
starkere Regulierungen in Europa: auf den meisten Kanalen DFS notig; 
auf einigen Kanalen kein Betrieb im Freien erlaubt; falls kein TPC
benutzt wird, muss die Sendeleistung reduziert werden 

Ad-hoc-Modus wird von den meisten Geraten nicht unterstutzt 
geringere Verbreitung, daher wenig verfugbare Gerate auf dem Markt und
hohe Kosten 


\section{Grundlagen von Mesh Netzen}

\subsection{Linux MadWiFi-Treiber}

Linux MadWifi-Treiber ist Linux Kernel Treiber fur WLAN-Karten mit
Atheros Chipsatz. Linux MadWifi-Treiber ist heutzutage einer der
fortgeschrittensten Linux Treiber fur WLAN-Karten. Der Treiber ist
stabil und hat eine gro?e Benutzergemeinschaft. Der MadWifi-Treiber
selbst ist Open-Source, verwendet aber eine propritare Softwareschicht
Hardware Abstraction Layer (HAL), die nur in binarer Form vorhanden
ist. 
 
 Das Hardware Abstraction Layer (HAL) wird vom MadWifi-Treiber
gebraucht, um die Atheros-Chips ansprechen zu konnen. Dafur wurde bisher
ein Closed-Source-Modul verwendet. Dies hat unter anderem damit zu tun,
dass die Atheros-Chipsatze prinzipiell auf Frequenzen funken konnten,
fur die sie nicht zugelassen sind - beispielsweise weil diese vom
Militar zur Kommunikation verwendet werden. 
 
 Durch das proprietare
Modul war der Madwifi-Treiber bisher jedoch von einer Aufnahme in den
Linux-Kernel ausgeschlossen. Die Entwickler hatten au?erdem das Problem,
dass sie Fehler unter Umstanden nicht beheben konnten, da sie nicht
nachvollziehen konnten, wie der HAL-Baustein arbeitet. 
 
 MadWifi
selbst wird daher ab sofort nicht weiterentwickelt. Stattdessen
setzen die Programmierer auf OpenHAL, eine Linux-Portierung des
HAL-Modules des in OpenBSD verfugbaren freien Atheros-Treibers. In der
Vergangenheit wurde vom Software Freedom Law Center (SFLC) bestatigt,
dass die durch Reverse Engineering entstandene Software keine Copyrights
verletzt. Solche Behauptungen hatten die Entwicklung lange ausgebremst. 


 Der neue Treiber "Ath5k" wird MadWifi nun ersetzen und soll nicht
nur die freie Komponente OpenHAL einsetzen, sondern auch mit dem neuen
Linux-WLAN-System Mac80211 zusammenarbeiten, so dass der Treiber in den
offiziellen Linux-Kernel gelangen kann. MadWifi soll jedoch weiter mit
Fehlerkorrekturen und HAL-Updates versorgt werden. 


\subsection{Ad-Hoc Routing-Protokolle}

\subsubsection{OLSR (Optimized Link State Routing)}

Optimized Link State Routing, kurz OLSR, ist ein Routingprotokoll
fur mobile Ad-hoc-Netze, das eine an die Anforderungen eines mobilen
drahtlosen LANs angepasste Version des Link State Routing darstellt. Es
wurde von der IETF mit dem RFC 3626 standardisiert. Bei diesem
verteilten flexiblen Routingverfahren ist allen Routern die vollstandige
Netztopologie bekannt, sodass sie von Fall zu Fall den kurzesten Weg zum
Ziel festlegen konnen. Als proaktives Routingprotokoll halt es die dafur
benotigten Informationen jederzeit bereit. 
 
 Ein in Mesh-Netzwerken
bekannter Vertreter von LSR ist OLSR von olsr.org. Inzwischen existieren
fur OLSR spezielle Erweiterungen. Mit der ETX-Erweiterung wird dem
Umstand Rechnung getragen, dass Links asymmetrisch sein konnen. Mit
dem Fisheye-Algorithmus ist OLSR auch fur gro?ere Netzwerke brauchbar
geworden, da Routen zu weiter entfernten Knoten weniger haufig neu
berechnet werden. Der entscheidende Nachteil ist aber der trotz
Fisheye-Algorithmus noch recht hohe Rechenaufwand von OLSRD, sobald
die Anzahl an Knoten ein gewisses Ma? ubersteigt (siehe Erfahrungen mit
den kapazitativ arg begrenzten CPUs der kleinen Meshrouter im Berliner
Freifunk-Netz). 
 


\subsubsection{B.A.T.M.A.N. (BETTER APPROACH TO MOBILE ADHOC NETWORKING)}

Ausgehend von den Erfahrungen mit Freifunk-OLSR begannen die Entwickler
aus der Freifunk-Community im Marz 2006 in Berlin damit, ein neues
Routingprotokoll fur drahtlose Meshnetzwerke zu entwickeln. Alle bisher
bekannten Routingalgorithmen versuchen, Routen entweder zu berechnen
(proaktive Verfahren) oder sie dann zu suchen, wenn sie gebraucht werden
(reaktive Verfahren). Das neue Protokoll B.A.T.M.A.N. berechnet oder
sucht im Gegensatz zu diesen Protokollen keine Routen ? es erfasst
lediglich, ob Routen zu anderen Knoten existieren und uberwacht ihre
Qualitat. Dabei interessiert es sich nicht dafur, wie eine Route verlauft,
sondern ermittelt lediglich, uber welchen direkten Nachbarn ein bestimmter
Netzwerkknoten am besten zu erreichen ist, und tragt diese Information
proaktiv in die Routingtabelle ein. 


