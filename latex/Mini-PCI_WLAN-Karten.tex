\subsubsection{Mini-PCI WLAN-Karten}

Mini-PCI ist eine vor allem f"ur die Nutzung in Notebooks und Laptops
miniaturisierte Version des PCI Steckplatzes, wie er in allen Desktop
PCs vorkommt. PCI steht dabei f"ur Peripheral Component Interconnect. Die
Abmessungen einer Mini-PCI Card betragen 6,0 x 4,6 x 0,5 cm.

Mini-PCI WLAN-Karten sind urspr"unglich f"ur Laptops gedacht, sind aber mit
entschprechenden Adaptoren (PCI-zu-MiniPCI) und externen Antennen auch
im normalen PCs zu verwenden. Als Vorteil ist dabei die Flexibilit"at zu
nehnen. Als Nachteil - die Zus"atzliche Kosten und Installationen. Meist
sind Mini-PCI Cards f"ur Wireless LAN bereits vom Hersteller eingebaut. Der
Vorteil der Ausf"uhrung als standardisiertes Modul liegt darin, da�
eine Mini-PCI Card in aller Regel einfach gegen eine andere Card - auch
eines anderen Herstellers - ausgetauscht werden kann. Im Falle der WLAN
Mini-PCI Module kann z.B. problemlos vom langsameren 802.11b Standard
auf ein schnelleres WLAN Modul nach 802.11g gewechselt werden.

TODO: \todo{Installationen?}
\textbf{Vorteile:}

\begin{itemize}
	\item K"onnen mit Hilfe eines Adapters zu einer PCI-WLAN-Karte umgebaut werden
	\item K"onnen leicht ausgetauscht werden 
\end{itemize}

\textbf{Nachteile:}

\begin{itemize}
		\item Meistens kostenintensiv
		\item Installationen
\end{itemize}

%%%%%%%%%%%%%%%%%%%%%%%%%%%%%%%%%%%%%%%%%%%%%%%%%%%%%%%%%%%%%%%%%%%%%%%%%%%%
%
% Wistron CM9 Atheros AR5213A
%
%%%%%%%%%%%%%%%%%%%%%%%%%%%%%%%%%%%%%%%%%%%%%%%%%%%%%%%%%%%%%%%%%%%%%%%%%%%%
\begin{wlandevice}{Wistron CM9 Atheros AR5213A}

\wlanimage{Wistron_CM9}{Wistron CM9 Atheros AR5213A}

\wlanchipset{Atheros AR5213A}

\begin{wlanieeestandard}
\item 802.11a/b/g
\end{wlanieeestandard}

\begin{wlanmode}
\item Ad-Hoc
\item Infrastruktur
\end{wlanmode}

\begin{wlansecurity}
\item WEP (40-, 104-, 128-bit)
\item WPA
\item WPA2
\end{wlansecurity}

\begin{wlandriver}
\item
Herrvorragende Unterst"utzung von MadWifi-Treiber \cite{madwifi},
auch Ad-Hoc-Modus.
\end{wlandriver}

\wlanprice{40}

\begin{wlaninstall}
\item
\url{http://madwifi.org/wiki/UserDocs/FirstTimeHowTo}
\end{wlaninstall}

\begin{wlanlink}
\item \url{http://www.alix-board.de/produkte/wistroncm9.html}
\item \url{http://www.pcengines.ch/cm9.htm}
\item \url{http://forum.openwrt.org/viewtopic.php?pid=10213}
\item \url{http://madwifi.org/ticket/1209}
\end{wlanlink}

\end{wlandevice}

%%%%%%%%%%%%%%%%%%%%%%%%%%%%%%%%%%%%%%%%%%%%%%%%%%%%%%%%%%%%%%%%%%%%%%%%%%%%
%
% Intel PRO/Wireless 3945
%
%%%%%%%%%%%%%%%%%%%%%%%%%%%%%%%%%%%%%%%%%%%%%%%%%%%%%%%%%%%%%%%%%%%%%%%%%%%%
\begin{wlandevice}{Intel PRO/Wireless 3945}

\wlanimage{Intel_3945ABG}{Intel PRO/Wireless 3945}

\wlanchipset{Intel}

\begin{wlanieeestandard}
\item 802.11a/b/g
\end{wlanieeestandard}

\begin{wlanmode}
\item Ad-Hoc
\item Infrastruktur
\end{wlanmode}

\begin{wlansecurity}
\item WEP (40-, 104-bit)
\item WPA
\item WPA2
\end{wlansecurity}

\begin{wlandriver}
\item
Es werden von Intel Treiber sowohl f"ur Windows als auch f"ur Linux
bereitgestellt.

\url{http://downloadcenter.intel.com/Product_Filter.aspx?ProductID=2259}

Von Intel wurde ein Projket f"ur die Unterst�tzung von Intel PRO/Wireless
3945 erstellt.

\url{http://ipw3945.sourceforge.net}

Der ipw3945-Treiber funktioniert auch im Ad-Hoc-Modus, aber nicht sehr stabil,
es kommt oft zu Verbindungsabbr"uchen.
\end{wlandriver}

\wlanprice{20-30}

\begin{wlaninstall}
\item
Im Gegensatz zu den "`klassischen"' Intel Wireless-Chips"atzen 2100- und
2200BG-Chips"atzen ist der Treiber f"ur den 3945ABG noch nicht im Kernel
verf"ugbar. Um auch damit kabellos ins Internet zu gehen,
sind ein paar Handgriffe notwendig.

\url{http://ipw3945.sourceforge.net/README.ipw3945}

\url{http://ipw3945.sourceforge.net/INSTALL}
\end{wlaninstall}

\begin{wlanlink}
\item \url{http://www.intel.com/network/connectivity/products/wireless/prowireless_mobile.htm}
\item \url{http://downloadcenter.intel.com/Product_Filter.aspx?ProductID=2259}
\item \url{http://ipw3945.sourceforge.net/}
\item \url{http://ipw3945.sourceforge.net/README.ipw3945}
\item \url{http://ipw3945.sourceforge.net/INSTALL}
\end{wlanlink}

\end{wlandevice}

%%%%%%%%%%%%%%%%%%%%%%%%%%%%%%%%%%%%%%%%%%%%%%%%%%%%%%%%%%%%%%%%%%%%%%%%%%%%
%
% Intel PRO/Wireless 2915
%
%%%%%%%%%%%%%%%%%%%%%%%%%%%%%%%%%%%%%%%%%%%%%%%%%%%%%%%%%%%%%%%%%%%%%%%%%%%%
\begin{wlandevice}{Intel PRO/Wireless 2915}

\wlanimage{Intel_2915ABG}{Intel PRO/Wireless 2915}

\wlanchipset{Intel}

\begin{wlanieeestandard}
\item 802.11a/b/g
\end{wlanieeestandard}

\begin{wlanmode}
\item Ad-Hoc
\item Infrastruktur
\end{wlanmode}

\begin{wlansecurity}
\item WEP (40-, 104-bit)
\item WPA
\item WPA2
\end{wlansecurity}

\begin{wlandriver}
\item
Es werden von Intel Treiber sowohl f"ur Windows als auch f"ur Linux
bereitgestellt.

\url{http://downloadcenter.intel.com/Product_Filter.aspx?ProductID=1847}

Von Intel wurde ein Projket f"ur die Unterst"utzung von Intel PRO/Wireless
2915 erstellt.

\url{http://ipw2200.sourceforge.net}

Der ipw2200-Treiber funktioniert auch im Ad-Hoc-Modus, aber nicht
sehr stabil, es kommt oft zu verbindungsabbr�chen. Der ipw2200-Treiber
ist im Kernel 2.6 enthalten, kann aber auch separat als Modul kompiliert
werden. Der im Kernel enthaltene Treiber unterst"utzt den Monitor-Modus
nicht.
\end{wlandriver}

\wlanprice{30}

\begin{wlaninstall}
\item
\url{http://ipw2200.sourceforge.net/README.ipw2200}

\url{http://ipw2200.sourceforge.net/INSTALL}
\end{wlaninstall}

\begin{wlanlink}
\item \url{http://support.intel.com/support/wireless/wlan/pro2915abg}
\item \url{http://download.intel.com/support/wireless/wlan/pro2915abg/sb/303330002us_channel.pdf}
\item \url{http://ipw2200.sourceforge.net/}
\item \url{http://www.intel.com/cd/personal/computing/emea/deu/234998.htm}
\item \url{http://downloadcenter.intel.com/Product_Filter.aspx?ProductID=1847}
\end{wlanlink}

\end{wlandevice}

%%%%%%%%%%%%%%%%%%%%%%%%%%%%%%%%%%%%%%%%%%%%%%%%%%%%%%%%%%%%%%%%%%%%%%%%%%%%
%
% Intel Wireless WiFi Link 4965AGN
%
%%%%%%%%%%%%%%%%%%%%%%%%%%%%%%%%%%%%%%%%%%%%%%%%%%%%%%%%%%%%%%%%%%%%%%%%%%%%
\begin{wlandevice}{Intel Wireless WiFi Link 4965AGN}

\wlanimage{Intel_4965AGN}{Intel Wireless WiFi Link 4965AGN}

\wlanchipset{Intel}

\begin{wlanieeestandard}
\item 802.11a/b/g/n(draft)
\end{wlanieeestandard}

\begin{wlanmode}
\item Ad-Hoc
\item Infrastruktur
\end{wlanmode}

\begin{wlansecurity}
\item WEP (40-, 104-bit)
\item WPA
\item WPA2
\end{wlansecurity}

\begin{wlandriver}
\item
\url{http://www.intellinuxwireless.org/}
\end{wlandriver}

\wlanprice{30}

\begin{wlaninstall}
\item
\url{http://www.intellinuxwireless.org/}
\end{wlaninstall}

\begin{wlanlink}
\item \url{http://www.intel.com/network/connectivity/products/wireless/wireless_n/overview.htm}
\item \url{http://www.intellinuxwireless.org/}
\item \url{http://www.wifi-info.de/intel-kuendigt-11n-chipsatz-fuer-centrino-notebooks-an/01/2007/}
\item \url{http://downloadcenter.intel.com/filter_results.aspx?strTypes=all&ProductID=2753&OSFullName=Linux*&lang=eng&strOSs=39&submit=Go\%21}
\end{wlanlink}

\end{wlandevice}
