\section{Hardware-Losungen fur den Aufbau eines Mesh-Netzwerkes}


Es gibt verschiedene Moglichkeiten ein Meshnetzwerk aufzubauen. Im Weiteren werden einige davon im Detail beschrieben. 

\subsection{PCs + WLAN-Karten}

Die einfachste Moglichkeit ware die Herkommlich en PCs mit WLAN-Karten zu einem Mesh-Router einzurichten. 

Man nimmt dabei einfach die Wlan-Karten (PCI, Mini-PCI oder PCMCIA)und baut diese in PCs oder in Laptops ein. 
Generelles Problem: 
 Ad-Hoc Modus bei Karten im 5Ghz Bereich ist von unausgereift bis nicht vorhanden. 

Hersteller haben gespart an der Entwicklung, da Ad-hoc modus einigerma?en kompliziert ist, und alle meist nur Infrastrukturmodus benutzt haben. Fehler liegen in Firmware von Chipsatz und im Treiber. 

Es gibt einen MadWiFi-Treiber, der fur eine Vielzahl von Chipsatzen entwickelt wurde und mit dem sollte es einigerma?en funktionieren, sobald dieser noch zusatzlich gepacht ist, und Firmware der Karte Ad-hoc zulasst. 

Generell wegen der geringen Verbreitung von 802.11a in Europa, sind nur wenige Karten erhaltlich. z.B konnten Karten mit Atheros Chipsatz, z.B AR5004X, uns weiterhelfen. 

Vorteile: 
Hardware kann noch nutzlich sein 
relativ einfache Installation 
Software Unterstutzung 
meherer WLAN- und Ethernet Interfaces moglich 

Nachteile: 
gross 
nich mobile 
Stromversorgung 
schlechte Sende- und Empfangqualitat, da die Antenne im elektromagnetischem Stornebel des PCs befindet 

\begin{wlandevice}{Linksys WMP55AG}

\wlanchipset
Atheros AR5213A

\wlanieeestandard
802.11a/b/g

\wlanmode
Ad-Hoc-Modus, Infrastruktur-Modus

\wlansecurity
 WPA
 LEAP
 WEP (40-, 104-, 128-bit)

\wlandriver
 Sehr gute Linux-Unterstutzung, madwifi-Treiber funktioniert
 mit dieser WLAN PCI-Karte ohne Probleme.
 Windows-Treiber werden von Linksys bereitgestellt.

\wlanprice
 ca. 90 Euro

\wlaninstall
 Lasst sich leicht sowohl unter Windows als auch unter Linux (madwifi-Treiber) installieren.
 http://madwifi.org/wiki/UserDocs/FirstTimeHowTo

 \end{wlandevice}
