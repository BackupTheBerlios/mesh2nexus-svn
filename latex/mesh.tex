\documentclass[a4paper, final, oneside, 2.8headlines]{scrartcl}

\setcounter{secnumdepth}{4}
\setcounter{tocdepth}{4}

\newcommand{\doctitle}{Hardwareplattformen und Systemsoftware 
f�r drahtlose vermaschte Kommunikationsnetze} % Dokumententitel
\newcommand{\version}{1.1.1} %Versionsnummer


%\usepackage[automark]{scrpage2}  % Kopf-/ Fusszeilen Version 1
%%erfordert in documentclass noch: headsepline, headinclude!
%\pagestyle{scrheadings}

\usepackage[automark]{scrpage2} % Kopf-/ Fusszeilen Version 2
\pagestyle{scrheadings}
\ohead{\includegraphics[height=1.2cm]{logo}}
\chead{}
\ihead{}

\usepackage{typearea} % Berechnung des Seitenspiegels
\usepackage{listings} % Code-Listings
\usepackage[pdftex]{graphicx}
\graphicspath{{images/}}
\usepackage[ngerman]{babel}
\usepackage[latin1]{inputenc}
\usepackage[T1]{fontenc}
\usepackage{ae,aecompl}
\usepackage[pdftex,hyperref,dvipsnames]{xcolor}
\usepackage{microtype} % optischer Randausgleich
\usepackage{array}
\usepackage{listings}
\usepackage{makecell}
\usepackage{color}

% Anfuehrungszeichen
\usepackage{xspace}
\newcommand{\enquote}[1]{\glqq #1\grqq \xspace}
\newcommand{\mc}[3]{\multicolumn{#1}{#2}{#3}}

% Links und PDF-Spezifika
\pdfcompresslevel=9   % Grafiken komprimieren
\usepackage{hyperref}
\definecolor{darkblue}{rgb}{0,0,.5}
\hypersetup{colorlinks=true, breaklinks=true, linkcolor=darkblue, 
menucolor=darkblue, pagecolor=darkblue, urlcolor=darkblue}
\hypersetup{pdftitle={\doctitle}}
\hypersetup{pdfsubject={Fachstudie "mesh"}}
\hypersetup{pdfauthor={S. Telejnikov, A. Egorenkov}}

\usepackage{float}
\restylefloat{figure}
\restylefloat{table}

% Farben:
\definecolor{todoDescription}{rgb}{0.5, 0.0, 0.75} 

%% Befehle f�r ToDos
\newcommand{\todo}[1][TODO]{\textcolor{red}{\textdagger}\marginpar{\tiny{\textbf{
\textcolor{todoDescription}{#1}}}}}

% Absatz Layout
\parindent 0cm
\parskip 2ex

% wlandevice environment
\newenvironment{wlandevice}[1]
{\newpage\paragraph{#1}\label{#1} \begin{description}}
{\end{description}}

\newcommand{\wlanimage}[2]
{
	\begin{figure}[H]
		\centering
		\includegraphics[width=0.5\textwidth]{#1}
		\caption{#2}
	\end{figure}
}

\newcommand{\wlanchipset}[1]
{\item[Chipsatz:] \rule{0mm}{0mm} \begin{itemize} \item #1 \end{itemize}}
\newcommand{\wlanprice}[1]
{\item[Preis:] \rule{0mm}{0mm} \begin{itemize} \item ca. #1 Euro \end{itemize}}

\newenvironment{wlanieeestandard}
{\item[IEEE Standards:] \rule{0mm}{0mm} \begin{itemize}}
{\end{itemize}}

\newenvironment{wlansecurity}
{\item[Sicherheit:] \rule{0mm}{0mm} \begin{itemize}}
{\end{itemize}}

\newenvironment{wlanmode}
{\item[Betriebsart:] \rule{0mm}{0mm} \begin{itemize}}
{\end{itemize}}

\newenvironment{wlandriver}
{\item[Treiber:] \rule{0mm}{0mm} \begin{itemize}}
{\end{itemize}}

\newenvironment{wlanfirmware}
{\item[Firmware:] \rule{0mm}{0mm} \begin{itemize}}
{\end{itemize}}

\newenvironment{wlaninstall}
{\item[Installation:] \rule{0mm}{0mm} \begin{itemize}}
{\end{itemize}}

\newenvironment{wlanextrainfo}
{\item[Weitere Informationen:] \rule{0mm}{0mm} \begin{itemize}}
{\end{itemize}}

\newenvironment{wlanlink}
{\item[Links:] \rule{0mm}{0mm} \begin{itemize}}
{\end{itemize}}

\definecolor{shelllstbgcolor}{gray}{.80}

\lstnewenvironment{shelllst}
{\lstset{language=bash,backgroundcolor=\color{shelllstbgcolor}}}
{}

\begin{document}


%---------------------------------------------------------------------&
% Titelseite

\begin{titlepage}

\begin{figure}[h]
\centering
\hfill
\begin{minipage}{0.2\textwidth}
\includegraphics[width=1.0\textwidth]{images/uni_logo.jpg}
\end{minipage}
\begin{minipage}{0.7\textwidth}
{\Huge\bf Universit\"at Stuttgart}\\[12pt]
{\Large\bf Fakult\"at Informatik, Elektrotechnik\\und Informationstechnik}
\end{minipage}
\end{figure}

\begin{figure}[h]
\centering
\begin{minipage}{0.2\textwidth}
\includegraphics[width=1.0\textwidth]{images/nexus_logo.jpg}
\end{minipage}
\end{figure}

\vspace{38pt}

\begin{center}
Fachstudie Nr. 85
\end{center}

\begin{center}
\Large\bf
Hardwareplattformen und\\
Systemsoftware\\
f\"ur drahtlose vermaschte\\
Kommunikationsnetze
\end{center}

\begin{center}
Alexander Egorenkov\\
Sergey Telezhnikov\\
Valeri Schneider
\end{center}

\begin{center}
\begin{tabular}{l@{\hspace{30pt}}l}
\bf Studiengang: & Softwaretechnik\\[5pt]
\bf Pr\"ufer:    & Prof. Dr. Kurt Rothermel\\[5pt]
\bf Betreuer:    & Dipl.-Inf. Lars Geiger\\[5pt]
\bf begonnen am: & November 2007\\[5pt]
\bf beendet am:  & Januar 2008\\[5pt]
\bf CR-Nummer:   & C.2.1, C.2.2, C.2.3, C.3\\[5pt]
\end{tabular}
\end{center}

\vfill

\begin{figure}[h]
\centering
\begin{minipage}{0.2\textwidth}
\includegraphics[width=1.0\textwidth]{images/ipvs_logo.jpg}
\end{minipage}
\begin{minipage}{0.4\textwidth}
\begin{center}
Institut f\"ur Parallele\\
und Verteilte Systeme\\
Abteilung Verteilte Systeme\\
Universit\"atsstra{\ss}e 38\\
D-70569 Stuttgart
\end{center}
\end{minipage}
\begin{minipage}{0.2\textwidth}
\includegraphics[width=1.0\textwidth]{images/vs_logo.jpg}
\end{minipage}
\end{figure}

\end{titlepage}



%\newpage
%\begin{abstract}

	\begin{center}
		\textbf{Abstract}
	\end{center}
	
	Mesh-Netze (engl. Wireless Mesh Network, WMN) 
	sind drahtlose Ad-Hoc-Netze bestehend aus station"aren 
	Mesh-Routern, die einen Routing-Backbone bilden, 
	und mobilen oder station"aren Mesh-Clients. 
	Die Mesh-Clients kommunizieren uber den Backbone 
	mit anderen Mesh-Clients oder erlangen "uber den 
	Backbone Zugang zum Internet. 
	Mesh-Netze k"onnen dabei auch gr"o"sere Bereiche, 
	beispielsweise ganze St"adte abdecken 
	(entsprechende Stadtnetze werden 
	aktuell z.B. durch Google installiert). 
	
	Ein entsprechendes Mesh-Netz muss f"ur die Forschungszwecke 
	f"ur den Sonderforschungsbereich (SFB) Nexus an der 
	Universit"at Stuttgart eingerichtet werden.
	
	Diese Fachstudie befasst sich mit der Ausarbeitung einer 
	Empfehlung f"ur die Beschaffung entsprechender Ger"ate 
	(\emph{Hardwareplattformen und Systemsoftware}) f"ur
	den Aufbau eines WMN.
	
%\end{abstract}

%---------------------------------------------------------------------&
\newpage
\tableofcontents

%---------------------------------------------------------------------&
%\newpage
\listoffigures

%---------------------------------------------------------------------&
\newpage
\section{Einleitung}

In diesem Abschnitt werden einige wichtige Begriffe, die im Laufe des
Dokuments auftauchen werden, kurz erl"autert.

\subsection{Grundlagen von Mesh-Netzen}

\subsubsection{Hintergrund}

Ein drahtloses vermaschtes Netz (engl. Wireless Mesh Network, WMN) besteht
aus einer Menge von Knoten, die "uber drahtlose Kommunikationstechniken
wie beispielsweise IEEE 802.11 (LINK) Nachrichten austauschen. Die
Vemaschung der Knoten erm"oglicht dabei nicht nur den Austausch von
Nachrichten zwischen unmittelbar benachbarten Knoten, sondern auch die
Vermittlung von Nachrichten an entfernte Knoten "uber mehrere Knoten
hinweg. Die Vermittlungsfunktionalit"at wird dabei oft von dedizierten
Vermittlungsknoten (engl. Mesh Router) bereitgestellt, die somit eine
drahtlose Kommunikationsinfrastruktur f"ur die Klienten (engl. Mesh
Client) bilden. Durch den Einsatz vergleichsweise kosteng"unstiger
Hardwarekomponenten und die Vermaschung der Knoten erm"oglichen WMNs die
kosteng"unstige Vernetzung auch gr"o"serer Gebiete. Entsprechende Netze
werden beispielsweise von Community-Projekten wie das Freifunk-Projekt (LINK)
oder Firmen wie Google (LINK) bereits heute in der Praxis f"ur den Aufbau gr""serer
Netze eingesetzt, um beispielsweise kosteng"unstige Internetzug"ange f"ur
Stadtteile oder ganze St"adte zu realisieren.

WMNs sind auch f"ur den Sonderforschungsbereich (SFB) Nexus an der
Universit"at Stuttgart
\url{http://www.nexus.uni-stuttgart.de} von gro"sem Interesse. Im
Zentrum der Forschungen des SFB stehen Umgebungsmodelle f"ur mobile
kontextbezogene Systeme. Umgebungsmodelle sind digitale Abbilder der
physischen Welt, die von kontextbezogenen Systemen genutzt werden, um
sich selbst"andig an die physische Umgebung des Benutzers anzupassen. Ein
einfaches Beispiel sind ortsbezogene Anwendungen, die beispielsweise
aufgrund der aktuellen geographischen Position eines Ger"ats automatisch
Informationen "uber nahe Restaurants, Sehensw"urdigkeiten, usw. selektieren
k"onnen. Zur Kommunikation, insbesondere mit mobilen Ger"aten, werden dabei
hybride Systeme betrachtet, in denen sowohl eine infrastrukturbasierte
Kommunikation als auch die direkte Ad-hoc-Kommunikation zwischen
mobilen Endsystemen m"oglich ist. Hierbei spielen WMNs als eine spezielle
Auspr"agung eines hybriden Kommunikationssystems eine wesentliche Rolle.


\subsubsection{Ad-Hoc}

Ein Ad-hoc-Netz bezeichnet in der Informationstechnologie eine drahtlose
Netzwerktopologie zwischen zwei oder mehr Endger"aten, die ohne feste
Infrastruktur auskommt.
(TODO) Mehr Tesxt.. Bilder: Adhoc, Infrastruktur!


\subsubsection{Mesh-Netz}
(TODO) Mesh-Netze sind Ad-hoc Netze

In einem vermaschten Netz (Mesh-Netz) ist jeder Netzwerkknoten mit einem
oder mehreren anderen verbunden. Die Informationen werden von Knoten
zu Knoten weitergereicht, bis sie das Ziel erreichen. Vermaschte
Netze sind im Regelfall selbstheilend und dadurch sehr zuverl"assig:
Wenn ein Knoten oder eine Verbindung blockiert ist oder ausf"allt, kann
sich das Netz darum herum neu stricken. Die Daten werden umgeleitet und
das Netzwerk ist nach wie vor betriebsf"ahig. 

(TODO) Einarbeitung in grundlegende WMN-Technologien???


\subsubsection{IEEE 802.11a/b/g}

\textbf{IEEE 802.11} (auch: Wireless LAN, WLAN, WiFi) bezeichnet eine IEEE-Norm
f"ur drahtlose Netzwerkkommunikation. Herausgeber ist das Institute of
Electrical and Electronics Engineers (IEEE).

\textbf{802.11a} spezifiziert eine weitere Variante der physikalischen Schicht,
die im 5-GHz-Band arbeitet und "Ubertragungsraten bis zu 54 MBit/s
ermoglicht. 

\textbf{Vorteile }
\begin{itemize}
	\item weniger genutztes Frequenzband, dadurch h"aufig
	st"orungsfreierer Betrieb m"oglich 
	\item in Deutschland 19 (bei BNetzA-Zulassung) nicht "uberlappende
	Kan"ale 
	\item h"ohere Reichweite, da mit 802.11h bis zu 1000 mW Sendeleistung
	m"oglich 
\end{itemize}

\textbf{Nachteile }
\begin{itemize}
	\item st"arkere Regulierungen in Europa: auf den meisten Kan"alen DFS
	n"otig 
	\item auf einigen Kan"alen kein Betrieb im Freien erlaubt 
	\item falls kein TPC benutzt wird, muss die Sendeleistung reduziert
	werden 
	\item Ad-hoc-Modus wird von den meisten Ger"aten
		nicht unterst"utzt
	\item geringere Verbreitung, daher wenig verf"ugbare Ger"ate
		auf dem Markt und hohe Kosten
\end{itemize}

\textbf{802.11b} ist ebenfalls eine alternative Spezifikation der physikalischen
Schicht, die mit dem bisher genutzten 2,4-GHz-Band auskommt und
"Ubertragungsraten bis zu 11 MBit/s erm"oglicht. 

\textbf{Vorteile}
\begin{itemize}	
	\item gebuhrenfreies freigegebenes ISM-Frequenzband 
	\item hohe Verbreitung und daher geringe Ger"atekosten 
\end{itemize}

\textbf{Nachteile}
\begin{itemize}	
	\item Frequenz muss mit anderen Ger"aten/Funktechniken geteilt werden
	(Bluetooth, Mikrowellenherde, etc.) 
	\item st"orungsfreier Betrieb von nur maximal 3 Netzwerken
	am selben Ort m"oglich, da effektiv nur 3 brauchbare
	(kaum "uberlappende) Kan"ale zur Verf"ugung stehen
	(in Deutschland: 1, 7, 13) 
\end{itemize}


\subsubsection{Linux MadWiFi-Treiber}

Linux MadWifi-Treiber ist Linux Kernel Treiber f"ur WLAN-Karten mit
Atheros-Chipsatz. Linux MadWifi-Treiber ist heutzutage einer der
fortgeschrittensten Linux Treiber f"ur WLAN-Karten. Der Treiber ist
stabil und hat eine gro"se Benutzergemeinschaft. Der MadWifi-Treiber
selbst ist Open-Source, verwendet aber eine proprit"are Softwareschicht
Hardware Abstraction Layer (HAL), die nur in bin"arer Form vorhanden
ist. Das Hardware Abstraction Layer (HAL) wird vom MadWifi-Treiber
gebraucht, um die Atheros-Chips ansprechen zu k"onnen. Daf"ur wurde bisher
ein Closed-Source-Modul verwendet. Dies hat unter anderem damit zu tun,
dass die Atheros-Chips"atze prinzipiell auf Frequenzen funken k"onnten,
f"ur die sie nicht zugelassen sind - beispielsweise weil diese vom
Milit"ar zur Kommunikation verwendet werden. Durch das proprit"are
Modul war der MadWifi-Treiber bisher jedoch von einer Aufnahme in den
Linux-Kernel ausgeschlossen. Die Entwickler hatten au"serdem das Problem,
dass sie Fehler unter Umst"anden nicht beheben konnten, da sie nicht
nachvollziehen konnten, wie der HAL-Baustein arbeitet. MadWifi
selbst wird daher ab sofort nicht weiterentwickelt. Stattdessen
setzen die Programmierer auf OpenHAL, eine Linux-Portierung des
HAL-Modules des in OpenBSD verf"ugbaren freien Atheros-Treibers. In der
Vergangenheit wurde vom Software Freedom Law Center (SFLC) best"atigt,
dass die durch Reverse Engineering entstandene Software keine Copyrights
verletzt. Solche Behauptungen hatten die Entwicklung lange ausgebremst. 
Der neue Treiber "`Ath5k"' wird MadWifi nun ersetzen und soll nicht
nur die freie Komponente OpenHAL einsetzen, sondern auch mit dem neuen
Linux-WLAN-System Mac80211 zusammenarbeiten, so dass der Treiber in den
offiziellen Linux-Kernel gelangen kann. MadWifi soll jedoch weiter mit
Fehlerkorrekturen und HAL-Updates versorgt werden. 

\subsubsection{Ad-Hoc Routing-Protokolle}
(TODO) Text .. was ist was.. 
Weiter werden OLSR ung .. vorgestellt..

\paragraph{OLSR}

Optimized Link State Routing, kurz OLSR, ist ein Routingprotokoll
f"ur mobile Ad-hoc-Netze, das eine an die Anforderungen eines mobilen
drahtlosen LANs angepasste Version des Link State Routing darstellt. Es
wurde von der IETF mit dem RFC 3626 standardisiert. Bei diesem
verteilten flexiblen Routingverfahren ist allen Routern die vollst"andige
Netztopologie bekannt, sodass sie von Fall zu Fall den k"urzesten Weg zum
Ziel festlegen k"onnen. Als proaktives Routingprotokoll h"alt es die daf"ur
ben"otigten Informationen jederzeit bereit. Ein in Mesh-Netzwerken
bekannter Vertreter von LSR ist OLSR von olsr.org. Inzwischen existieren
f"ur OLSR spezielle Erweiterungen. Mit der ETX-Erweiterung wird dem
Umstand Rechnung getragen, dass Links asymmetrisch sein k"onnen. Mit
dem Fisheye-Algorithmus ist OLSR auch fur gro"sere Netzwerke brauchbar
geworden, da Routen zu weiter entfernten Knoten weniger h"aufig neu
berechnet werden. Der entscheidende Nachteil ist aber der trotz
Fisheye-Algorithmus noch recht hohe Rechenaufwand von OLSRD, sobald
die Anzahl an Knoten ein gewisses Ma"s ubersteigt.

\paragraph{B.A.T.M.A.N.}

Ausgehend von den Erfahrungen mit Freifunk-OLSR begannen die Entwickler
aus der Freifunk-Community im M"arz 2006 in Berlin damit, ein neues
Routingprotokoll f"ur drahtlose Meshnetzwerke zu entwickeln. Alle bisher
bekannten Routingalgorithmen versuchen Routen entweder zu berechnen
(proaktive Verfahren) oder sie dann zu suchen, wenn sie gebraucht werden
(reaktive Verfahren). Das neue Protokoll B.A.T.M.A.N. 
(BETTER APPROACH TO MOBILE ADHOC NETWORKING) berechnet oder
sucht im Gegensatz zu diesen Protokollen keine Routen, es erfasst
lediglich, ob Routen zu anderen Knoten existieren und "uberwacht ihre
Qualit"at. Dabei interessiert es sich nicht daf"ur, wie eine Route verl"auft,
sondern ermittelt lediglich, "uber welchen direkten Nachbarn ein bestimmter
Netzwerkknoten am besten zu erreichen ist, und tr"agt diese Information
proaktiv in die Routingtabelle ein. 

\subsubsection{Firmware f"ur WLAN-Router}

(TODO) Text .. was ist was..

\paragraph{OpenWRT}

OpenWRT ist eine GNU/Linux-Distribution f"ur WLAN-Router. Anstatt einer
statischen Firmware setzt OpenWRT auf ein voll beschreibbares Dateisystem
sowie einen Paketmanager. OpenWRT l"auft unter anderem auf Ger"aten der
Firmen Linksys, ALLNET, ASUS, Belkin, Buffalo, Microsoft und Siemens.

\textbf{Vorteile:}

\begin{itemize}
	\item Flexibilit"at
	\item Erweiterbarkeit
	\item Individualisierbarkeit
	\item Sicherheit
	\item Gewohnte Linux-Flexibilit"at und Funktionsumfang! 
\end{itemize}

\textbf{Nachteile:}

\begin{itemize}
	\item Standardm"a"sig sind nur die n"otigsten Unix-Tools vorhanden 
\end{itemize}

\textbf{Links:}

\begin{itemize}
	\item \url{http://openwrt.org/}
	\item \url{http://toh.openwrt.org/}
\end{itemize}

\subsection{Existierende L"osungen und Projekte}

(TODO) Einbisschen Text zu jedem Projekt (googeln)

\begin{itemize}
\item FreiFunk \url{http://freifunk.net/wiki/Meshing}
\item OpenNet \url{http://wiki.opennet-initiative.de/index.php/Hauptseite}
\item \url{http://www-i4.informatik.rwth-aachen.de/mcg/projects/umic-mesh/} 
\item \url{http://umic-mesh.net/}
\item (TODO) Google..

\end{itemize}

\section{Aufgabenstellung}

F�r Forschungszwecke soll innerhalb des SFB Nexus (URL) 
ein WMN installiert werden. 

Dieses WMN dient 
\begin{itemize}

	\item einerseits Nexus-Anwendungen, insbesondere Anwendungen 
auf mobilen Ger�ten, als \emph{Kommunikationsmedium}. 

\item Andererseits soll dieses WNIN auch als \emph{Testbed} 
zur Erforschung verschiedene Erweiterungen von WMNs dienen, 

\end{itemize}

beispielsweise der Untersuchung neuartige kontextbezogener 
Kommunikationsmechanismen, der Erforschung von 
Publish/Subscribe-Diensten f�r WMNs oder der 
Verwaltung von Umgebungsmodellen innerhalb eines 
hybriden Systems wie es ein WMN darstellt. 

Ziel dieser Fachstudie ist die Ausarbeitung einer 
Empfehlung f�r die Beschaffung entsprechender Ger�te 
(\emph{Hardwareplattformen und Systemsoftware}) f�r
den Aufbau eines WMN.) \\

Das Vorgehen umfasst im einzelnen:

\begin{itemize}
	
	\item Einarbeitung in grundlegende WMN-Technologien
	\item Analyse der Anforderungen des Nexus-Projektes an ein WNN
	\item Erstellung einer �bersicht �ber aktuelle verf�gbare 
	Hardwareplattformen und Systemsoftware f�r WMN
	\item Bewertung der analysierten Systeme hinsichtlich 
	der ermittelten Anforderungen 	
	\item Ausarbeitung einer Empfehlung f�r eine geeignetes 
	WNN hinsichtlich Hardwareplattform und Systemsoftware
	
\end{itemize}


\section{Anforderungen}

Nach der Einarbeitung in WMN-Technologien und 
Analyse der Anforderungen des Nexus-Projektes
wurden folgendes festgehalten:

\begin{itemize}
	
	\item IEEE 802.11a kompatibel (5Ghz-Frequenzen) 
	
	ob es 802.11a Karten gibt, die im Ad-hoc-Modus arbeiten? 
	ob es neben einzelnen Karten auch komplette stand-alone 
	Mesh-Produkte gibt, die 802.11a kompatibel sind?

	\item Ad-hoc Modus (erkl�hrung)
	
	\item Treiber fu Linux (und Windows) 

	\item Open-Source Firmware fur Router 

	\item Abdeckung des Gebaudes Universitatsstra�e 38 

	\item Zusatzlich zu Wireless Mesh Network auch weitere (Netzwerk-)Schnittstelle zur Verwaltung vorsehen 

	\item OS nicht festgelegt, soll Ergebnis der Fachstudie sein 

	\item Betriebsystem vorschlagen.. 

	\item Freiheit bei Routingprotokollen 

	\item Routing Protokolle auswechelbar.. (daemon start, exit..) 

	\item Konfigurierung und Instrumentierung 

	\item Topologie verandern bzw. erfassen 

	\item Abfragen Visualisieren 

	\item Nach Moglichkeit keine selber gebastelten Losungen 

	\item Schon ware, die angestrebte Standardisierung von Mesh-Netzen zu unterstutzen 

	\item MESH STANDART 11n Draft als Vorteil 

	\item Verbindung mit Informatik-Netz nur uber Gateway mit strikter Filterung 

	\item nur eine Richtung (UNItoMesh) fur die Verwaltung ) 

	\item AUFWANDSCHATZUNG (wie viele Knoten usw. ) 

	\item Budget max. 25.000 Euro (evtl. mehr in Zukunft) 

	\item 4-5 Mesh-Knoten pro Quadrat 

	\item Separates Gateway notwendig? Vermutlich sinnvoll. 

	\item Bei Router - Speicherkapazitat wichtig (falls uberhaupt in Frage kommt) 

	\item FOCUS -> PC + Wlan-Karten + 

	\item MIMO System (PC + 2 Wlan-Karten) Testen.. 

	\item Funk auf verschiedenen Frequenzbandern (Performanceverbesserung) 

	\item PDAs (bzw. andere kleine Clients) mit 802.11a?
	
\end{itemize}



\section{Hardware-L�sungen f�r den Aufbau eines Mesh-Netzwerkes}

Es gibt verschiedene Moglichkeiten ein Meshnetzwerk aufzubauen. Im Weiteren werden einige davon im Detail beschrieben. 

\subsection{PCs + WLAN-Karten}

Die einfachste M"oglichkeit w"are die herk"ommlichen PCs mit
WLAN-Karten zu einem Mesh-Router einzurichten.  Man nimmt dabei
einfach die WLAN-Karten (PCI, MiniPCI(e) oder PCMCIA) und baut diese
in PCs oder in Laptops ein. Man installiert dann auf diesen Rechnern
entsprechende Treiber, die WLAN-Karten im Ad-Hoc Modus betreiben k"onnen
und Routing-Software, z.B. OLSR-Protokoll.

Es besteht aber das Problem, dass heutige WLAN-Karten den Ad-Hoc Modus im 5 GHz
Frequenzband gar nicht oder sehr schlecht unterst"utzen. Das liegt daran,
dass der Ad-Hoc Modus viel komplizierter als Infrastruktur Modus ist und
in den meisten F"allen werden WLAN-Karten sowieso nur im Infrastruktur Modus
eingesetzt.

Es gibt aber WLAN-Karten, bei denen der Ad-Hoc Modus im 5 GHz Frequenzband
sehr gut unterst"utzt wird. Das sind WLAN-Karten von Intel und WLAN-Karten,
die auf Atheros Chips"atzen basieren. Diese WLAN-Karten werden 
in folgenden Abschnitten detaillierter beschrieben und analysiert.

Die Hardware f"ur Mesh-Router auf PC-Basis kann man in 3 Gruppen unterteilen:

\begin{itemize}
\item PCI WLAN-Karten
\item MiniPCI(e) WLAN-Karten
\item PCMCIA WLAN-Karten
\end{itemize}

F"ur den Einsatz in Mesh-Routern kommen eigentlich nur PCI und MiniPCI(e)
WLAN-Karten in Frage. PCMCIA WLAN-Karten sind eher f"ur Mesh-Clients geeignet,
obwohl sie auch in Mesh-Routern ohne Probleme eingesetzt werden k"onnen.
Die meisten PCs haben n"ahmlich keinen PCMCIA Bus.

Im folgenden werden Vorteile und Nachteile von Mesh-Routern auf PC-Basis
erw"ahnt.

\textbf{Vorteile:}

\begin{itemize} 
\item PCs sind nicht teuer, flexibel und leicht erweiterbar
\item Hardware kann sp"ater f"ur andere Zwecke eingesetzt werden
\item Installation und Konfiguration von Hardware und Software ist einfacher
im Vergleich zu SoHO-Routern
\item Sehr gro"se Menge an verschiedener Software vorhanden
\item Mehrere WLAN- und Ethernet-Schnittstellen m"oglich 
\end{itemize}

\textbf{Nachteile: }

\begin{itemize}
\item PCs sind gro"s und station"ar
\item Brauchen mehr Strom im Vergleich zu SoHO-Routern
\item Ohne externe Antennen schlechte Sende- und Empfangqualit"at,
da sich die Antennen im elektromagnetischen St"orfeld des PCs befindet
\end{itemize}

\subsubsection{PCI-WLAN-Karten}

%%%%%%%%%%%%%%%%%%%%%%%%%%%%%%%%%%%%%%%%%%%%%%%%%%%%%%%%%%%%%%%%%%%%%%%%%%%%
%
% Linksys WMP55AG
%
%%%%%%%%%%%%%%%%%%%%%%%%%%%%%%%%%%%%%%%%%%%%%%%%%%%%%%%%%%%%%%%%%%%%%%%%%%%%
\begin{wlandevice}{Linksys WMP55AG}

\wlanimage{Linksys_WMP55AG}{Linksys WMP55AG}

\wlanchipset{Atheros AR5213A}

\begin{wlanieeestandard}
\item 802.11abg
\end{wlanieeestandard}

\begin{wlanmode}
\item Ad-Hoc-Modus
\item Infrastruktur-Modus
\end{wlanmode}

\begin{wlansecurity}
\item WEP (40-, 104-, 128-bit)
\item WPA
\item LEAP
\end{wlansecurity}

\begin{wlandriver}
\item
Sehr gute Linux-Unterstutzung, madwifi-Treiber funktioniert
mit dieser WLAN PCI-Karte ohne Probleme.
Windows-Treiber werden von Linksys bereitgestellt.
\end{wlandriver}

\wlanprice{90}

\begin{wlaninstall}
\item
Lasst sich leicht sowohl unter Windows als auch unter Linux (madwifi-Treiber) installieren.
http://madwifi.org/wiki/UserDocs/FirstTimeHowTo
\end{wlaninstall}

\end{wlandevice}

%%%%%%%%%%%%%%%%%%%%%%%%%%%%%%%%%%%%%%%%%%%%%%%%%%%%%%%%%%%%%%%%%%%%%%%%%%%%
%
% Netgear WAG311
%
%%%%%%%%%%%%%%%%%%%%%%%%%%%%%%%%%%%%%%%%%%%%%%%%%%%%%%%%%%%%%%%%%%%%%%%%%%%%
\begin{wlandevice}{Netgear WAG311}

\wlanimage{Netgear_WAG311}{Netgear WAG311}

\wlanchipset{}

\begin{wlanieeestandard}
\item
\end{wlanieeestandard}

\begin{wlanmode}
\item
\end{wlanmode}

\begin{wlansecurity}
\item
\end{wlansecurity}

\begin{wlandriver}
\item
\end{wlandriver}

\wlanprice{}

\begin{wlaninstall}
\item
\end{wlaninstall}

\end{wlandevice}

%%%%%%%%%%%%%%%%%%%%%%%%%%%%%%%%%%%%%%%%%%%%%%%%%%%%%%%%%%%%%%%%%%%%%%%%%%%%
%
% D-Link DWL-A520
%
%%%%%%%%%%%%%%%%%%%%%%%%%%%%%%%%%%%%%%%%%%%%%%%%%%%%%%%%%%%%%%%%%%%%%%%%%%%%
\begin{wlandevice}{D-Link DWL-A520}

\wlanimage{DLink_DWLA520}{D-Link DWL-A520}

\wlanchipset{}

\begin{wlanieeestandard}
\item
\end{wlanieeestandard}

\begin{wlanmode}
\item
\end{wlanmode}

\begin{wlansecurity}
\item
\end{wlansecurity}

\begin{wlandriver}
\item
\end{wlandriver}

\wlanprice{}

\begin{wlaninstall}
\item
\end{wlaninstall}

\end{wlandevice}

%%%%%%%%%%%%%%%%%%%%%%%%%%%%%%%%%%%%%%%%%%%%%%%%%%%%%%%%%%%%%%%%%%%%%%%%%%%%
%
% Gigabyte GN-WPEAG
%
%%%%%%%%%%%%%%%%%%%%%%%%%%%%%%%%%%%%%%%%%%%%%%%%%%%%%%%%%%%%%%%%%%%%%%%%%%%%
\begin{wlandevice}{Gigabyte GN-WPEAG}

\wlanimage{Gigabyte_GNWPEAG}{Gigabyte GN-WPEAG}

\wlanchipset{}

\begin{wlanieeestandard}
\item
\end{wlanieeestandard}

\begin{wlanmode}
\item
\end{wlanmode}

\begin{wlansecurity}
\item
\end{wlansecurity}

\begin{wlandriver}
\item
\end{wlandriver}

\wlanprice{}

\begin{wlaninstall}
\item
\end{wlaninstall}

\end{wlandevice}

\subsubsection{MiniPCI(e) WLAN-Karten}

MiniPCI(e) ist eine vor allem f"ur die Nutzung in Notebooks und Laptops
miniaturisierte Version des PCI Steckplatzes, wie er in allen Desktop
PCs vorkommt. Die Abmessungen einer MiniPCI Karte betragen 6,0 x 4,6 x 0,5 cm.
Die Abmessungen einer MiniPCIe Karte betragen 3 cm x 5 cm x 0.4 cm.

MiniPCI(e) WLAN-Karten sind urspr"unglich f"ur Laptops gedacht, k"onnen aber
mit entschprechenden Adaptern (MiniPCI(e)-to-PCI) und externen Antennen auch
in normalen PCs verwendet werden.

Im folgenden werden Vorteile und Nachteile von MiniPCI(e) WLAN-Karten
erl"autert.

\textbf{Vorteile:}

\begin{itemize}
\item K"onnen mit Hilfe eines Adapters zu einer PCI WLAN-Karte umgebaut werden
\item Leicht austauschbar
\item Sehr gute Treiber-Unterst"utzung unter Linux und Windows
\end{itemize}

\textbf{Nachteile:}

\begin{itemize}
\item Brauchen einen PCI-Adapter f"ur den PCI-Bus
\item Haben keine Antenne (extra Kosten)
\end{itemize}

Wir haben nur eine MiniPCI und drei MiniPCIe WLAN-Karten gefunden,
die den Ad-Hoc Modus im 5 GHz Frequenzband unterst"utzen. Diese WLAN-Karten
basieren entweder auf Intel Chips"atzen oder Atheros Chips"atzen.

%%%%%%%%%%%%%%%%%%%%%%%%%%%%%%%%%%%%%%%%%%%%%%%%%%%%%%%%%%%%%%%%%%%%%%%%%%%%
%
% Wistron CM9 Atheros AR5213A
%
%%%%%%%%%%%%%%%%%%%%%%%%%%%%%%%%%%%%%%%%%%%%%%%%%%%%%%%%%%%%%%%%%%%%%%%%%%%%
\begin{wlandevice}{Wistron CM9 Atheros AR5213A}

\wlanimage{Wistron_CM9}{Wistron CM9 Atheros AR5213A}

\wlanchipset{Atheros AR5213A}

\begin{wlanieeestandard}
\item 802.11a/b/g
\end{wlanieeestandard}

\begin{wlanmode}
\item Ad-Hoc
\item Infrastruktur
\end{wlanmode}

\begin{wlansecurity}
\item WEP (40-, 104-, 128-bit)
\item WPA
\item WPA2
\end{wlansecurity}

\begin{wlandriver}
\item
Herrvorragende Unterst"utzung von MadWifi-Treiber \cite{madwifi},
auch Ad-Hoc-Modus.
\end{wlandriver}

\wlanprice{40}

\begin{wlaninstall}
\item
\url{http://madwifi.org/wiki/UserDocs/FirstTimeHowTo}
\end{wlaninstall}

\begin{wlanlink}
\item \url{http://www.alix-board.de/produkte/wistroncm9.html}
\item \url{http://www.pcengines.ch/cm9.htm}
\item \url{http://forum.openwrt.org/viewtopic.php?pid=10213}
\item \url{http://madwifi.org/ticket/1209}
\end{wlanlink}

\end{wlandevice}

%%%%%%%%%%%%%%%%%%%%%%%%%%%%%%%%%%%%%%%%%%%%%%%%%%%%%%%%%%%%%%%%%%%%%%%%%%%%
%
% Intel PRO/Wireless 3945
%
%%%%%%%%%%%%%%%%%%%%%%%%%%%%%%%%%%%%%%%%%%%%%%%%%%%%%%%%%%%%%%%%%%%%%%%%%%%%
\begin{wlandevice}{Intel PRO/Wireless 3945}

\wlanimage{Intel_3945ABG}{Intel PRO/Wireless 3945}

\wlanchipset{Intel}

\begin{wlanieeestandard}
\item 802.11a/b/g
\end{wlanieeestandard}

\begin{wlanmode}
\item Ad-Hoc
\item Infrastruktur
\end{wlanmode}

\begin{wlansecurity}
\item WEP (40-, 104-bit)
\item WPA
\item WPA2
\end{wlansecurity}

\begin{wlandriver}
\item
Es werden von Intel Treiber sowohl f"ur Windows als auch f"ur Linux
bereitgestellt.

\url{http://downloadcenter.intel.com/Product_Filter.aspx?ProductID=2259}

Von Intel wurde ein Projket f"ur die Unterst�tzung von Intel PRO/Wireless
3945 erstellt.

\url{http://ipw3945.sourceforge.net}

Der ipw3945-Treiber funktioniert auch im Ad-Hoc-Modus, aber nicht sehr stabil,
es kommt oft zu Verbindungsabbr"uchen.
\end{wlandriver}

\wlanprice{20-30}

\begin{wlaninstall}
\item
Im Gegensatz zu den "`klassischen"' Intel Wireless-Chips"atzen 2100- und
2200BG-Chips"atzen ist der Treiber f"ur den 3945ABG noch nicht im Kernel
verf"ugbar. Um auch damit kabellos ins Internet zu gehen,
sind ein paar Handgriffe notwendig.

\url{http://ipw3945.sourceforge.net/README.ipw3945}

\url{http://ipw3945.sourceforge.net/INSTALL}
\end{wlaninstall}

\begin{wlanlink}
\item \url{http://www.intel.com/network/connectivity/products/wireless/prowireless_mobile.htm}
\item \url{http://downloadcenter.intel.com/Product_Filter.aspx?ProductID=2259}
\item \url{http://ipw3945.sourceforge.net/}
\item \url{http://ipw3945.sourceforge.net/README.ipw3945}
\item \url{http://ipw3945.sourceforge.net/INSTALL}
\end{wlanlink}

\end{wlandevice}

%%%%%%%%%%%%%%%%%%%%%%%%%%%%%%%%%%%%%%%%%%%%%%%%%%%%%%%%%%%%%%%%%%%%%%%%%%%%
%
% Intel PRO/Wireless 2915
%
%%%%%%%%%%%%%%%%%%%%%%%%%%%%%%%%%%%%%%%%%%%%%%%%%%%%%%%%%%%%%%%%%%%%%%%%%%%%
\begin{wlandevice}{Intel PRO/Wireless 2915}

\wlanimage{Intel_2915ABG}{Intel PRO/Wireless 2915}

\wlanchipset{Intel}

\begin{wlanieeestandard}
\item 802.11a/b/g
\end{wlanieeestandard}

\begin{wlanmode}
\item Ad-Hoc
\item Infrastruktur
\end{wlanmode}

\begin{wlansecurity}
\item WEP (40-, 104-bit)
\item WPA
\item WPA2
\end{wlansecurity}

\begin{wlandriver}
\item
Es werden von Intel Treiber sowohl f"ur Windows als auch f"ur Linux
bereitgestellt.

\url{http://downloadcenter.intel.com/Product_Filter.aspx?ProductID=1847}

Von Intel wurde ein Projket f"ur die Unterst"utzung von Intel PRO/Wireless
2915 erstellt.

\url{http://ipw2200.sourceforge.net}

Der ipw2200-Treiber funktioniert auch im Ad-Hoc-Modus, aber nicht
sehr stabil, es kommt oft zu verbindungsabbr�chen. Der ipw2200-Treiber
ist im Kernel 2.6 enthalten, kann aber auch separat als Modul kompiliert
werden. Der im Kernel enthaltene Treiber unterst"utzt den Monitor-Modus
nicht.
\end{wlandriver}

\wlanprice{30}

\begin{wlaninstall}
\item
\url{http://ipw2200.sourceforge.net/README.ipw2200}

\url{http://ipw2200.sourceforge.net/INSTALL}
\end{wlaninstall}

\begin{wlanlink}
\item \url{http://support.intel.com/support/wireless/wlan/pro2915abg}
\item \url{http://download.intel.com/support/wireless/wlan/pro2915abg/sb/303330002us_channel.pdf}
\item \url{http://ipw2200.sourceforge.net/}
\item \url{http://www.intel.com/cd/personal/computing/emea/deu/234998.htm}
\item \url{http://downloadcenter.intel.com/Product_Filter.aspx?ProductID=1847}
\end{wlanlink}

\end{wlandevice}

%%%%%%%%%%%%%%%%%%%%%%%%%%%%%%%%%%%%%%%%%%%%%%%%%%%%%%%%%%%%%%%%%%%%%%%%%%%%
%
% Intel Wireless WiFi Link 4965AGN
%
%%%%%%%%%%%%%%%%%%%%%%%%%%%%%%%%%%%%%%%%%%%%%%%%%%%%%%%%%%%%%%%%%%%%%%%%%%%%
\begin{wlandevice}{Intel Wireless WiFi Link 4965AGN}

\wlanimage{Intel_4965AGN}{Intel Wireless WiFi Link 4965AGN}

\wlanchipset{Intel}

\begin{wlanieeestandard}
\item 802.11a/b/g/n(draft)
\end{wlanieeestandard}

\begin{wlanmode}
\item Ad-Hoc
\item Infrastruktur
\end{wlanmode}

\begin{wlansecurity}
\item WEP (40-, 104-bit)
\item WPA
\item WPA2
\end{wlansecurity}

\begin{wlandriver}
\item
\url{http://www.intellinuxwireless.org/}
\end{wlandriver}

\wlanprice{30}

\begin{wlaninstall}
\item
\url{http://www.intellinuxwireless.org/}
\end{wlaninstall}

\begin{wlanlink}
\item \url{http://www.intel.com/network/connectivity/products/wireless/wireless_n/overview.htm}
\item \url{http://www.intellinuxwireless.org/}
\item \url{http://www.wifi-info.de/intel-kuendigt-11n-chipsatz-fuer-centrino-notebooks-an/01/2007/}
\item \url{http://downloadcenter.intel.com/filter_results.aspx?strTypes=all&ProductID=2753&OSFullName=Linux*&lang=eng&strOSs=39&submit=Go\%21}
\end{wlanlink}

\end{wlandevice}

\input{PCMCIA_WLAN-Karten}

\newpage

\subsection{WLAN-Router}

Die Kombination aus Access Point und Router wird h"aufig als WLAN-Router
bezeichnet. Das ist solange korrekt, soweit es einen WAN-Port gibt. Das
Routing findet dann zwischen WLAN, LAN und WAN statt. Fehlt dieser
WAN-Port, handelt es sich hier lediglich um Marketing-Begriffe, da reine
Access Points auf OSI-Ebene 2 arbeiten und somit Bridges und keine
Router sind. Oft sind das aber keine vollst"andigen Router, da diese
Ger"ate ausschlie"slich als Internetzugangs-Systeme dienen und nur mit
aktiviertem PPPoE (oder PPPoA) sowie NAT-Routing (oder IP-Masquerading)
eingesetzt werden k"onnen.

In diesem Abschnitt werden wir so genannte stand-alone Router
betrachten, die als mesh-Router in einem WMN in Frage kommen.
Alle solche WLAN-Router haben wir in 2 Gruppen unterteilt:

\begin{itemize}
\item SoHO-Router
\item Professionelle Router
\end{itemize}

In folgenden Abschnitten werde wir beide Gruppen von WLAN-Routern genauer
betrachten.

\subsubsection{SoHo-Router}

Man kann herk"ommliche WLAN-Router f"ur Heimanwender (SoHO-Router - small
or home office) kaufen, die sich mit alternativer Firmware (spezielle
Linux-Software mit OLSR-daemon) zu einem Mesh-Router umr"usten
lassen. Ein WLAN-Router ist die Kombination eines normalen Router
(Kabelrouter) mit einem Accesspoint. Es gibt solche mit eingebauten
Modem und andere mit einem Anschluss (WAN-Port) daf"ur (f"ur Modems mit
LAN-Anschluss). Ein Nachteil ist, dass es viele Modelle gibt, die eine
fix verbaute Antenne haben, die nicht gewechselt werden kann.

\textbf{Vorteile:}

\begin{itemize}
	\item Klein und	und handlich
	\item Mobil und flexibel
	\item Sehr g"unstig
	\item Gute Reichweite
	\item Wenig Stromverbrauch
	\item Leichte Konfiguration und Installation
\end{itemize}

\textbf{Nachteile:}

\begin{itemize}
	\item Eingeschr"ankte Software-Unterst"utzung
	\item Open-Source Firmware schwer zu finden
	\item Durch das "Offnen von Ger"aten und das Einspielen von
	fremder Firmware erlischt die Garantie des Herstellers
	\item Eingeschr"ankter Funktionsumfang
\end{itemize}

%%%%%%%%%%%%%%%%%%%%%%%%%%%%%%%%%%%%%%%%%%%%%%%%%%%%%%%%%%%%%%%%%%%%%%%%%%%%
%
% Linksys WRT54G v1.0
%
%%%%%%%%%%%%%%%%%%%%%%%%%%%%%%%%%%%%%%%%%%%%%%%%%%%%%%%%%%%%%%%%%%%%%%%%%%%%
\begin{wlandevice}{Linksys WRT54G v1.0}

\wlanimage{Linksys_WRT54G}{Linksys WRT54G v1.0}

\begin{wlanieeestandard}
\item 802.11b/g
\item 802.11a/b/g (wenn man die mitgelieferte Mini-PCI WLAN-Karte
durch z.B. Atheros 802.11a/b/g WLAN-Karte austauscht)
\end{wlanieeestandard}

\begin{wlanmode}
\item Ad-Hoc
\item Infrastruktur
\end{wlanmode}

TODO: \todo{Erlaeterung warum diese benoetigt wird, was ist damit moeglich usw. Beschreibung oder/und Verweis auf OpenWRT abschnitt..}

\begin{wlanfirmware}
\item
Es gibt mehrere fremde frei verf"ugbare Firmware f"ur dieses Ger"at.
Alle unten aufgef"uhrten Firmware sind Open-Source Projekte:

OpenWRT

\url{http://wiki.openwrt.org/OpenWrtDocs/Hardware/Linksys/WRT54G}

DD-WRT

\url{http://www.dd-wrt.com/wiki/index.php/Linksys_WRT54G/GL/GS/GX}
\end{wlanfirmware}

\wlanprice{40-50}

\begin{wlaninstall}
\item
Die mitgelieferte Mini-PCI WLAN-Karte durch z.B. Atheros 802.11a Mini-PCI
austauschen und oben erw"ahnte frei verf"ugbare Firmware installieren
(siehe oben Firmware).
\end{wlaninstall}

\begin{wlanextrainfo}
\item
Ein Mini-PCI Slot ist f"ur eine WLAN-Karte vorhanden.
\end{wlanextrainfo}

\begin{wlanlink}
\item \url{http://wiki.openwrt.org/OpenWrtDocs/Hardware/Linksys/WRT54G}
\item \url{http://www.dd-wrt.com/wiki/index.php/Linksys_WRT54G/GL/GS/GX}
\item \url{http://forum.opennet-initiative.de/thread.php?threadid=505&sid=56c53647db6353a41e9a3100f00d02c4}
\item \url{http://www.linksysinfo.org/forums/showthread.php?t=47124}
\end{wlanlink}

\end{wlandevice}

%%%%%%%%%%%%%%%%%%%%%%%%%%%%%%%%%%%%%%%%%%%%%%%%%%%%%%%%%%%%%%%%%%%%%%%%%%%%
%
% Linksys WRT55AG
%
%%%%%%%%%%%%%%%%%%%%%%%%%%%%%%%%%%%%%%%%%%%%%%%%%%%%%%%%%%%%%%%%%%%%%%%%%%%%
\begin{wlandevice}{Linksys WRT55AG}

\wlanimage{Linksys_WRT55AG}{Linksys WRT55AG}

\begin{wlanieeestandard}
\item 802.11a/b/g
\end{wlanieeestandard}

\begin{wlanmode}
\item Ad-Hoc
\item Infrastruktur
\end{wlanmode}

\begin{wlanfirmware}
\item
Open-Source Firmware befindet sich noch in Entwicklung:

Modifizierte Version von OpenWRT Kamikaze

\url{http://legacy.not404.com/cgi-bin/trac.fcgi/wiki/OpenWRT/Atheros/Linksys/WRT55AGv2#KamikazeKernelonWRT55AGv2}

OpenWRT

\url{http://wiki.openwrt.org/OpenWrtDocs/Hardware/Linksys/WRT55AG}
\end{wlanfirmware}

\wlanprice{70-80}

\begin{wlanextrainfo}
\item
2 Slots sind f"ur Mini-PCI WLAN-Karten vorhanden.
\end{wlanextrainfo}

\begin{wlanlink}
\item \url{http://wiki.openwrt.org/OpenWrtDocs/Hardware/Linksys/WRT55AG}
\item \url{http://www.tomsnetworking.de/content/tests/j2003a/test_linksys_wrt55ag/index.html}
\item \url{http://reviews.cnet.com/routers/linksys-wrt55ag-wireless-a/4505-3319_7-21131921.html}
\item \url{http://legacy.not404.com/cgi-bin/trac.fcgi/wiki/OpenWRT/Atheros/Linksys/WRT55AGv2}
\end{wlanlink}

\end{wlandevice}

%%%%%%%%%%%%%%%%%%%%%%%%%%%%%%%%%%%%%%%%%%%%%%%%%%%%%%%%%%%%%%%%%%%%%%%%%%%%
%
% Asus WL500G/GP
%
%%%%%%%%%%%%%%%%%%%%%%%%%%%%%%%%%%%%%%%%%%%%%%%%%%%%%%%%%%%%%%%%%%%%%%%%%%%%
\begin{wlandevice}{Asus WL500G/GP}

\wlanimage{Asus_WL500G}{Asus WL500G/GP}

\begin{wlanieeestandard}
\item 802.11b/g
\item 802.11a/b/g (wenn man die mitgelieferte Mini-PCI WLAN-Karte
durch z.B. Atheros 802.11a/b/g WLAN-Karte austauscht)
\end{wlanieeestandard}

\begin{wlanmode}
\item Ad-Hoc
\item Infrastruktur
\end{wlanmode}

\begin{wlanfirmware}
\item
Es sind mehrere fremde frei verf"ugbare Firmware f"ur dieses Ger"at.
Alle unten aufgef"uhrten Firmware sind Open-Source Projekte:

OpenWRT

\url{http://wiki.openwrt.org/OpenWrtDocs/Hardware/Asus/WL500G}

\url{http://wiki.openwrt.org/OpenWrtDocs/Hardware/Asus/WL500GP}

FreeWRT

\url{http://www.freewrt.org/trac/wiki/Documentation/Hardware/AsusWL500G}

\url{http://www.freewrt.org/trac/wiki/Documentation/Hardware/AsusWL500GP}

Olegs custom firmware

\url{http://oleg.wl500g.info}
\end{wlanfirmware}

\wlanprice{70-80}

\begin{wlaninstall}
\item
Die mitgelieferte Mini-PCI WLAN-Karte durch z.B. Atheros 802.11a Mini-PCI
austauschen und oben erw"ahnte frei verf"ugbare Firmware installieren
(siehe oben Firmware).

\url{http://wiki.opennet-initiative.de/index.php/Mini-PCI_Umbau}
\end{wlaninstall}

\begin{wlanextrainfo}
\item
Ein Mini-PCI Slot ist f"ur eine WLAN-Karte vorhanden.
\end{wlanextrainfo}

\begin{wlanlink}
\item \url{http://wiki.opennet-initiative.de/index.php/AP9}
\item \url{http://wiki.openwrt.org/OpenWrtDocs/Hardware/Asus/WL500G}
\item \url{http://wiki.openwrt.org/OpenWrtDocs/Hardware/Asus/WL500GP}
\item \url{http://www.freewrt.org/trac/wiki/Documentation/Hardware/AsusWL500G}
\item \url{http://www.freewrt.org/trac/wiki/Documentation/Hardware/AsusWL500GP}
\item \url{http://wl500g.dyndns.org/}
\item \url{http://oleg.wl500g.info/}
\item \url{http://au.asus.com/products.aspx?l1=12&l2=43}
\item \url{http://www.freifunk-bno.de/component/option,com_smf/Itemid,88/topic,910.msg10357/}
\item \url{http://www.cyber-wulf.de/a_wl500g.html}
\item \url{http://wiki.openwrt.org/OpenWrtDocs/Hardware/Asus/WL500G}
\item \url{http://forum.opennet-initiative.de/print.php?threadid=505&page=6&sid=460903353d70c65fad4960105ab76cdd}
\item \url{http://forum.openwrt.org/viewtopic.php?pid=41756}
\item \url{http://www.familie-prokop.de/asus-wl500gp/index.html}
\end{wlanlink}

\end{wlandevice}

\paragraph{Andere WLAN-Router}

\begin{itemize}

\item Netgear HR314

802.11a WLAN-Router, unterst"utzt Ad-Hoc- und Infrastruktur-Modus,
keine Open-Source Firmware vorhanden, kostet ca. 30 Euro

\url{http://www.wi-fiplanet.com/reviews/article.php/1559091}

\end{itemize}

\subsubsection{Professionelle Router}

In diesem Abschnitt werden so genannte Stand-alone Mesh-Router betrachtet. 
Die Begriffe, die daf"ur oft als Synonyme verwendet werden, sind dabei: 

\begin{itemize}	
	\item Routerboards
	\item Stand-alone Mesh-Router 
	\item Minicomputers 
	\item Single-Board-Computers (SBC) 
	\item Access Points 	
\end{itemize}
 

Im Projekt (\url{http://umic-mesh.net}) wurden professionelle Router
eingesetzt, das sind spezielle Router-Boards mit Steckpl"atzen f"ur
MiniPCI WLAN-karten. Boards kosten etwa 100-200 Euro, dazu muss man allerdings
noch passende WLAN-Karten kaufen + Antennen + Kabel + Netzteil + Geh"ause,
also keine billige L"osung. 

Man k"onnte aber nur diese Karten kaufen + Adapter PCI-MiniPCI und in Rechner
einbauen (Das w"are dann die platzsparende Version von \emph{PCs + WLAN-Karte}).
WLAN-Karten z.B Wistron Neweb CM9 Atheros 802.11a/b/g Mini-PCI, hier 
\url{http://www.pcengines.ch/cm9.htm}.

Boards sind hier
\url{http://www.pcengines.ch/wrap.htm, http://www.pcengines.ch/alix.htm}

Es gibt noch diese kleine Mesh-Router, wie von Meraki. Die haben wohl
ihre eigene Firmware drin und eigene Routingprotokolle oder eigene
Implementierungen davon besser gesagt. Hier ein Paar, die 802.11a
unterst"utzen, sind aber outdoor, haben also gro"se Reichweiten.
Ob es sinnvoll ist, sie im Gebaude einzusetzen:
Aphelion 3300AG Outdoor Wireless Access Point - 802.11a/b/g,
Aphelion 600AG/605AG Intelligenter sequentieller Wireless Access Point f"ur
den Au"senbereich mit den Standards 802.11a/b/g 
\url{http://www.abcdata.de/abcdataneu/WLAN_MESH_Aphelion.php}
oder PLANET MAP-2100 indoor sind aber zum Teil sehr teuer (1200 Euro !!!). 

\textbf{Vorteile:}

\begin{itemize}	
	\item Outdoor (in unserem Fall unrelevant) 
	\item Gro�e Reichweiten
\end{itemize}
 

\textbf{Nachteile:} 

\begin{itemize}	
	\item Zum Teil sehr teuer (1200 Euro !!!)
\end{itemize}
 

\textbf{Links:}

\begin{itemize}	
	\item \url{http://wiki.opennet-initiative.de/index.php/WRAP}
	\item \url{http://www.abcdata.de/abcdataneu/WLAN_MESH_Aphelion.php}
	\item \url{http://www.aerial.net/shop/product_info.php?cPath=33&products_id=351}
	\item \url{http://forum.openwrt.org/viewtopic.php?id=9655}
\end{itemize}

\subsection{Access Points}

Ein WLAN-Accesspoint ist der Verbindungspunkt eines kabelbasierten
Netzwerkes zu einem WLAN. Der Accesspoint ist Basisstation f"ur alle
WLAN-Clienten, zu der sie eine drahtlose Verbindung aufbauen.
Sendet ein WLAN-Client Daten, die f"ur einen Empf"anger im kabelbasierten
Netzwerkteil bestimmt sind, so \emph{reicht} der Accesspoint diese Daten "uber
das Kabelnetz an den Empf"anger weiter. Weiterhin kann ein Accesspoint
auch mehrere WLAN-Clienten untereinander verbinden. Somit ist der Accesspoint
quasi ein kabelloser Switch. 

Dieser hat (je nach Austattung) einige der folgenden Optionen: 
\begin{itemize}
	\item Ein oder mehrer integrierte WLAN-Module
	\item Einen integrierten DHCP-Server 
	\item Umfangreiche Sicherheits- und Verschl"usselungsmoglichkeiten 
	

		\item WEP, WPA und WPA2 dienen der Verschl"usselung der zu ubetragenden Daten 
		\item MAC-Filter und SSID Optionen 
		\item Einstellungen bez"uglich dem Remotezugriff 

	
	\item Verschiedene Arbeitsmodi 

		\item Accesspoint (AP) 
		\item Bridge (Point-to-Point oder Point-to-Multipoint) 
		\item Repeater
		\item MESSID 

\end{itemize}
 
Intel PRO/Wireless 5000 

\begin{itemize}
  \item \url{http://support.intel.com/support/wireless/wlan/pro5000/accesspoint}
  \item \url{http://www.pcmag.com/article2/0,1759,5524,00.asp}
\end{itemize}

Linksys WAP55AG 

\begin{itemize}
  \item \url{http://www.tomsnetworking.de/content/aktuelles/news_beitrag/news/851/6/index.html}
\end{itemize}

NETGEAR WAB102 

\begin{itemize}
  \item \url{http://kbserver.netgear.com/products/WAB102.asp}
  \item \url{http://reviews.cnet.com/wireless-access-points/netgear-wab102-802-11a/4505-3265_7-20708150.html}
 \item \url{http://archive.cert.uni-stuttgart.de/bugtraq/2003/12/msg00159.html}
\end{itemize}


\subsection{Mesh-Clients - PDAs und Handys}

PDAs und Handys sind typische Clients f"ur ein WMN (siehe \ref{sec:WMN}).
Sie verbinden sich mit einem Mesh-Router und kommunizieren mit dem WMN
"uber diesen Mesh-Router.  Diese Mesh-Clients nehmen am Routing im WMN
nicht teil. Sie schicken bzw. empfangen alle ihre Daten an bzw. vom
assozierten Mesh-Router.

Zur Zeit sind die WLAN-f"ahige PDAs und Handys sehr teuer und
es gibt nur sehr wenige, die den Standard IEEE 802.11a unterst"utzen.

Hier sind einige WLAN-f"ahige PDAs und Handys:

\begin{itemize}	

\item Apple iPhone Smartphone

IEEE 802.11b/g

\url{http://pocketpccentral.net/smartphone/apple/iphone.htm}

\url{http://www.tomshardware.com/de/test-apple-iphone-vs-ipaq-hw6910-blackberry-pearl-htc-touch,testberichte-239720.html}

\item RIM BlackBerry 8820 Smartphone

IEEE 802.11a/b/g

\url{http://www.blackberry8800series.com}

\url{http://www.reghardware.co.uk/2007/08/14/review\_blackberry\_8820/}

\url{http://eu.blackberry.com/eng/devices/device-detail.jsp?navId=H0,C201,P563#tab\_tab\_overview}

\item Motorola Symbol MC70 Smartphone

IEEE 802.11a/b/g

\url{http://www.handheld-loesungen.com/symbol\_mc70.htm}

\item i-mate K-jam Smartphone

IEEE 802.11b/g

\url{http://www.lordpercy.com/imate\_kjam\_review.htm}

\url{http://www.mobiletechreview.com/i-mate\_K-JAM.htm}

\item Sony Ericsson G900 Smartphone

IEE 802.11b/g

\url{http://www.sonyericsson.com/cws/corporate/press/pressreleases/pressreleasedetails/g700andg900global-20080210}

\item OQO Model 02 UMPC

IEEE 802.11a/b/g

\url{http://www.worldofppc.com/HWTests/oqo02.htm}

\end{itemize}


\section{Systemsoftware f�r Mesh-Netzwerk}

Betriebsystem: Windows/ Linux - gleichwertig!!! \\

\textbf{Windows:} 
\begin{itemize}	
	\item Treber meistens vorhanden (eventuell update notwendig) Intel, Atheros - getestet 
	\item Olsr Daemon installiern und konfigurieren (GUI vorhanden) 
\end{itemize}

\textbf{Linux: }
\begin{itemize}	
	\item Madwifi installiern 
	\item Olsr Daemon installiern und konfigurieren
\end{itemize}

\subsection{Linux MadWiFi-Treiber}

Linux MadWifi-Treiber ist Linux Kernel Treiber fur WLAN-Karten mit
Atheros Chipsatz. Linux MadWifi-Treiber ist heutzutage einer der
fortgeschrittensten Linux Treiber fur WLAN-Karten. Der Treiber ist
stabil und hat eine gro?e Benutzergemeinschaft. Der MadWifi-Treiber
selbst ist Open-Source, verwendet aber eine propritare Softwareschicht
Hardware Abstraction Layer (HAL), die nur in binarer Form vorhanden
ist.   Das Hardware Abstraction Layer (HAL) wird vom MadWifi-Treiber
gebraucht, um die Atheros-Chips ansprechen zu konnen. Dafur wurde bisher
ein Closed-Source-Modul verwendet. Dies hat unter anderem damit zu tun,
dass die Atheros-Chipsatze prinzipiell auf Frequenzen funken konnten,
fur die sie nicht zugelassen sind - beispielsweise weil diese vom
Militar zur Kommunikation verwendet werden.   Durch das proprietare
Modul war der Madwifi-Treiber bisher jedoch von einer Aufnahme in den
Linux-Kernel ausgeschlossen. Die Entwickler hatten au?erdem das Problem,
dass sie Fehler unter Umstanden nicht beheben konnten, da sie nicht
nachvollziehen konnten, wie der HAL-Baustein arbeitet.   MadWifi
selbst wird daher ab sofort nicht weiterentwickelt. Stattdessen
setzen die Programmierer auf OpenHAL, eine Linux-Portierung des
HAL-Modules des in OpenBSD verfugbaren freien Atheros-Treibers. In der
Vergangenheit wurde vom Software Freedom Law Center (SFLC) bestatigt,
dass die durch Reverse Engineering entstandene Software keine Copyrights
verletzt. Solche Behauptungen hatten die Entwicklung lange ausgebremst. 
 Der neue Treiber "Ath5k" wird MadWifi nun ersetzen und soll nicht
nur die freie Komponente OpenHAL einsetzen, sondern auch mit dem neuen
Linux-WLAN-System Mac80211 zusammenarbeiten, so dass der Treiber in den
offiziellen Linux-Kernel gelangen kann. MadWifi soll jedoch weiter mit
Fehlerkorrekturen und HAL-Updates versorgt werden. 

\subsection{Ad-Hoc Routing-Protokolle}

\subsubsection{OLSR (Optimized Link State Routing)}

Optimized Link State Routing, kurz OLSR, ist ein Routingprotokoll
fur mobile Ad-hoc-Netze, das eine an die Anforderungen eines mobilen
drahtlosen LANs angepasste Version des Link State Routing darstellt. Es
wurde von der IETF mit dem RFC 3626 standardisiert. Bei diesem
verteilten flexiblen Routingverfahren ist allen Routern die vollstandige
Netztopologie bekannt, sodass sie von Fall zu Fall den kurzesten Weg zum
Ziel festlegen konnen. Als proaktives Routingprotokoll halt es die dafur
benotigten Informationen jederzeit bereit.   Ein in Mesh-Netzwerken
bekannter Vertreter von LSR ist OLSR von olsr.org. Inzwischen existieren
fur OLSR spezielle Erweiterungen. Mit der ETX-Erweiterung wird dem
Umstand Rechnung getragen, dass Links asymmetrisch sein konnen. Mit
dem Fisheye-Algorithmus ist OLSR auch fur gro?ere Netzwerke brauchbar
geworden, da Routen zu weiter entfernten Knoten weniger haufig neu
berechnet werden. Der entscheidende Nachteil ist aber der trotz
Fisheye-Algorithmus noch recht hohe Rechenaufwand von OLSRD, sobald
die Anzahl an Knoten ein gewisses Ma� ubersteigt (siehe Erfahrungen mit
den kapazitativ arg begrenzten CPUs der kleinen Meshrouter im Berliner
Freifunk-Netz).  

\subsubsection{B.A.T.M.A.N. (BETTER APPROACH TO MOBILE ADHOC NETWORKING)}

Ausgehend von den Erfahrungen mit Freifunk-OLSR begannen die Entwickler
aus der Freifunk-Community im Marz 2006 in Berlin damit, ein neues
Routingprotokoll fur drahtlose Meshnetzwerke zu entwickeln. Alle bisher
bekannten Routingalgorithmen versuchen, Routen entweder zu berechnen
(proaktive Verfahren) oder sie dann zu suchen, wenn sie gebraucht werden
(reaktive Verfahren). Das neue Protokoll B.A.T.M.A.N. berechnet oder
sucht im Gegensatz zu diesen Protokollen keine Routen ? es erfasst
lediglich, ob Routen zu anderen Knoten existieren und uberwacht ihre
Qualitat. Dabei interessiert es sich nicht dafur, wie eine Route verlauft,
sondern ermittelt lediglich, uber welchen direkten Nachbarn ein bestimmter
Netzwerkknoten am besten zu erreichen ist, und tragt diese Information
proaktiv in die Routingtabelle ein. 

\subsection{OpenWRT}

OpenWRT ist eine GNU/Linux-Distribution f�r WLAN-Router. Anstatt einer
statischen Firmware setzt OpenWRT auf ein voll beschreibbares Dateisystem
sowie einen Paketmanager. OpenWRT l�uft unter anderem auf Ger�ten der
Firmen Linksys, ALLNET, ASUS, Belkin, Buffalo, Microsoft und Siemens.

Vorteile:
\begin{itemize}
\item Flexibilit�t
\item Erweiterbarkeit
\item Individualisierbarkeit
\item Sicherheit
\item Gewohnte Linux-Flexibilit�t und Funktionsumfang!!! 
\end{itemize}

Nachteile:
\begin{itemize}
\item Standardm��ig sind nur die n�tigsten Unix-Tools vorhanden 
\end{itemize}

Links:
\begin{itemize}
\item \url{http://openwrt.org/}
\item \url{http://toh.openwrt.org/}
\end{itemize}
\section{Tests}
Die Tests, die w�hrend der Fachstudie in Nexus-Labor mit der Test-Hardware 
durchgef�hrt wurden, sind in diesen Abschnitt detalliert beschrieben.

\subsection{Hardware}
\begin{itemize}
	\item 2 PCs mit \emph{Wistron CM9 Atheros AR5213A} Wlan Karten
  \item 1 Laptop mit Intel mini-PCI Wlan Karte
\end{itemize}

\subsection{Software}

\subsubsection{Madwifi}

Treiber sind nur fur Fedora 2.6.18 kernel!!!

1. Treiber installieren

\begin{verbatim}
	# svn checkout http://svn.madwifi.org/madwifi/trunk madwifi	
	# cd madwifi	
	# make	
	# make install
\end{verbatim}

2a. Treiber manuell laden 

\begin{verbatim}
	# modprobe ath_pci
\end{verbatim}

2b. Treiber automatisch laden

\begin{verbatim}
	# mkdir /etc/modules.autoload.d/
	# echo ath_pci >> /etc/modules.autoload.d/kernel-2.6
\end{verbatim}

3a. Network-Config automatisch

Datei erstellen:

\begin{verbatim}
	/etc/sysconfig/network-scripts/ifcfg-ath1
\end{verbatim}

und editieren..

\begin{verbatim}
	# Silicon Integrated Systems [SiS] SiS900 PCI Fast Ethernet
	DEVICE=ath1
	ONBOOT=yes
	
	BOOTPROTO=static
	IPADDR=192.168.2.1x
	NETMASK=255.255.255.0
	
	ESSID=mesh
	MODE=ad-hoc
	CHANNEL=36 # 5.18 GHz
	KEY=s:0nexus0suxen0
	# 108-Bit WEP 13 zeichen
\end{verbatim}

3b. Network-Config manuel

\begin{verbatim}
	# iwlist ath1 frequency
	
	          Channel 01 : 2.412 GHz
	          Channel 02 : 2.417 GHz
	          Channel 03 : 2.422 GHz
	          Channel 04 : 2.427 GHz
	          Channel 05 : 2.432 GHz
	          Channel 06 : 2.437 GHz
	          Channel 07 : 2.442 GHz
	          Channel 08 : 2.447 GHz
	          Channel 09 : 2.452 GHz
	          Channel 10 : 2.457 GHz
	          Channel 11 : 2.462 GHz
	          Channel 36 : 5.18 GHz
	          Channel 40 : 5.2 GHz
	          Channel 42 : 5.21 GHz
	          Channel 44 : 5.22 GHz
	          Channel 48 : 5.24 GHz
	          Channel 50 : 5.25 GHz
	          Channel 52 : 5.26 GHz
	          Channel 56 : 5.28 GHz
	          Channel 58 : 5.29 GHz
	          Channel 60 : 5.3 GHz
	          Channel 64 : 5.32 GHz
	          Channel 149 : 5.745 GHz
	          Channel 152 : 5.76 GHz
	          Channel 153 : 5.765 GHz
	          Channel 157 : 5.785 GHz
	          Channel 160 : 5.8 GHz
	          Channel 161 : 5.805 GHz
	          Channel 165 : 5.825 GHz
	          Current Channel=0	
	# ifconfig ath1 inet 192.168.0.1/24	
	# iwconfig ath1 essid mesh
	# iwconfig ath1 mode ad-hoc
	# iwconfig ath1 channel 36
	# iwconfig ath1 enc n1e2x3u4s5
\end{verbatim}


\subsubsection{OLSR Daemon}

1. olsrd installieren 

\begin{verbatim}
	# cvs -d:pserver:anonymous@olsrd.cvs.sourceforge.net:/cvsroot/olsrd login
	# cvs -z3 -d:pserver:anonymous@olsrd.cvs.sourceforge.net:/cvsroot/olsrd co olsrd-current
	# cd olsrd-current
	# make
	# make install
\end{verbatim}

2. Plug-ins fur olsrd installieren 

\begin{verbatim}
# cd lib/"plugin-name"
# make 
# make install 
# chcon -t textrel_shlib_t /usr/lib/olsrd_httpinfo.so.0.1 (!)
\end{verbatim}

3. olsrd kofigurieren

\begin{itemize}
	\item Datei /etc/olsrd.conf erstellen und editieren!!! (sieh file) 
	\item TCP Port 8080 fur Httpinfo und 8081 fur Dot UDP 698 fur Eingehende 
				Pakete erlauben. Datei /etc/sysconfig/iptables editieren: 
\end{itemize}

\begin{verbatim}
-A RH-Firewall-1-INPUT -p tcp --dport 8080 -m state --state NEW -j ACCEPT
-A RH-Firewall-1-INPUT -p tcp --dport 8081 -m state --state NEW -j ACCEPT
-A RH-Firewall-1-INPUT -i ath1 -p udp --sport 698 -j ACCEPT
\end{verbatim}

4. olsrd starten 

\begin{verbatim}
 # olsrd
\end{verbatim}


\subsection{Topologie}

    \begin{figure}[H]
      \centering
      \includegraphics[width=1.0\textwidth]{images/Topology.jpg}
      \caption{Topology}
      \label{fig:Topology}
    \end{figure}


\subsection{Ergebnisse}
\wlanimage{Olsr_Route}{Http Info}

\section{Fazit}

\subsection{�bersicht}

\begin{figure} [htbp]
	\centering
		\includegraphics[width=0.50\textwidth]{WMN}
	\caption{Mesh Netz}
	\label{fig:WMN}
\end{figure}

\begin{verbatim}
  Position := Wurzel;  
  for i in 1..m do
    if (Position = Kante) then	
      if Zeichen auf dem Pfad im Baum = s'(i) then   
\end{verbatim}

Fachstudie
Hardwareplattformen und Systemsoftware f?r drahtlose vermaschte
Kommunikationsnetze

Bearbeiter:	Alex Egorenkov, Sergey Telejnikov, Valeri Schneider
Betreuer: Dipl.-Inf. Frank D?rr
Pr?fer: Prof. Dr. Kurt Rothermel
Zeitraum: November 2007 - Januar 2008

Hintergrund:
Ein drahtloses vermaschtes Netz (engl. Wireless Mesh Network, WMN) besteht aus einer Menge von Knoten, die ?ber drahtlose Kommunikationstechniken wie beispielsweise IEEE 802.11 Nachrichten austauschen. Die Vemaschung der Knoten erm?glicht dabei nicht nur den Austausch von Nachrichten zwischen unmittelbar benachbarten Knoten, sondern auch die Vermittlung von Nachrichten an entfernte Knoten ?ber mehrere Knoten hinweg. Die Vermittlungsfunktionalit?t wird dabei oft von dedizierten Vermittlungsknoten (engl. Mesh Router) bereitgestellt, die somit eine drahtlose Kommunikationsinfrastruktur f?r die Klienten (engt. Mesh Client) bilden. Durch den Einsatz vergleichsweise kosteng?nstiger Hardwarekomponenten und die Vermaschung der Knoten erm?glichen WMNs die kosteng?nstige Vernetzung auch gr??erer Gebiete. Entsprechende Netze, werden beispielsweise von Community-Projekten wie dein Freifunkprojekt oder Firmen wie Google bereits heute in der Praxis f?r den Aufbau gr??erer Netze eingesetzt, um beispielsweise kosteng?nstige Internetzug?nge f?r Stadtteile oder ganze St?dte zu realisieren.

WMNs sind auch f?r den Sonderforschungsbereichs (SFB) Nexus an der Universit?t Stuttgart von gro?em Interesse. Im Zentrum der Forschungen des SFB stehen Umgebungsmodelle f?r mobile kontextbezogene Systeme. Umgebungsmodelle sind digitale Abbilder der physischen Welt, die von kontextbezogenen Systemen genutzt werden, um sich selbst?ndig an die physische Umgebung des Benutzers anzupassen. Ein einfaches Beispiel sind ortsbezogene Anwendungen, die beispielsweise aufgrund der aktuellen geographischen Position eines Ger?ts automatisch Informationen ?ber nahe Restaurants. Sehensw?rdigkeiten, usw. selektieren k?nnen. Zur Kommunikation. insbesondere mit mobilen Ger?ten, werden dabei hybride Systeme betrachtet, in denen sowohl eine infrastrukturbasierte Kommunikation als auch die direkte Ad-hoc-Kommunikation zwischen mobilen Endsystemen m?glich ist. Hierbei spielen WMNs als eine spezielle Auspr?gung eines hybriden Kommunikationssystems eine wesentliche Rolle.
Aufgabenstellung:
F?r Forschungszwecke soll innerhalb des SFB Nexus ein WMN installiert werden. Dieses WMN dient einerseits Nexus-Anwendungen, insbesondere Anwendungen auf mobilen Ger?ten, als Kommunikationsmedium. Andererseits soll dieses WNIN auch als Testbed zur Erforschung verschiedene Erweiterungen von WMNs dienen, beispielsweise der Untersuchung neuartige kontextbezogener Kommunikationsmechanismen, der Erforschung von Publish/Subscribe-Diensten f?r WMNs oder der Verwaltung von Umgebungsmodellen innerhalb eines hybriden Systems wie es ein WMN darstellt. Ziel dieser Fachstudie ist die Ausarbeitung einer Empfehlung f?r die Beschaffung entsprechender Ger?te (Hardwareplattformen und Systemsoftware) f?r den Aufbau eines WMN.
Das Vorgehen umfasst im einzelnen:

"	 Einarbeitung in grundlegende WMN-Technologien.
"	 Analyse der Anforderungen des Nexus-Projektes an ein WNN.
"	 Erstellung einer ?bersicht ?ber aktuelle verf?gbare Hardwareplattformen und Systemsoftware f?r     WMN.
"	 Bewertung der analysierten Systeme hinsichtlich der ermittelten Anforderungen und Ausarbeitung einer Empfehlung f?r eine geeignetes WNN hinsichtlich Hardwareplattform und Systemsoftware.
Die Ergebnisse der Studie sind in einer schriftlichen Ausarbeitung zu dokumentieren und in einem Vortrag innerhalb des Abteilungskolloquiums vorzustellen.
2
'PV.5



%---------------------------------------------------------------------&

\section{Anhang}

\subsection{olsrd.conf}
\label{olsrd.conf}

Listing der Datei \textbf{/etc/olsrd.conf}:
\lstinputlisting[language=bash,backgroundcolor=\color{shelllstbgcolor}]
{olsrd.conf}

\subsection{olsrd Startup-Skript}
\label{olsrd_startup.sh}

Listing der Datei \textbf{/etc/init.d/olsrd}:
\lstinputlisting[language=bash,backgroundcolor=\color{shelllstbgcolor}]
{olsrd_startup.sh}

\subsection{topology.pl}
\label{topology.pl}

Listing der Datei \textbf{/var/htppd/cgi-bin/topology.pl}:
\lstinputlisting[language=Perl,backgroundcolor=\color{shelllstbgcolor}]
{topology.pl}


%---------------------------------------------------------------------&
\end{document}
