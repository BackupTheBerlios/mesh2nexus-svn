\newpage
\subsubsection{PCMCIA WLAN-Karten}

Diese WLAN-Karten sind f"r Notebooks gedacht. Heutzutage ist es jedoch
"ublich, dass die Notebooks schon einen integriertes WLAN-Modul (Mini-PCI)
eingebaut haben. Damit ist die Notwendigkeit dieser Module nur noch f"ur
Notebooks "alteren Generationen notwendig. Die meisten Module haben keinen
Anschluss f"ur eine externe Antenne.

%%%%%%%%%%%%%%%%%%%%%%%%%%%%%%%%%%%%%%%%%%%%%%%%%%%%%%%%%%%%%%%%%%%%%%%%%%%%
%
% Netgear WAG311
%
%%%%%%%%%%%%%%%%%%%%%%%%%%%%%%%%%%%%%%%%%%%%%%%%%%%%%%%%%%%%%%%%%%%%%%%%%%%%
\begin{wlandevice}{Proxim Orinoco Gold 8480-WD}

\wlanimage{Proxim_Orinoco_Gold_8480WD}{Proxim Orinoco Gold 8480-WD}

\wlanchipset{Atheros AR5212}

\begin{wlanieeestandard}
\item 802.11a/b/g
\end{wlanieeestandard}

\begin{wlanmode}
\item Ad-Hoc
\item Infrastruktur
\end{wlanmode}

\begin{wlansecurity}
\item WEP (40-, 104-bit)
\item WPA
\item WPA2
\end{wlansecurity}

\begin{wlandriver}
\item
Unter Linux herrvorragende Unterst"utzung von MadWifi-Treiber,
auch Ad-Hoc-Modus.

\url{http://madwifi.org/}
\end{wlandriver}

\wlanprice{80}

\begin{wlaninstall}
\item
\url{http://madwifi.org/wiki/UserDocs/FirstTimeHowTo}
\end{wlaninstall}

\begin{wlanlink}
\item \url{http://www.proxim.com/products/wifi/client/abgcard/index.html}
\item \url{http://madwifi.org/wiki/Compatibility/Proxim}
\end{wlanlink}

\end{wlandevice}

%%%%%%%%%%%%%%%%%%%%%%%%%%%%%%%%%%%%%%%%%%%%%%%%%%%%%%%%%%%%%%%%%%%%%%%%%%%%
%
% Netgear WAG511
%
%%%%%%%%%%%%%%%%%%%%%%%%%%%%%%%%%%%%%%%%%%%%%%%%%%%%%%%%%%%%%%%%%%%%%%%%%%%%
\begin{wlandevice}{Netgear WAG511}

\wlanimage{Netgear_WAG511}{Netgear WAG511}

\wlanchipset{Atheros AR5001X+}

\begin{wlanieeestandard}
\item 802.11a/b/g
\end{wlanieeestandard}

\begin{wlanmode}
\item Ad-Hoc
\item Infrastruktur
\end{wlanmode}

\begin{wlansecurity}
\item WEP (40-, 104-, 128-bit)
\item WPA
\item WPA2
\item PTP, P2TP, IPSec, VPN pass-through
\end{wlansecurity}

\begin{wlandriver}
\item
Von Netgear werden nur Windows Treiber angeboten.

\url{http://www.netgear.de/de/Support/download.html?func=Detail&id=10676}

Unter Linux herrvorragende Unterst"utzung von MadWifi-Treiber,
auch Ad-Hoc-Modus.

\url{http://madwifi.org/}
\end{wlandriver}

\wlanprice{50-60}

\begin{wlaninstall}
\item
\url{http://www.lrz-muenchen.de/services/netz/mobil/funklan-installation/installation/windowsxp_wg511/flan-instwxp.html}

\url{http://madwifi.org/wiki/UserDocs/FirstTimeHowTo}
\end{wlaninstall}

\begin{wlanlink}
\item \url{http://www.netgear.com/Products/Adapters/AGDualBandWirelessAdapters/WAG511.aspx}
\item \url{http://www.netgear.de/de/Support/download.html?func=Detail&id=10676}
\item \url{http://www.netgear.de/Produkte/Wireless/DualBand/WAG511/index.html}
\item \url{http://madwifi.org/wiki/Compatibility/Netgear}
\end{wlanlink}

\end{wlandevice}

%%%%%%%%%%%%%%%%%%%%%%%%%%%%%%%%%%%%%%%%%%%%%%%%%%%%%%%%%%%%%%%%%%%%%%%%%%%%
%
% SMC 2536W-AG
%
%%%%%%%%%%%%%%%%%%%%%%%%%%%%%%%%%%%%%%%%%%%%%%%%%%%%%%%%%%%%%%%%%%%%%%%%%%%%
\begin{wlandevice}{SMC 2536W-AG}

\wlanimage{SMC_2536WAG}{SMC 2536W-AG}

\wlanchipset{Atheros AR5001}

\begin{wlanieeestandard}
\item 802.11a/b/g
\end{wlanieeestandard}

\begin{wlanmode}
\item Ad-Hoc
\item Infrastruktur
\end{wlanmode}

\begin{wlansecurity}
\item WEP (40-, 104-, 128-bit)
\item WPA
\item WPA2
\end{wlansecurity}

\begin{wlandriver}
\item
Von SMC werden nur Treiber f"ur Windows angeboten.

\url{http://www.smc.com/index.cfm?event=downloads.searchResultsDetail&localeCode=EN_USA&productCategory=9&partNumber=2916&modelNumber=348&knowsPartNumber=false&userPartNumber=&docId=3103}

Unter Linux herrvorragende Unterst"utzung von MadWifi-Treiber,
auch Ad-Hoc-Modus.

\url{http://madwifi.org/}
\end{wlandriver}

\wlanprice{80}

\begin{wlaninstall}
\item
\url{http://madwifi.org/wiki/UserDocs/FirstTimeHowTo}
\end{wlaninstall}

\begin{wlanlink}
\item \url{http://www.smc.com/index.cfm?event=viewProduct&cid=9&scid=49&localeCode=EN\%5FUSA&pid=348}
\item \url{http://www.smc.com/files/AC/2536Wag_Ds_ww.pdf}
\item \url{http://madwifi.org/wiki/Compatibility/SMC}
\item \url{http://forums.fedoraforum.org/archive/index.php/t-101517.html}
\end{wlanlink}

\end{wlandevice}

%%%%%%%%%%%%%%%%%%%%%%%%%%%%%%%%%%%%%%%%%%%%%%%%%%%%%%%%%%%%%%%%%%%%%%%%%%%%
%
% Linksys WPC55AG
%
%%%%%%%%%%%%%%%%%%%%%%%%%%%%%%%%%%%%%%%%%%%%%%%%%%%%%%%%%%%%%%%%%%%%%%%%%%%%
\begin{wlandevice}{Linksys WPC55AG}

\wlanimage{Linksys_WPC55AG}{Linksys WPC55AG}

\wlanchipset{Atheros AR5212 oder AR5006X}

\begin{wlanieeestandard}
\item 802.11a/b/g
\end{wlanieeestandard}

\begin{wlanmode}
\item Ad-Hoc
\item Infrastruktur
\end{wlanmode}

\begin{wlansecurity}
\item WEP (40-, 104-, 128-bit)
\item WPA
\item WPA2
\end{wlansecurity}

\begin{wlandriver}
\item
Von Linksys werden nur Treiber f"ur Windows bereitgestellt.

\url{http://www.linksys.com/servlet/Satellite?c=L_Product_C2&childpagename=US\%2FLayout&cid=1115416827328&pagename=Linksys\%2FCommon\%2FVisitorWrapper}

Auch hier kann man Treiber f"ur Windows finden:

\url{http://www.phoenixnetworks.net/atheros.php}

Unter Linux herrvorragende Unterst"utzung von MadWifi-Treiber,
auch Ad-Hoc-Modus.

\url{http://madwifi.org/}
\end{wlandriver}

\wlanprice{50-60}

\begin{wlaninstall}
\item
\url{http://madwifi.org/wiki/UserDocs/FirstTimeHowTo}
\end{wlaninstall}

\begin{wlanlink}
\item \url{http://www.linksys.com/servlet/Satellite?c=L_Product_C2&childpagename=US\%2FLayout&cid=1115416827328&pagename=Linksys\%2FCommon\%2FVisitorWrapper}
\item \url{http://www.phoenixnetworks.net/atheros.php}
\item \url{http://madwifi.org/}
\item \url{http://madwifi.org/wiki/Compatibility/Linksys}
\item \url{http://reviews.cnet.com/adapters-nics/linksys-wpc55ag-dual-band/4505-3380_7-21128291.html}
\item \url{http://www.google.de/search?q=Linksys+WPC55AG}
\item \url{http://lists.funkfeuer.at/pipermail/discuss/2006-September/001592.html}
\item \url{http://www.uk-surplus.com/manuals/brochures/linksyswpc55duser.pdf}
\end{wlanlink}

\end{wlandevice}

\paragraph{Andere PCMCIA-WLAN-Karten}

\begin{itemize}

\item Intel PRO/Wireless 5000

Chipsatz Intel,
802.11a WLAN PCMCIA-Karte, unterst"utzt Ad-Hoc- und Infrastruktur-Modus,
Treiber von Intel nur f"ur Windows vorhanden, f"ur Linux werden keine
Treiber entwickelt, kostet ca. 150 Euro

\url{http://www.intel.com/support/wireless/wlan/pro5000/lancardbus}

\item Netgear WAB501
Chipsatz Atheros AR5211,
802.11a/b WLAN-Karte,
MadWifi-Treiber Unterst"utzung

\url{http://kbserver.netgear.com/products/WAB501.asp}\\
\url{http://madwifi.org/wiki/Compatibility/Netgear}

\item Netgear WG511U

Chipsatz Atheros AR5004X,
802.11a/g WLAN-Karte,
MadWifi-Treiber Unterst"utzung

\url{http://www.netgear.com/Products/Adapters/AGDualBandWirelessAdapters/WG511U.aspx}\\
\url{http://madwifi.org/wiki/Compatibility/Netgear}

\item Proxim Orinoco Silver 8481-WD

Chipsatz Atheros AR5001X+,
802.11a/b/g WLAN-Karte,
MadWifi-Treiber Unterst"utzung,
kostet ca 80-90 Euro

\url{http://www.proxim.com/products/wifi/client/abgcard/index.html}\\
\url{http://madwifi.org/wiki/Compatibility/Proxim}

\item Cisco Aironet CB21AG

Chipsatz Atheros 5212,
802.11a/b/g WLAN-Karte,
Madwifi-Treiber Unterst"utzung,
kostet ca 100 Euro

\url{http://madwifi.org/wiki/Compatibility/Cisco}

\end{itemize}
