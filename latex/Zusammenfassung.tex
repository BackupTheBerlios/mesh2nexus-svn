\section{Fazit}

(TODO) TEXT

\subsection{�bersicht}

Dieser Abschnitt stellt ein "Ubersicht "uber aktuelle verf"ugbare 
Hardwareplattformen und Systemsoftware f"ur WMN dar.
	
\begin{figure}[htbp]
\begin{center}
\setlength{\extrarowheight}{4pt}
\begin{tabular}{|l|c|c|c|c|c|c|c|c|c|c|}
\hline
% HEAD BEGIN
 &
\rotatebox{90}{IEEE 802.11a} & \rotatebox{90}{Ad-Hoc Modus} &
\rotatebox{90}{Treiber (Linux/Windows)} & \rotatebox{90}{Open-Source Firmware} &
\rotatebox{90}{LAN-Anschluss} & \rotatebox{90}{Sicherheit} &
\rotatebox{90}{Installation} & \rotatebox{90}{Konfiguration} &
\rotatebox{90}{Mini-PCI Slot} & \rotatebox{90}{IEEE 802.11n}\\
% HEAD END
\hline
Linksys WMP55AG                  & ++  & + & ++ & -   & ++ & ++  & + & + & -  & - \\
\hline
Netgear WAG311                   & ++  & + & ++ & -   & ++ & ++  & + & + & -  & - \\
\hline
D-Link DWL-A520                  & +   & + & ++ & -   & ++ & +   & + & + & -  & - \\
\hline
Gigabyte GN-WPEAG                & ++  & + & ++ & -   & ++ & +++ & + & + & -  & - \\
\hline
Intel PRO/Wireless 5000          & +   & + & +  & -   & ++ & +   & + & + & -  & - \\
\hline
D-Link DWL-AG530                 & ++  & + & ++ & -   & ++ & +++ & + & + & -  & - \\
\hline
D-Link DWL-G550                  & ++  & + & ++ & -   & ++ & +++ & + & + & -  & - \\
\hline
\hline
Wistron CM9 Atheros AR5213A      & ++  & + & ++ & -   & ++ & +++ & + & + & -  & - \\
\hline
Intel PRO/Wireless 3945          & ++  & + & ++ & -   & ++ & +++ & + & + & -  & - \\
\hline
Intel PRO/Wireless 2915          & ++  & + & ++ & -   & ++ & +++ & + & + & -  & - \\
\hline
Intel Wireless WiFi Link 4965AGN & +++ & + & ++ & -   & ++ & +++ & + & + & -  & + \\
\hline
\hline
Linksys WRT54G v1.0              & ++  & + & -  & ++  & +  & +++ & - & + & +  & - \\
\hline
Linksys WRT55AG                  & ++  & + & -  & +   & +  & +   & - & + & ++ & - \\
\hline
Asus WL500G/GP                   & ++  & + & -  & +++ & +  & +++ & - & + & +  & - \\
\hline
Netgear HR314                    & +   & - & -  & -   & +  & +   & - & + & -  & - \\
\hline
\end{tabular}
\end{center}
\caption{�bersicht und Bewertung von  Hardware}
\end{figure}

\begin{figure}[htbp]
\begin{center}
\setlength{\extrarowheight}{4pt}
\begin{tabular}{|l|c|c|c|c|}
\hline
% HEAD BEGIN
 &
\rotatebox{90}{Betriebssysteme} & \rotatebox{90}{Installation} &
\rotatebox{90}{Konfiguration} & \rotatebox{90}{Visualisierung}\\
% HEAD END
\hline
OLSRD        & +++ & + & + & ++ \\
\hline
B.A.T.M.A.N. & +   & + & + & -  \\
\hline
Meshcom & +   & + & - & +  \\
\hline
\end{tabular}
\end{center}
\caption{�bersicht und Bewertung von Software}
\end{figure}


\subsection{Aufwandsch"atzung und Bewertung}

(TODO) Bewertung der analysierten Systeme hinsichtlich 
der ermittelten Anforderungen.. 	
	
\subsection{Empfehlung f"ur eine geeignete WMN}

(TODO) Ausarbeitung einer Empfehlung f"ur eine geeignete 
WMN hinsichtlich Hardwareplattform und Systemsoftware	
