\section{Fazit}

\subsection{�bersicht}

Dieser Abschnitt stellt ein "Ubersicht "uber aktuelle verf"ugbare 
Hardwareplattformen und Systemsoftware f"ur WMN dar. Auf Abbildung
\ref{fig:"Ubersicht und Bewertung von Hardware}
wird die "Ubersicht und unsere Bewertung der Hardware und auf
Abbildung \ref{fig:"Ubersicht und Bewertung von Software} wird
die "Ubersicht und unsere Bewertung der Software dargestellt.
	
\begin{figure}[h]
\begin{center}
\setlength{\extrarowheight}{4pt}
\capstart
\begin{tabular}{|p{4cm}|c|c|c|c|c|c|c|c|c|c|}
\hline
% HEAD BEGIN
 &
\rotatebox{90}{IEEE 802.11a} & \rotatebox{90}{Ad-Hoc Modus} &
\rotatebox{90}{Treiber (Linux/Windows)} & \rotatebox{90}{Open-Source Firmware} &
\rotatebox{90}{LAN-Anschluss} & \rotatebox{90}{Sicherheit} &
\rotatebox{90}{Installation} & \rotatebox{90}{Konfiguration} &
\rotatebox{90}{Mini-PCI Slot} & \rotatebox{90}{IEEE 802.11n}\\
% HEAD END
\hline
Linksys WMP55AG                  & ++  & + & ++ & -   & ++ & ++  & ++ & + & -  & - \\
\hline
Netgear WAG311                   & ++  & + & ++ & -   & ++ & ++  & ++ & + & -  & - \\
\hline
D-Link DWL-A520                  & +   & + & ++ & -   & ++ & +   & ++ & + & -  & - \\
\hline
Gigabyte GN-WPEAG                & ++  & + & ++ & -   & ++ & +++ & ++ & + & -  & - \\
\hline
Intel PRO/Wireless 5000          & +   & + & +  & -   & ++ & +   & ++ & + & -  & - \\
\hline
D-Link DWL-AG530                 & ++  & + & ++ & -   & ++ & +++ & ++ & + & -  & - \\
\hline
D-Link DWL-G550                  & ++  & + & ++ & -   & ++ & +++ & ++ & + & -  & - \\
\hline
\hline
Wistron CM9 Atheros AR5213A      & ++  & + & ++ & -   & ++ & +++ & ++ & + & -  & - \\
\hline
Intel PRO/Wireless 3945          & ++  & + & ++ & -   & ++ & +++ & +  & + & -  & - \\
\hline
Intel PRO/Wireless 2915          & ++  & + & ++ & -   & ++ & +++ & +  & + & -  & - \\
\hline
Intel Wireless WiFi Link 4965AGN & +++ & + & ++ & -   & ++ & +++ & +  & + & -  & + \\
\hline
\hline
Linksys WRT54G v1.0              & ++  & + & -  & ++  & +  & +++ & -  & + & +  & - \\
\hline
Linksys WRT55AG                  & ++  & + & -  & +   & +  & +   & -  & + & ++ & - \\
\hline
Asus WL500G/GP                   & ++  & + & -  & +++ & +  & +++ & -  & + & +  & - \\
\hline
Netgear HR314                    & +   & - & -  & -   & +  & +   & -  & + & -  & - \\
\hline
\end{tabular}
\end{center}
\caption{"Ubersicht und Bewertung von Hardware}
\label{fig:"Ubersicht und Bewertung von Hardware}
\end{figure}

\begin{figure}[h]
\begin{center}
\setlength{\extrarowheight}{4pt}
\capstart
\begin{tabular}{|l|c|c|c|c|c|}
\hline
% HEAD BEGIN
 &
\rotatebox{90}{Betriebssysteme} & \rotatebox{90}{Installation} &
\rotatebox{90}{Konfiguration} & \rotatebox{90}{Visualisierung} &
\rotatebox{90}{Open-Source}\\
% HEAD END
\hline
OLSRD        & +++ & + & + & ++ & +\\
\hline
B.A.T.M.A.N. & +   & + & + & -  & +\\
\hline
Meshcom      & +   & + & - & +  & -\\
\hline
\end{tabular}
\end{center}
\caption{"Ubersicht und Bewertung von Software}
\label{fig:"Ubersicht und Bewertung von Software}
\end{figure}
	
\subsection{Empfehlung f"ur ein geeignetes WMN}

Um eines stabiles, f"ur die Forschung geeignets Mesh-Netzwerk aufzubauen, 
sind vor allem folgende Hardware zu empfehlen: 

\begin{itemize}
	\item PCs + Wlan Karten mit Atheros Chipsatz (z.b \emph{Wistron CM9 Atheros AR5213A})
\end{itemize}

Weiterhin kann das Mesh-Netzwerk zu einem heterogenen WMN erweitert werden, 
indem folgende WLAN-Router eingesetzt werden:

\begin{itemize}
	\item Linksys WRT54G v1.0  
\end{itemize}

Als Software-System f"ur das aufzubauende WMN hat sich Linux als Betriebsystem 
und \textbf{olsrd} als Routing Software als besonders geeignet erwiesen. 

Um das ganze Informatikgeb"aude abzudecken, w"urden ca. 30-45 PCs reichen.
Das heisst ca. 10-15 PCs pro Stock.
(TODO) BILD!
Jeder Rechner ist dabei mit zwei Wlan-Karten zu gestaten, um MIMO zu realisieren..
Kosten f"ur die notwendige Hardware liegen damit unter der Budget-Grenze.
