\section{Systemsoftware f�r Mesh-Netzwerk}

Betriebsystem: Windows/ Linux - gleichwertig!!! \\

\textbf{Windows:} 
\begin{itemize}	
	\item Treber meistens vorhanden (eventuell update notwendig) Intel, Atheros - getestet 
	\item Olsr Daemon installiern und konfigurieren (GUI vorhanden) 
\end{itemize}

\textbf{Linux: }
\begin{itemize}	
	\item Madwifi installiern 
	\item Olsr Daemon installiern und konfigurieren
\end{itemize}

\subsection{Linux MadWiFi-Treiber}

Linux MadWifi-Treiber ist Linux Kernel Treiber fur WLAN-Karten mit
Atheros Chipsatz. Linux MadWifi-Treiber ist heutzutage einer der
fortgeschrittensten Linux Treiber fur WLAN-Karten. Der Treiber ist
stabil und hat eine gro?e Benutzergemeinschaft. Der MadWifi-Treiber
selbst ist Open-Source, verwendet aber eine propritare Softwareschicht
Hardware Abstraction Layer (HAL), die nur in binarer Form vorhanden
ist.   Das Hardware Abstraction Layer (HAL) wird vom MadWifi-Treiber
gebraucht, um die Atheros-Chips ansprechen zu konnen. Dafur wurde bisher
ein Closed-Source-Modul verwendet. Dies hat unter anderem damit zu tun,
dass die Atheros-Chipsatze prinzipiell auf Frequenzen funken konnten,
fur die sie nicht zugelassen sind - beispielsweise weil diese vom
Militar zur Kommunikation verwendet werden.   Durch das proprietare
Modul war der Madwifi-Treiber bisher jedoch von einer Aufnahme in den
Linux-Kernel ausgeschlossen. Die Entwickler hatten au?erdem das Problem,
dass sie Fehler unter Umstanden nicht beheben konnten, da sie nicht
nachvollziehen konnten, wie der HAL-Baustein arbeitet.   MadWifi
selbst wird daher ab sofort nicht weiterentwickelt. Stattdessen
setzen die Programmierer auf OpenHAL, eine Linux-Portierung des
HAL-Modules des in OpenBSD verfugbaren freien Atheros-Treibers. In der
Vergangenheit wurde vom Software Freedom Law Center (SFLC) bestatigt,
dass die durch Reverse Engineering entstandene Software keine Copyrights
verletzt. Solche Behauptungen hatten die Entwicklung lange ausgebremst. 
 Der neue Treiber "Ath5k" wird MadWifi nun ersetzen und soll nicht
nur die freie Komponente OpenHAL einsetzen, sondern auch mit dem neuen
Linux-WLAN-System Mac80211 zusammenarbeiten, so dass der Treiber in den
offiziellen Linux-Kernel gelangen kann. MadWifi soll jedoch weiter mit
Fehlerkorrekturen und HAL-Updates versorgt werden. 

\subsection{Ad-Hoc Routing-Protokolle}

\subsubsection{OLSR (Optimized Link State Routing)}

Optimized Link State Routing, kurz OLSR, ist ein Routingprotokoll
fur mobile Ad-hoc-Netze, das eine an die Anforderungen eines mobilen
drahtlosen LANs angepasste Version des Link State Routing darstellt. Es
wurde von der IETF mit dem RFC 3626 standardisiert. Bei diesem
verteilten flexiblen Routingverfahren ist allen Routern die vollstandige
Netztopologie bekannt, sodass sie von Fall zu Fall den kurzesten Weg zum
Ziel festlegen konnen. Als proaktives Routingprotokoll halt es die dafur
benotigten Informationen jederzeit bereit.   Ein in Mesh-Netzwerken
bekannter Vertreter von LSR ist OLSR von olsr.org. Inzwischen existieren
fur OLSR spezielle Erweiterungen. Mit der ETX-Erweiterung wird dem
Umstand Rechnung getragen, dass Links asymmetrisch sein konnen. Mit
dem Fisheye-Algorithmus ist OLSR auch fur gro?ere Netzwerke brauchbar
geworden, da Routen zu weiter entfernten Knoten weniger haufig neu
berechnet werden. Der entscheidende Nachteil ist aber der trotz
Fisheye-Algorithmus noch recht hohe Rechenaufwand von OLSRD, sobald
die Anzahl an Knoten ein gewisses Ma� ubersteigt (siehe Erfahrungen mit
den kapazitativ arg begrenzten CPUs der kleinen Meshrouter im Berliner
Freifunk-Netz).  

\subsubsection{B.A.T.M.A.N. (BETTER APPROACH TO MOBILE ADHOC NETWORKING)}

Ausgehend von den Erfahrungen mit Freifunk-OLSR begannen die Entwickler
aus der Freifunk-Community im Marz 2006 in Berlin damit, ein neues
Routingprotokoll fur drahtlose Meshnetzwerke zu entwickeln. Alle bisher
bekannten Routingalgorithmen versuchen, Routen entweder zu berechnen
(proaktive Verfahren) oder sie dann zu suchen, wenn sie gebraucht werden
(reaktive Verfahren). Das neue Protokoll B.A.T.M.A.N. berechnet oder
sucht im Gegensatz zu diesen Protokollen keine Routen ? es erfasst
lediglich, ob Routen zu anderen Knoten existieren und uberwacht ihre
Qualitat. Dabei interessiert es sich nicht dafur, wie eine Route verlauft,
sondern ermittelt lediglich, uber welchen direkten Nachbarn ein bestimmter
Netzwerkknoten am besten zu erreichen ist, und tragt diese Information
proaktiv in die Routingtabelle ein. 

\subsection{OpenWRT}

OpenWRT ist eine GNU/Linux-Distribution f�r WLAN-Router. Anstatt einer
statischen Firmware setzt OpenWRT auf ein voll beschreibbares Dateisystem
sowie einen Paketmanager. OpenWRT l�uft unter anderem auf Ger�ten der
Firmen Linksys, ALLNET, ASUS, Belkin, Buffalo, Microsoft und Siemens.

Vorteile:
\begin{itemize}
\item Flexibilit�t
\item Erweiterbarkeit
\item Individualisierbarkeit
\item Sicherheit
\item Gewohnte Linux-Flexibilit�t und Funktionsumfang!!! 
\end{itemize}

Nachteile:
\begin{itemize}
\item Standardm��ig sind nur die n�tigsten Unix-Tools vorhanden 
\end{itemize}

Links:
\begin{itemize}
\item \url{http://openwrt.org/}
\item \url{http://toh.openwrt.org/}
\end{itemize}