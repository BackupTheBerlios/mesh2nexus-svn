\section{Systemsoftware f"ur Mesh-Netzwerk}

Betriebsystem: Windows/Linux - gleichwertig!!!\\

\textbf{Windows:} 
\begin{itemize}	
	\item Treiber meistens vorhanden (eventuell update notwendig)
	Intel, Atheros - getestet

	\item OLSR-daemon installieren und konfigurieren (GUI vorhanden) 
\end{itemize}

\textbf{Linux: }
\begin{itemize}	
	\item MadWifi-Treiber installieren 
	\item OLSR-daemon installieren und konfigurieren
\end{itemize}

\subsection{Linux MadWiFi-Treiber}

Linux MadWifi-Treiber ist Linux Kernel Treiber f"ur WLAN-Karten mit
Atheros-Chipsatz. Linux MadWifi-Treiber ist heutzutage einer der
fortgeschrittensten Linux Treiber f"ur WLAN-Karten. Der Treiber ist
stabil und hat eine gro"se Benutzergemeinschaft. Der MadWifi-Treiber
selbst ist Open-Source, verwendet aber eine proprit"are Softwareschicht
Hardware Abstraction Layer (HAL), die nur in bin"arer Form vorhanden
ist.   Das Hardware Abstraction Layer (HAL) wird vom MadWifi-Treiber
gebraucht, um die Atheros-Chips ansprechen zu k"onnen. Daf"ur wurde bisher
ein Closed-Source-Modul verwendet. Dies hat unter anderem damit zu tun,
dass die Atheros-Chips"atze prinzipiell auf Frequenzen funken k"onnten,
f"ur die sie nicht zugelassen sind - beispielsweise weil diese vom
Milit"ar zur Kommunikation verwendet werden.   Durch das proprit"are
Modul war der MadWifi-Treiber bisher jedoch von einer Aufnahme in den
Linux-Kernel ausgeschlossen. Die Entwickler hatten au"serdem das Problem,
dass sie Fehler unter Umst"anden nicht beheben konnten, da sie nicht
nachvollziehen konnten, wie der HAL-Baustein arbeitet.   MadWifi
selbst wird daher ab sofort nicht weiterentwickelt. Stattdessen
setzen die Programmierer auf OpenHAL, eine Linux-Portierung des
HAL-Modules des in OpenBSD verf"ugbaren freien Atheros-Treibers. In der
Vergangenheit wurde vom Software Freedom Law Center (SFLC) best"atigt,
dass die durch Reverse Engineering entstandene Software keine Copyrights
verletzt. Solche Behauptungen hatten die Entwicklung lange ausgebremst. 
Der neue Treiber "`Ath5k"' wird MadWifi nun ersetzen und soll nicht
nur die freie Komponente OpenHAL einsetzen, sondern auch mit dem neuen
Linux-WLAN-System Mac80211 zusammenarbeiten, so dass der Treiber in den
offiziellen Linux-Kernel gelangen kann. MadWifi soll jedoch weiter mit
Fehlerkorrekturen und HAL-Updates versorgt werden. 

\subsection{Ad-Hoc Routing-Protokolle}

\subsubsection{OLSR (Optimized Link State Routing)}

Optimized Link State Routing, kurz OLSR, ist ein Routingprotokoll
f"ur mobile Ad-hoc-Netze, das eine an die Anforderungen eines mobilen
drahtlosen LANs angepasste Version des Link State Routing darstellt. Es
wurde von der IETF mit dem RFC 3626 standardisiert. Bei diesem
verteilten flexiblen Routingverfahren ist allen Routern die vollst"andige
Netztopologie bekannt, sodass sie von Fall zu Fall den k"urzesten Weg zum
Ziel festlegen k"onnen. Als proaktives Routingprotokoll h"alt es die daf"ur
ben"otigten Informationen jederzeit bereit.   Ein in Mesh-Netzwerken
bekannter Vertreter von LSR ist OLSR von olsr.org. Inzwischen existieren
f"ur OLSR spezielle Erweiterungen. Mit der ETX-Erweiterung wird dem
Umstand Rechnung getragen, dass Links asymmetrisch sein k"onnen. Mit
dem Fisheye-Algorithmus ist OLSR auch fur gro"sere Netzwerke brauchbar
geworden, da Routen zu weiter entfernten Knoten weniger h"aufig neu
berechnet werden. Der entscheidende Nachteil ist aber der trotz
Fisheye-Algorithmus noch recht hohe Rechenaufwand von OLSRD, sobald
die Anzahl an Knoten ein gewisses Ma"s ubersteigt.

\subsubsection{B.A.T.M.A.N. (BETTER APPROACH TO MOBILE ADHOC NETWORKING)}

Ausgehend von den Erfahrungen mit Freifunk-OLSR begannen die Entwickler
aus der Freifunk-Community im M"arz 2006 in Berlin damit, ein neues
Routingprotokoll f"ur drahtlose Meshnetzwerke zu entwickeln. Alle bisher
bekannten Routingalgorithmen versuchen Routen entweder zu berechnen
(proaktive Verfahren) oder sie dann zu suchen, wenn sie gebraucht werden
(reaktive Verfahren). Das neue Protokoll B.A.T.M.A.N. berechnet oder
sucht im Gegensatz zu diesen Protokollen keine Routen, es erfasst
lediglich, ob Routen zu anderen Knoten existieren und "uberwacht ihre
Qualit"at. Dabei interessiert es sich nicht daf"ur, wie eine Route verl"auft,
sondern ermittelt lediglich, "uber welchen direkten Nachbarn ein bestimmter
Netzwerkknoten am besten zu erreichen ist, und tr"agt diese Information
proaktiv in die Routingtabelle ein. 

\subsection{OpenWRT}

OpenWRT ist eine GNU/Linux-Distribution f"ur WLAN-Router. Anstatt einer
statischen Firmware setzt OpenWRT auf ein voll beschreibbares Dateisystem
sowie einen Paketmanager. OpenWRT l"auft unter anderem auf Ger"aten der
Firmen Linksys, ALLNET, ASUS, Belkin, Buffalo, Microsoft und Siemens.

Vorteile:
\begin{itemize}
\item Flexibilit"at
\item Erweiterbarkeit
\item Individualisierbarkeit
\item Sicherheit
\item Gewohnte Linux-Flexibilit"at und Funktionsumfang!!! 
\end{itemize}

Nachteile:
\begin{itemize}
\item Standardm"a"sig sind nur die n"otigsten Unix-Tools vorhanden 
\end{itemize}

Links:
\begin{itemize}
\item \url{http://openwrt.org/}
\item \url{http://toh.openwrt.org/}
\end{itemize}
